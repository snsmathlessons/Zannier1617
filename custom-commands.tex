\usepackage{xifthen}
%\usepackage{xparse}
%\usepackage{etoolbox}% http://ctan.org/pkg/etoolbox
%\usepackage[linktoc=all]{hyperref} % Per i collegamenti ipertestuali. [linktoc=all] mette i link nell'indice

% https://it.sharelatex.com/blog/2011/03/27/how-to-write-a-latex-class-file-and-design-your-own-cv.html

\newcommand{\frdx}{ \framebox[\width]{ $\Rightarrow$ } }
\newcommand{\frsx}{ \framebox[\width]{ $\Leftarrow$ } }
\newcommand{\hrar}{\hookrightarrow}
\newcommand{\opp}{\text{ oppure }}
\def\checkmark{\tikz\fill[scale=0.4](0,.35) -- (.25,0) -- (1,.7) -- (.25,.15) -- cycle;} 
\newcommand{\crossmark}{$\times$}

%mathbb mathcal mathfrak e mathbm per le lettere dell'alfabeto e anche mathbb per quelle greche
\def\mydeflett#1{\expandafter\def\csname bb#1\endcsname{\mathbb{#1}}
		\expandafter\def\csname c#1\endcsname{\mathcal{#1}}
		\expandafter\def\csname k#1\endcsname{\mathfrak{#1}}
		\expandafter\def\csname bl#1\endcsname{\mathbf{#1}}}
\def\mydefalllett#1{\ifx#1\mydefalllett\else\mydeflett#1\expandafter\mydefalllett\fi}
\mydefalllett ABCDEFGHIJKLMNOPQRSTUVWXYZ\mydefalllett

\def\mydeffrakmath#1{\expandafter\def\csname k#1\endcsname{\mathfrak{#1}}}
\def\mydefallfrak#1{\ifx#1\mydefallfrak\else\mydeffrakmath#1\expandafter\mydefallfrak\fi}
\mydefallfrak abcdefghijklmnopqrstuvwxyz\mydefallfrak

\def\mydefgreek#1{\expandafter\def\csname bl#1\endcsname{\text{\boldmath$\mathbf{\csname #1\endcsname}$}}}
\def\mydefallgreek#1{\ifx\mydefallgreek#1\else\mydefgreek{#1}%
   \lowercase{\mydefgreek{#1}}\expandafter\mydefallgreek\fi}
\mydefallgreek {Gamma}{Delta}{Theta}{Lambda}{Xi}{Pi}{Sigma}{Upsilon}{Phi}{Varphi}{Psi}{Omega}{alpha}{beta}{gamma}{delta}{epsilon}{varepsilon}{zeta}{eta}{theta}{iota}{kappa}{lambda}{mu}{nu}{xi}{omicron}{pi}{rho}{sigma}{tau}{upsilon}{phi}{varphi}{chi}{psi}{omega}\mydefallgreek

\newcommand{\de}{\text{ d}}

%nuovi comandi per svariate cose (alcune per evitare di scrivere
\newcommand{\sse}{\Leftrightarrow}
\newcommand{\Rar}{\Rightarrow}
\newcommand{\rar}{\rightarrow}
\newcommand{\ol}[1]{\overline{#1}}
\newcommand{\ot}[1]{\widetilde{#1}}
\newcommand{\oc}[1]{\widehat{#1}}
\renewcommand{\hat}{\widehat}
\newcommand{\tc}{\text{ t.c. }}
\newcommand{\nowlog}{ Senza perdità di generalità }
\newcommand{\wlogsupp}{ Senza perdità di generalità possiamo supporre }
\newcommand{\spa}{ Supponiamo per assurdo }

% Ti supplico, cambiali
\newcommand{\norma}[1]{\mid\mid #1 \mid\mid}
\newcommand{\abs}[1]{\mid #1 \mid}
\newcommand{\scal}[2]{\langle #1 \mid #2 \rangle}
\newcommand{\floor}[1]{\lfloor #1 \rfloor}


\newcommand{\chr}{\text{char }}
\newcommand{\Ker}{\mbox{Ker } }
\newcommand{\Deg}{\mbox{deg }}
\newcommand{\Det}{\mbox{det }}
\newcommand{\Dim}{\mbox{dim }}
\newcommand{\End}{\mbox{End }}
\newcommand{\Rad}{\mbox{Rad }}
\newcommand{\Ann}{\mbox{Ann }}
\newcommand{\Sp}{\mbox{Sp }}
\renewcommand{\Im}{\kI\km\,}
\renewcommand{\Re}{\kR\ke\,}
\newcommand{\Rk}{\mbox{rk }}
\newcommand{\Tr}{\text{tr }}
\newcommand{\Aut}{\text{Aut }}
\newcommand{\GL}{\text{GL}}
\newcommand{\Omeo}{\text{Omeo }}
\newcommand{\Fix}{\text{Fix }}
\newcommand{\Img}{\text{Im }}
\newcommand{\Supp}{\text{Supp }}
\newcommand{\Span}{\text{Span }}
\newcommand{\Id}{\text{Id}}

% Cusu-comandi
\newcommand{\sdR}{superficie di Riemann}
\newcommand{\quotient}[2]{\left.\raisebox{.2em}{$#1$}\middle/\raisebox{-.2em}{$#2$}\right.}
\newcommand{\Lar}{\Leftarrow}
\newcommand{\lar}{\leftarrow}
\newcommand{\isom}{\simeq}
% Il meno insiemistico
\newcommand{\minus}{\backslash}
\newcommand{\dst}{\displaystyle}
\newcommand{\matrice}[4]{\left(\begin{array}{cc} #1 & #2 \\ #3 & #4 \end{array} \right)}

% Comando per plottare una curva algebrica
% Nell'ordine i parametri sono:
% 1. Equazione della curva (in due variabili)
% 2. xrange (in notazione [inizio:fine])
% 3. yrange (come sopra)
% DAMETTERE 4. opzioni (tipo se metterci la griglia o no)
\newcommand{\curvegraph}[3]{
    \begin{tikzpicture}
    \draw[very thin,color=gray] (-1.9,-3.9) grid (3.9,3.9);
    \draw[->] (-2,0) -- (4.2,0) node[right] {$x$};
    \draw[->] (0,-4.2) -- (0,4.2) node[above] {$y$};
    \draw plot[id=curve, raw gnuplot, smooth] function{
    f(x,y) = #1;
    set xrange #2;
    set yrange #3;
    set view 0,0;
    set isosample 1000,1000;
    set size square;
    set cont base;
    set cntrparam levels incre 0,0.1,0;
    unset surface;
    splot f(x,y);
    };
    \end{tikzpicture}
}

\sloppy

