\chapter{31/01/17 - Azione di $SL_2 \left( \mathbb{Z} \right)$ su $\mathcal{H}$ e Funzioni Modulari}
\justify

\section{Azione di $SL_2 \left( \mathbb{Z} \right)$ su $\mathcal{H}$}

Siano $F=\left\{z \in \mathcal{H} | -\frac{1}{2} \leq \Re(z) \leq 0
\wedge |z| \geq 1 \right\} \cup
\left\{z \in \mathcal{H} | 0 < \Re(z) < \frac{1}{2} \wedge |z| > 1 \right\}$,
$G=SL_2 \left( \mathbb{Z} \right)/\{\pm Id\}$,   $\rho=\frac{-1+i\sqrt{3}}{2}$,
$S=\left( \begin{array}{cc} 0 & 1 \\ -1 & 0 \end{array} \right) \in G$,
$T=\left( \begin{array}{cc} 1 & 1 \\ 0 & 1 \end{array} \right) \in G$.

\begin{teorema}[1]
$\forall z \in \mathcal{H} \  \exists g \in G$ tale che $gz \in \mathcal{H}$.
Inoltre:
\begin{itemize}
	\item se $z,gz \in \bar{F}$, allora $z \in \partial F$ e $g=S$ oppure $g=T$.
	\item se $z \in F$ e $z=gz$,
				allora $z=i$ e $g \in \left\langle S \right\rangle$ oppure
				$z= \rho$ e $g \in \left\langle ST \right\rangle$
\end{itemize}
\end{teorema}

\begin{corollario}
F è un dominio fondamentale.

$Stab(i)=\left\langle S \right\rangle$, 
$Stab(\rho)=\left\langle ST \right\rangle$ e $Stab(z)=\emptyset$ se 
$z \notin \{i, \rho \}$.
\end{corollario}

\begin{teorema}
G è generato da S e T
\end{teorema}

\notamargine{In realtà si potrebbe dimostrare anche che G è il gruppo libero
generato da $S$ e $T$ modulo le relazioni $S^2=Id$ e $(TS)^3=Id$}

\begin{proof}
Sia $G'=\left\langle S,T \right\rangle$. Sia $z \in \mathcal{H}$ e $z' \in G'z$ l'elemento con $\Im(z')$ massimo possibile. Tale elemento esiste perché se
$\sigma = \left( \begin{array}{cc} a & b \\ c & d \end{array} \right) \in G'$,
allora $\Im(\sigma z') = \frac{\Im(z')}{|cz'+d|^2}$ ed essendo
$c,d \in \mathbb{Z}$, $|cz'+d|^2 \geq \Im(z')^2$ (per $c \neq 0$, se no fa 1) per cui $\Im(\sigma z')$ è limitato dall'alto e il sup viene raggiunto perché
ci sono solo un numero finito di $c,d \in \mathbb{Z}$ per cui
$|cz'+d|^2 \leq 1$.
Possiamo anche supporre che $z=z'$ e, a meno di comporre con qualche potenza
di $T$, supponiamo anche che $-\frac{1}{2} \leq \Re(z) < \frac{1}{2}$.

Verifichiamo quindi che $|z| \geq 1$ (se poi fosse $|z|=1$ e
$z \in \bar{F} \setminus F$ basta applicarci $S$).
$\Im(Sz) = \frac{\Im(z)}{|z|^2} \leq \Im(z)$ per la massimalità di $\Im(z)$.
Quindi $|z| \geq 1$.


Siano ora $g \in G$, $z'=gz$ e supponiamo che $z,z' \in \bar{F}$ e (wlog)
che $\Im(z) \leq \Im(z')$.
$$
z'= \frac{az+b}{cz+d} \frac{c \bar{z} +d}{c \bar{z} +d} =
\frac{ac|z|^2 + bd + bc \bar{z} + adz}{|cz+d|^2} \stackrel{det=1}{=}
\frac{ac|z|^2 + bd + bc \bar{z} + (bc+1)z}{|cz+d|^2} =
$$
$$
\frac{ac|z|^2 + bd + bc (\bar{z} + z) + z}{|cz+d|^2} =
\frac{ac|z|^2 + bd + (2bc+1) \Re(z)}{|cz+d|^2} + i \frac{\Im(z)}{|cz+d|^2}
$$
Ora, usando che $z,z' \in \bar{F}$, per cui $|z|,|z'| \geq 1$ e
$|\Re(z)|, |\Re(z')| \leq \frac{1}{2}$, si ottiene che, se $c \neq 0$,
$\Im(z') = \frac{\Im(z)}{|cz+d|^2} \leq \frac{\Im(z)}{c^2 \Im(z)^2}
= \frac{1}{c^2 \Im(z)} \leq \frac{2}{c^2 \sqrt{3}}$, quindi $|c| \leq 1$.

Se $c=0$, allora deve essere $z'=z+b$, per cui $z'=z$ e $g=Id$, oppure
$|\Re(z')| = \frac{1}{2}$ e $|\Re(z)| = -\frac{1}{2}$ o viceversa,
cioè $z,z' \in \partial F$ e $g=T$.

Se invece $c=1$ (il caso $c=-1$ è uguale perché
$G=SL_2 \left( \mathbb{Z} \right)/\{\pm Id\}$),
$\Im(z') \leq \frac{1}{\Im(z)} \Rightarrow \Im(z) \leq \frac{2}{\sqrt{3}}$.
Allora $\Im(z') \leq \frac{2}{\sqrt{3}|z+d|^2} \Rightarrow
|z+d|^2 \leq \frac{4}{3} \Rightarrow |z|^2+d^2+2d \Re(z) \leq \frac{4}{3}$. 
Quindi $d=0$ o $d=\pm 1$.

Se $d=0$, $\Im(z') = \frac{\Im(z)}{|z|^2} \leq \Im(z)$, quindi $|z|=1$.

Se $d=\pm 1$, similmente si ottiene $|z \pm 1|=1$, da cui $z=\rho$ e
$z'=\rho +1$ o viceversa.

\bigskip
Vediamo ora che $G'=G$:

Sia $z=2i$ e sia $g \in G$. Per quanto visto sopra, $\exists \sigma \in G'$
tale che $\sigma g(z) \in \bar{F}$. Allora, dato che $z \notin \partial F$,
$\sigma g(z)=z$.
Verifichiamo quindi che gli unici stabilizzatori non banali sono quelli
previsti dal corollario, così potremo dedurre che $\sigma g = Id$ e quindi
$G=G'$.

Abbiamo visto che se $z=gz$ con $z \in F$ e $g \in G$, deve essere
$|cz+d|=1$ e $c \in \{ 0,1,-1 \}$.

Se $c=0$, $d=\pm 1$ e, dato che $det(g)=1$, $a=d$. Quindi
$g=\left( \begin{array}{cc} 1 & b \\ 0 & 1 \end{array} \right)$ che non ha
punti fissi per $b\neq 0$ ed è l'identità per $b=0$.

Se $c=1$, $|z+d|=1$ da cui $d \in \{0,1,-1 \}$. Se $d=0$, dal determinante
si ottiene $b=-1$, quindi $gz=\frac{az-1}{z}=z$. Risolvendo in $z$ si trova
che le uniche soluzioni contenute in $F$ sono $z=i$ (per $a=0$ e quindi $g=S$)
e $z=\rho$ (per a=-1 e quindi $g=(ST)^2$).
Se $d=1$, $|z+1|=1$ che ha $z=\rho$ come unica soluzione in $F$. Risolvendo il
sistema dato da $\frac{a \rho +b}{\rho +1}=\rho$ e $a-b=1$, si ottiene $g=ST$.
Se $d=1$, $|z-1|=1$ che non ha soluzioni in $F$.
(Questi ultimi 2 casi lui li ha lasciati per esercizio)
\end{proof}

\begin{osservazione}
Al quoziente $\mathcal{H}/G$ si può dare una struttura di superficie di Riemann
(che non è quella che viene da $\mathbb{C}$),
identificando le rette $\Re(z)=\pm \frac{1}{2}$ e i due archi di circonferenza
sul bordo di $F$. Il quoziente è omeomorfo a $\mathbb{C}$.
\end{osservazione}

\section{Forme quadratiche binarie intere}
Una forma quadratica binaria intera è un'espressione del tipo:
$$ ax^2+bxy+cy^2 \qquad a,b,c,d \in \mathbb{Z} $$
Si vogliono classificare a meno di equivalenza lineare con elementi di
$SL_2 \left( \mathbb{Z} \right)$.

\begin{osservazione}
Il discriminante $\Delta =b^2-4ac$ è invariante per trasformazioni lineari
invertibili. Inoltre, fissato $\Delta$, il numero di classi di equivalenza con
quel discriminante è FINITO (questo però è difficile).
Quando $\Delta < 0$, è utile considerare $\xi \in \mathcal{H}$ che risolve
$a \xi^2 + b \xi + c =0$. Per quanto abbiamo visto, $\exists \sigma \in G$
tale che $\sigma \xi \in F$. Tramite $\sigma$ si ottiene la forma ridotta secondo
Gauss.
\end{osservazione}

\section{Funzioni Modulari}

\begin{definizione}
Una funzione $f$ meromorfa su $\mathcal{H}$ di dice debolmente modulare di peso $2k$ (per $k \in \mathbb{N}$), se $\forall
\left( \begin{array}{cc} a & b \\ c & d \end{array} \right) \in
SL_2 \left( \mathbb{Z} \right)$ si ha:
$$ f\left( \frac{az+b}{cz+d} \right) = (cz+d)^{2k} f(z)
\qquad \forall z \in \mathbb{C}$$
\end{definizione}

\begin{osservazione}
$\displaystyle{gz=\frac{az+b}{cz+d} \rightarrow
\frac{dgz}{dz}=\frac{1}{(cz+d)^2}}$. Quindi la condizione della definizione di
funzione debolmente modulare può essere scritta come $f(gz)(d(gz))^k=f(z)(dz)^k$.
Sono k-forme differenziali.
\end{osservazione}

\begin{osservazione}
Dalla definizione segue immediatamente che $f(z+1)=f(z)$, cioè che una
funzione debolmente modulare è periodica di periodo $1$. Quindi $f$ si può scrivere in serie di Fourier in $q=e^{2 \pi i z}$, cioè
$\displaystyle{f(z)=\widetilde{f}(q)=\sum^{+\infty}_{n=-\infty}{a_n q^n}}$.
Inoltre $q$ manda $\mathcal{H}$ in $D^*$, quindi $\widetilde{f}$ è meromorfa
in $D^*$ e $\displaystyle{\sum^{+\infty}_{n=-\infty}{a_n q^n}}$ è la sua
serie di Laurent.
\end{osservazione}

\begin{definizione}
Nelle notazioni di sopra, se $\widetilde{f}$ è meromorfa anche in $D$, cioè
se gli $a_n$ con indice $n<0$ non nulli sono in numero finito, la $f$
è "meromorfa all'$\infty$" e si dice funzione modulare.

Se poi $\widetilde{f}$ è olomorfa in $0$, cioè se tutti gli $a_n$ con $n<0$
sono nulli, la $f$ è "olomorfa all'$\infty$" e si dice forma modulare.

Infine, se anche $a_0=0$, cioè $f(\infty)=0$, $f$ si dice forma modulare
cuspidale
\end{definizione}

\section{Esempio: Le Serie di Eisenstein}

\begin{definizione}
Se $L$ è un reticolo in $\mathbb{C}$, per $k \geq 2$ poniamo
$G_k(L):=\displaystyle{\sum_{\lambda \in L \setminus \{ 0 \}}{\lambda ^{-2k}}}$.
\end{definizione}

\begin{osservazione}
Proprietà di $G_k$:

\begin{itemize}
\item Le $G_k$ sono $(-2k)$-omogenee, cioè
$L_1=cL_2 \Rightarrow G_k(L_1)=c^{-2k}G_k(L_2)$.
Scrivendo $L=\mathbb{Z} \tau + \mathbb{Z}$, con $\tau \in \mathcal{H}$,
si può anche vedere $G_k(L) = G_k(\tau)$ come funzione su $\mathcal{H}$.
In questo modo, $\displaystyle{G_k(z)=\sum_{(m,n) \in \mathbb{Z}^2 \setminus
(0,0)}{\frac{1}{(mz+n)^{2k}}}}$, che converge uniformemente sui compatti
di $\mathcal{H}$.
\item $G_k(z)$ converge puntualmente su $\mathcal{H}$ e su $-\mathcal{H}$, ma non su tutto $\mathbb{C}$. Infatti se $z \in \mathbb{R}$ i denominatori possono
essere arbitrariamente vicini a $0$ e quindi la serie diverge.
\item Ricordando la definizione di $g_2$ e $g_3$ \textbf{\textcolor[rgb]{1,0,0}{(qui si potrebbe mettere il riferimento alla definizione)}}, si ha: $g_2(z)=60G_2(z)$ e $g_3(z)=140G_3(z)$.
\item Se $\left( \begin{array}{cc} a & b \\ c & d \end{array} \right) \in
SL_2 \left( \mathbb{Z} \right)$,
$\displaystyle{ G_k \left(\frac{az+b}{cz+d} \right) = 
\sum_{(m,n) \in \mathbb{Z}^2 \setminus (0,0)}
{\frac{(cz+d)^{2k}} {(m(az+b)+n(cz+d))^{2k}}} =}$
$\displaystyle{ =(cz+d)^{2k} \sum_{m,n}{\frac{1} {(m(az+b)+n(cz+d))^{2k}}} =
(cz+d)^{2k} \sum_{m,n}{\frac{1}{(mz+n)^{2k}}} }$, perché
$SL_2 \left( \mathbb{Z} \right)$ lascia invariati i reticoli. Quindi $G_k$
è debolmente modulare.
\end{itemize}
\end{osservazione}


\begin{proposizione}
Le $G_k$ sono forme modulari.
\end{proposizione}

\begin{proof}
Sono tutte olomorfe su $\mathcal{H}$ per teoremi classici di convergenza.
Se $G_k$ avesse un polo o un sigolarità essenziale all'$\infty$, ci sarebbero
delle successioni $\{ z_n \} \subset \mathcal{H}$ tali che
$|z_n| \rightarrow +\infty$ e $|G_k(z_n)| \rightarrow +\infty$.
Ma $\displaystyle{\lim_{\Im(z) \rightarrow +\infty} G_k(z)=
2 \sum_{n=1}^{+\infty}{\frac{1}{n^{2k}}} = 2 \zeta(2k)}$, perché i termini
con $m \neq 0$ vanno a $0$ uniformemente. Quindi le $G_k$ sono olomorfe
all'$\infty$.
\end{proof}

\begin{osservazione}
$\Delta = g_2 ^3 (z) - 27 g_3 ^2 (z)$ è una forma modulare di peso $12$ che
non si annulla mai in $\mathcal{H}$.
\end{osservazione}