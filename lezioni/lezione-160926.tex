\chapter{26 Settembre 2016} %TODO: Trovare un titolo.
%TODO: Abbassare le notamargine...

\section{Preliminari ed esempi}
% TODO: definizione giusta di superficie di Riemann: forse è meglio sia direttamente nelle lezioni precedenti.
In questa lezione discuteremo di alcuni esempi di superfici di Riemann, e cominceremo ad introdurre alcuni risultati che ci aiuteranno a classificare specifiche classi di superfici di Riemann a meno di isomorfismo.



\textbf{Esempi di superfici di Riemann compatte:}
\begin{enumerate}
  \item $\bbC$ (o un qualsiasi aperto connesso di $\bbC$). In questo caso l'unica carta è tutto l'insieme, con la mappa di immersione.
  \item La sfera di Riemann $\bbP^1(\bbC)$. $\bbP^1(\bbC)=U_0\cup U_1$ dove $U_0=\{(x_0:x_1)\in \bbP^1(\bbC) | \ x_0\neq 0\}$ e $U_1=\{(x_0:x_1)\in \bbP^1(\bbC) | \ x_1\neq 0\}$, con le mappe $\varphi_0(x_0:x_1)=\frac{x_1}{x_0}$ e $\varphi_1(x_0:x_1)=\frac{x_0}{x_1}$.
  \item Curve algebriche non singolari in $\bbP^2(\bbC)$ ($\{(x:y:z)\in \bbP^2(\bbC) | \ f(x,y,z)=0\}$, dove $f\in \bbC[x,y,z]$ è un polinomio irriducibile. %TODO Dimstrazione.
\end{enumerate}
\notamargine{ACHTUNG! Manca la dimostrazione che le curve algebriche non singolari sono superfici di Riemann.}
\begin{osservazione}
    In $\bbP^2(\bbC)$ vale che ogni curva definita da un polinomio riducibile (che non sia una potenza di un irriducibile) è singolare. Infatti, detto $f(x,y,z)=p(x,y,z)\cdot q(x,y,z)$, se $p$ e $q$ non sono una potenza di uno stesso polinomio irriducibile, deve esistere un punto isolato in $\{p(x,y,z)=0\}\cap\{q(x,y,z)=0\}\subseteq \bbP^2(\bbC)$. In questo punto la curva è formata da due "bracci" che si intersecano, e quindi non è localmente esprimibile come grafico, pertanto necessariamente entrambe le derivate parziali si annullano, dunque è singolare. %TODO qui ci starebbe molto bene un disegnino (magari nelle note).
\end{osservazione}
Durante il corso, vorremmo arrivare a questo teorema:
\begin{teorema}[di Chow]
    Gli esempi precedenti costituiscono tutti gli esempi di superfici di Riemann compatte immerse in $\bbP^2(\bbC)$.
\end{teorema}
\begin{osservazione}
    Una superficie di Riemann meno un numero finito di punti resta una superficie di Riemann. Questo è dovuto al fatto che un aperto connesso di $\bbC$ meno un punto resta un aperto connesso.
\end{osservazione}
%TODO Curve algebriche singolari meno i punti singolari


\textbf{Esempi di superfici di Riemann non compatte:}
\begin{enumerate} %TODO Far partire l'indice da 4
  \item $\bbC/\bbZ$.
  \item $\bbC/L$, dove $L$ è un reticolo (discreto) di rango 2.
\end{enumerate}
\begin{proof} \textit{(che sono superfici di Riemann)}
    Dimostriamo solo che $\bbC/\bbZ$ lo è, la dimostrazione per $\bbC/L$ è analoga. Considero $\pi:\bbC\rightarrow\bbC/\bbZ$ la proiezione al quoziente, e $Y:=\{z\in\bbC|\ 0\leq Re(z)<1\}$ una striscia verticale di rappresentanti. Ricopro $Y\subseteq\bbC$ con dischi $D_\alpha$ (aperti) di raggio $1$ (in modo che $D_\alpha$ non contenga mai due punti che al quoziente sono uguali). Si osserva facilmente che $\pi_{|D_\alpha}$ è un omeomorfismo, scegliamo $\varphi_\alpha=\pi_{|D_\alpha}^{-1}$, si verifica facilmente soddisfare le proprietà richieste dalla definizione.

    Nel caso del reticolo di rango $2$, la dimostrazione si fa prendendo come $Y$ un parallelogrammo, e raggio dei dischi abbastanza piccolo da impedire che ci possano essere due punti equivalenti nello stesso disco.
\end{proof}

\begin{definizione}
Siano $X$ e $Y$ due superfici di Riemann, $x_0\in X$, $y_0=f(x_0)$. $f:X\rightarrow Y$ si dice olomorfa in $x_0$ se esiste un intorno $A$ di $x_0$ tale che $A\subseteq U_\alpha$ e detto $V_\beta$ un aperto del ricoprimento di $Y$ che contiene $y_0$, vale che $\psi_\beta \circ f \circ \varphi_\alpha^{-1}: \varphi_\alpha(A)\rightarrow \bbC$ è una funzione olomorfa.
\end{definizione}

\notamargine{Più precisamente si dovrebbe richiedere che per OGNI $U_\alpha$ e $V_\beta$ che contengono $x_0$ o $y_0$ valga quella proprietà, ma questo è equivalente alla definizione data per la proprietà di compatibilità sulle intersezioni delle $\varphi_\alpha$ e $\psi_\beta$.}

\begin{definizione}
Due superfici di Riemann $X$ e $Y$ si dicono isomorfe (o conformemente equivalenti) se esiste $f:X\rightarrow Y$ invertibile, olomorfa con inversa olomorfa.
\end{definizione}

\notamargine{Una tale $f$ viene definita biolomorfismo.}

\begin{osservazione}
Sia $X$ una superficie di Riemann, $Y$ uno spazio topologico, $f:X\rightarrow Y$ un omeomorfismo. Allora posso trasportare su $Y$ la struttura complessa di $X$, ricoprendolo con aperti $V_\alpha=f(U_\alpha)$ e mappe $\psi_\alpha=\varphi_\alpha \circ f^{-1}_{|f(U_\alpha)}:f(U_\alpha)\rightarrow\bbC$
\end{osservazione}

\notamargine{Con questa struttura di varietà, chiaramente $Y$ è isomorfo ad $X$}

\section{Classificazione delle superfici di Riemann}
Cerchiamo di muoverci verso un risultato riguardo la classificazione delle superfici di Riemann. Lo schema con cui affronteremo il problema consiste nel considerare un rivestimento universale della superficie di Riemann e cercare esprimere la superficie di partenza come quoziente dello spazio rivestente per un gruppo di automorfismi. In questo modo sposteremo il problema sullo studio dei sottogruppi del gruppo di automorfismi di (speriamo poche) superfici fissate.
\begin{osservazione}
Sia $X$ una superficie di Riemann, se $\pi:Y\rightarrow X$ è un rivestimento, allora $Y$ è in modo naturale una superficie di Riemann.
\end{osservazione} 
\begin{proof}[Idea della dimostrazione]
È possibile fare un raffinamento degli $U_\alpha\subseteq X$ in modo da renderli "compatibili" con gli aperti banalizzanti del rivestimento (per esempio, posso considerare le intersezioni con essi). In questo modo, avendo gli $U_\alpha$ inclusi in un aperto banalizzante, è possibile "tirarli su" sullo spazio ricoprente, e definire le $psi_\alpha=\phi_alpha \circ \pi$. La verifica che sono rispettate le proprietà della definizione è banale.
\end{proof}
Enunciamo adesso il risultato chiave che ci permetterà di procedere con la classificazione:
\begin{teorema}[di Riemann]
Ogni superficie di Riemann semplicemente connessa è biolomorfa ad uno dei seguenti tre modelli:
\begin{enumerate}
  \item La sfera di Riemann $\bbP^1(\bbC)=:\widehat{\bbC}$.
  \item Il piano complesso $\bbC$.
  \item Il disco di Poincaré $D$.
\end{enumerate}
\end{teorema} 
\begin{proof}
La dimostrazione non verrà trattata in questo corso a causa dell'eccessiva difficoltà.
Nella prossima lezione vedremo che queste tre superfici non sono biolomorfe (è una conseguenza del teorema di Liouville.
\end{proof}
\begin{osservazione}
Sia $\pi:\widetilde{X}\rightarrow X$ un rivestimento universale, $x_0\in X$. Allora la fibra $\pi^{-1}(x_0)$ è discreta in $\widetilde{X}$, ed esiste un gruppo di omeomorfismi di $\widetilde{X}$ che preserva le fibre ed agisce in modo transitivo su di esse. Chiamato $G$ tale gruppo, si ha che $X \simeq \widetilde{X}/G$.
\end{osservazione}
\begin{definizione}
Sia $G$ un gruppo che agisce su uno spazio topologico $X$. L'azione di $G$ si dice \textit{propriamente discontinua} se per ogni compatto $K\subseteq X$ esiste solo un numero finito di elementi di $G$ tali che $g(K)\cap K =\varnothing$.
\end{definizione}
\begin{fatto}
Sia $X$ una superficie di Riemann con un gruppo di automorfismi olomorfi $G<Aut(X)$. Allora $X/G$ è in modo naturale una superficie di Riemann se valgono le seguenti due proprietà:
\begin{enumerate}
  \item L'azione di $G$ su $X$ è propriamente discontinua.
  \item Gli elementi di $G\setminus\{\Id\}$ agiscono senza punti fissi.
\end{enumerate}
\end{fatto}
\notamargine{Se non sono soddisfatte le due condizioni, allora $X\rightarrow X/G$ non è nemmeno un rivestimento, a prescindere dalla struttura complessa.} 