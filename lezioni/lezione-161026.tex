\chapter{26/10/2016 - Delle Cose}
	Abbiamo distinto le superfici di Riemann in tre tipi:
	\begin{itemize}
	 \item La superficie sferica $\hat\bbC$;
	 \item Le superfici euclidee, ottenute quozientando $\bbC$;
	 \item Le superfici iperboliche, ottenute da quozienti del disco $D$.
	\end{itemize}

	\begin{proposizione}
		Ogni mappa biolomorfa da una superficie sferica a una euclidea è costante. Allo stesso modo, ogni mappa biolomorfa da una superficie euclidea a una iperbolica è costante.
	\end{proposizione}
	
	\begin{proof}
		Siano $S_1$, $S_2$ e $S_3$ tre superfici, rispettivamente sferica, euclidea e iperbolica, e siano $f: S_1 \rar S_2$, $g: S_2 \rar S_3$ mappe biolomorfe, come nel diagramma.
		
		% INSERIRE DIAGRAMMA
		
		Poichè sia $\hat\bbC$ sia $\bbC$ sono semplicemente connessi, per il solito lemma $f \circ \pi_1$ e $g \circ \pi_2$ si sollevano a $\tilde f, \tilde g$. 
		
		Ma $\tilde f$, ristretta a $\bbC$, è una mappa intera (olomorfa su $\bbC$) limitata (perchè all'infinito deve tendere ad $f(\infty)$). Quindi per Liouville è costante.
		Stesso discorso vale per $\tilde g$, che è olomorfa su $\bbC$ ed è limitata (ha immagine contenuta nel disco). Quindi anche $f$ e $g$ devono essere costanti.
		
		
	\end{proof}

