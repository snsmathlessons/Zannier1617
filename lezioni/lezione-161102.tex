\chapter{02 Novembre 2016 - Isogenie dei Tori e Anello degli Endomorfismi}
\justify

\newthought{Mostriamo}, in breve, che il $\bbP_{1}\bbC $ non ammette strutture di gruppo olomorfo: supponiamo infatti per assurdo che $\exists \star: \bbP^1\bbC \times \bbP^1\bbC \rar \bbP^1\bbC$ legge di gruppo (ed olomorfa). Allora le parziali $L_x$ definita da $z \mapsto x \star z \in \Aut(\bbP^1\bbC)$ hanno tutte un punto fisso (visto nelle lezioni precedenti); ma se $L_x(y) = y$ deve essere (per cancellazione) che $x = e$, l'identità, per ogni $x$. Si ottiene quindi l'assurdo, avendo la sfera due punti.

Abbiamo inoltre mostrato che i tori ammettono come unica struttura di gruppo olomorfo quella di gruppo quoziente di $\bbC$ sul reticolo $L$.

\section{Leggi di gruppo su superfici di Riemann}
\begin{osservazione}
Sia G un gruppo olomorfo. $\forall g \in G$ le traslazioni $L_{g}$ e $R_{g}$ sono mappe biolomorfe senza punti fissi. 
\end{osservazione}

\noindent Il nostro obiettivo, in questa sezione, è quello di mostrare come nessuna superficie di Riemann iperbolica ammetta una struttura di gruppo olomorfo. In realtà ci basterà mostrarlo solo per il disco unitario grazie al seguente lemma:
\begin{lemma}
Sia $X$ un gruppo topologico con $\mu : X\times X \rar X$ legge di gruppo. Sia inoltre $p :\widetilde{X} \rar X$ il rivestimento universale.
Allora $\mu$ si solleva ad una $\tilde{\mu} : \widetilde{X} \times \widetilde{X} \rar \widetilde{X}$ che rende $\widetilde{X}$ un gruppo topologico.
\end{lemma}

\begin{proof}

			\begin{diagram}
				\widetilde{X}\times\widetilde{X}	& \rTo^{\widetilde{\mu}} 	& \widetilde{X}	\\
				\dTo<{p \times p}	&					& \dTo>{p}\\
				X\times X				& \rTo^\mu 		& X 
			\end{diagram}

\noindent Sia $e \in X$ l'identità di $X$ e sia $\tilde{e} \in \widetilde{X}$ un punto t.c. $p(\tilde{e})=e$. Prodotto di semplicemente connessi è semplicemente connesso, dunque la mappa $\mu \circ (p \times p)$ solleva ad una mappa $\widetilde{\mu}$ che chiude il diagramma. Basta ora mostrare che $\widetilde{\mu}$ è una legge di moltiplicazione {\it (un po' di verifiche emozionantissime)}.\\

{\bf Elemento neutro:}
	\begin{diagram}
		\{\tilde{e}\}\times\widetilde{X}	& \rTo^{\widetilde{\mu}}_{\pi_{2}} 	& \widetilde{X}	\\
		\dTo<{p \times p}			&					& \dTo>{p}\\
		\{e\}\times X				& \rTo^\mu 				& X 
	\end{diagram}

\noindent Per restrizione, $\tilde{\mu}$ chiude il diagramma. Ma anche la proiezione sul secondo fattore fa commutare il quadrato; per unicità del sollevamento a punto base fissato si ha che $\tilde{\mu}(\tilde{e}, \bullet) \equiv \pi_{2}(\bullet)$. Analogamente si ha che $\tilde{\mu}(\bullet, \tilde{e}) \equiv \pi_{1}(\bullet)$ e dunque $\tilde{e}$ è elemento neutro per $\tilde{\mu}$.
        
{\bf Associatività} e {\bf inverso} si ottengono in maniera del tutto analoga (cioè sfruttando l'unicità del sollevamento).
\notamargine{Per l'associatività utilizzare la diagonale sulla faccia del cubo}
\end{proof}

\begin{teorema}
Il disco unitario in $\bbC$ non ammette una struttura di gruppo olomorfa
\end{teorema}

\begin{proof}
Per assurdo; ci sono due possibili vie.\\
Conoscendo i gruppi di Lie, basta considerare la mappa esponenziale dal tangente nel disco: dovrebbe essere una mappa non costante da $\bbC$ nel disco. \hfill \Lightning \\
Altrimenti, sia $\star : D\times D \rar D$ una legge di gruppo olomorfa. Sia inoltre $x^{*}$ l'inverso di $x$ per $\star$. Senza perdita di generalità, si può supporre che $0 \in D$ sia l'elemento neutro.
Allora $\forall x\in D$ la mappa $L_{x}: z\mapsto x\star z$ è un automorfismo olomorfo. \\
Dunque,  $$x\star z = c(x)\frac{z -a(x)}{1- \overline{a}(x) z}$$
con $|c(x)|\equiv 1$ e $a:D \rar D$.\\
Per $z=0$ si ha $x=-c(x)a(x)$.\\
Per $z=a(x)$ si ha $x*a(x)=0$ $\Rar$ $a(x)=x^{*}$ $\Rar$ $a(x)$ è olomorfa $\Rar$ (per $x\neq 0$) $c(x)=-\frac{x}{x^{*}}$ $\Rar$ $c(x)$ è olomorfa per $x\neq 0$, ma $|c(x)| \equiv 1$ $\Rar$ per il teorema della mappa aperta deve essere $ \ c(x) \equiv k \neq 0$. Da cui $a(x)= -\frac{x}{k}$.\\
Allora, $$x\star 1/2 = k\frac{1-2a(x)}{2-\overline{a}(x)}$$
ma questa non è olomorfa ( $\overline{a}(x)$ non lo è, il resto sì). 

\end{proof}

\begin{corollario}
  Curve algebriche non ellittiche \underline{non} ammettono struttura di gruppo olomorfo.
  Questo fatto sarebbe facilmente vero se sapessimo che le curve algebriche di grado $d \ge 4$ sono necessariamente rivestite dal disco (Il problema è escludere che siano rivestite da $\bbC$)
\end{corollario}

\section{Endomorfismi di un Toro}
\newthought{In questa sezione} vogliamo studiare l'anello delle isogenie di un toro in sè. Ma prima classifichiamo un po' meglio le isogenie tra tori.

\begin{proposizione}
Siano $T_{1}$ e $T_{2}$ due tori.\\
$\exists f:T_{1} \rar T_{2}$ isogenia $\sse$ $\exists \Gamma < T_{1}$ sottogruppo finito t.c. $\sfrac{T_{1}}{\Gamma} \simeq T_{2}$ come gruppi olomorfi. 
\end{proposizione}
\begin{proof}
  \fbox{$\Rar$} È il primo teorema di omomorfismo (ci sarebbe da mostrare che gli isomorfismi sono olomorfi ma ci crediamo).
  
  \fbox{$\Leftarrow$} Sia $\pi:\bbC \rar T_{1} = \sfrac{\bbC}{L}$ la proiezione al quoziente. Allora $\overline{L}=\pi^{-1}(\Gamma)$ è un sottogruppo discreto di $\bbC$ (rimane discreto perche $\Gamma$ è finito). Per teoremi di omomorfismo (di nuovo sarebbe da mostrare l'olomorfia):
$$\sfrac{T_{1}}{\Gamma} \simeq \sfrac{\sfrac{\bbC}{L}}{\Gamma} \simeq \sfrac{\bbC}{\overline{L}}$$
Di conseguenza $\sfrac{T_{1}}{\Gamma}$ è un toro $T_{2}$ e la proiezione al quoziente è isogenia tra $T_{1}$ e $T_{2}$.
\end{proof}

Più avanti definiremo le funzioni di Weierstrass, che ci permetteranno di trasformare i tori in curve algebriche (ellittiche). Le isogenie verranno trasformate in funzioni razionali tra le curve.

\subsection{Anello degli Endomorfismi di un Toro}

Fissiamo un toro $T=\sfrac{\bbC}{L}$ e consideriamo l'insieme $End(T)$ delle isogenie del toro in sè, con in più la mappa nulla.

\begin{proposizione} $End(T)$ ha una naturale struttura di anello (somma tra mappe e prodotto di composizione). Inoltre vale che:
\begin{enumerate}
\item $End(T)$ è un anello commutativo
\item $\bbZ \subseteq End(T)$
\item $End(T) \hookrightarrow \bbC$
\item $End(T) \otimes_{\bbZ} \bbQ$ è un campo ({\it d'ora in poi, se non diversamente specificato, tutti i tensori saranno su $\bbZ$})
\end{enumerate}
\end{proposizione}

\begin{proof} {\it Lui ha dato tutto ciò per buono. Non so quanto sia utile dimostrarlo ma vabbè}
  
3. Abbiamo mostrato precedentemente che ogni isogenia è passaggio al quoziente di un'opportuna moltiplicazione per scalare su $\bbC$. Per gli endomorfismi vale un risultato più forte: lo scalare che induce l'isogenia è unico.
Infatti, siano $\lambda$ e $\mu$ elementi di $\bbC$ che inducono la stessa isogenia $f:T\rar T$. Allora, $\forall x\in \bbC,  \ (\lambda - \mu)x \in L$. Quindi, l'ideale generato da $\lambda -\mu$ deve essere contenuto nel reticolo, che è discreto $\Rar$ $\lambda - \mu = 0$. \'E quindi ben definita la mappa $f \mapsto \mu$ che associa ad ogni endomorfismo lo scalare di cui è passaggio al quoziente. Questa mappa è chiaramente un omomorfismo iniettivo di anelli.

1. ovvia dalla 3. (perché l'immersione è di anelli)

2. Poichè $L$ è uno $\bbZ$-modulo, $\forall n \in \bbZ, nL\subseteq L$. Inoltre, sia $x\in \bbC \setminus L$. Allora $x/n \not\in L$ e si ha che $n[x/n]=[x]$ $\Rar$ la moltipicazione per n è surgettiva sul toro $\Rar$ $\bbZ \subseteq End(T)$

4. Sia $t=\sum\limits_{i=0}^{k} \mu_{i} \otimes \frac{p_{i}}{q_{i}}$. Sia $q$ il minimo comune multiplo dei $q_{i}$ e siano $a_{i} \in \bbZ$ tali che $\forall i, q=q_{i}a_{i}$. Allora,\\
$t=\sum\limits_{i=0}^{k} \mu_{i}\otimes \frac{p_{i}}{q_{i}} = \sum\limits_{i=0}^{k} \mu_{i} \otimes \frac{p_{i}a_{i}}{q}= \sum\limits_{i=0}^{k} a_{i}p_{i}\mu_{i}\otimes \frac{1}{q} = (\sum\limits_{i=0}^{k} a_{i}p_{i}\mu_{i})\otimes \frac{1}{q}$ $\Rar$ ogni tensore è semplice. L'inverso di un tensore $\mu \otimes \frac{p}{q}$ (dove $\mu$ è una mappa non nulla di grado $d$) è il tensore $\hat{\mu} \otimes \frac{q}{pd}$, con $\hat{\mu}$ l'endomorfismo duale di $\mu$ (la moltiplicazione è quella indotta sull'anello tensore dai due anelli $End(T)$ e $\bbQ$. Sui tensori semplici diventa semplicemente la moltiplicazione "componente per componente").
\end{proof}

Volendo, potremmo fare gli stessi discorsi (anello degli endomorfismi etc...) sulle curve ellittiche (che sono definite anche su altri campi). Cosa cambia? In $char \ K >0$ {\bf non } è detto che $End(T)$ sia commutativo (con $T$ curva ellittica).

\noindent Cerchiamo ora di caratterizzare un po' meglio $F=End(T) \otimes \bbQ$. Sia $T=\tau \bbZ + \bbZ$ e $\mu : T \rar T$ l'isogenia indotta dalla moltiplicazione per $\mu$. La condizione di mappare il reticolo in sé si traduce nel sistema:
\begin{equation*}
\begin{cases}
	\mu \tau = a\tau + b\\
	\mu = c\tau + d\\
\end{cases}
\end{equation*}
Con $M=
\begin{pmatrix}
a & b \\
c & d
\end{pmatrix}$, $ \ M\in \mathfrak{M}_{2}(\bbZ)$.\\
Se $\bar{\mu}=\mu$ $\Rar$ $c=0$ $\Rar$ $\mu \in \bbZ$.\\
Altrimenti $\mu$ è quadratico immaginario perchè è autovalore della matrice $M$, ma non razionale.
Quando posso avere i due casi?\\
Si deve avere $\tau = \frac{a\tau + b}{c\tau + d}$ $\Rar$ $c\tau^{2} + (d-a)\tau - b=0$\\
$\tau$ non è immaginario quadratico $\Rar$ $End(T)=\bbZ$.
$\tau$ è immaginario quadratico $\Rar$ $\bbZ \subsetneq End(T)$. Questo fenomeno si chiama {\bf moltiplicazione complessa dei tori}.

Ricordiamo che, data un'isogenia $c$, rappresentata dalla moltiplicazione per $\mu$, e detto $\hat\mu$ il rappresentante dell'isogenia duale, vale la relazione: $\mu \hat\mu = \deg c$. Nel caso particolare in cui un'isogenia sia invertibile, $\deg c = 1$, e necessariamente $\hat\mu = \bar\mu$

\noindent {\bf $\underline{\mbox{Esercizio:}}$} $c_{i}: T_{1} \rar T_{2}$ ($i=1 , 2$) isogenie, $c_{1}+c_{2} \neq 0$ $\Rar$ $\hat{c_{1} + c_{2}}=\hat{c_{1}}+\hat{c_{2}}$.\\
{\bf $\underline{\mbox{Hint:}}$} Basta dimostrarlo nel caso degli endomorfismi, ponendo $\hat{c_{2}}\circ c_{1} \in End(T_{1})$

