\chapter{05 Dicembre 2016 - Campi di funzioni, Estensioni di mappe razionali a mappe olomorfe}
\justify

\section{Campi di funzioni}

Sia $E: y^2=4x^3-ax-b$ una cubica non degenere su $\bbC$.
Siano $x,y:E\rightarrow\bbC$ le funzioni proiezione su $\bbC$, definite sulla cubica.
\begin{definizione}
Indico con $\bbC(t)$ le funzioni razionali in una variabile, quindi $4t^3-ax-b$ è un elemento di $\bbC(t)$, perciò sia $u$ radice quadrata di $f(t)=4t^3-at-b$ e sia $\bbC(t,u)$ l'estensione di $\bbC(t)$ con $u$.\\
Siano inoltre $\bbC(x)$ e con $\bbC(x,y)$ i campi delle funzioni razionali sulle $x$ e $y$ appena definite, che quindi sono funzioni con dominio $E$ e codominio $\bbC$.
\end{definizione}

\begin{osservazione}
Si noti che $\bbC(t)\subseteq\bbC(t,u)$ è un'estensione di grado $2$.\\
Sì noti che $\bbC(x)$ è un sottoinsieme delle funzioni razionali di una variabile, calcolate poi in $x$. In particolare è il sottoinsieme delle funzioni razionali in una variabile con denominatore che non zero se calcolato in $x$, calcolate in $x$. Quindi lo si può pensare come un sottoinsieme delle funzioni razionali non definite in qualche punto.
Analogo per $\bbC(x,y)$.
\end{osservazione}

\begin{proposizione}
$\bbC(x)\cong_{\bbC}\bbC(t)$
\end{proposizione}

\begin{proof}
Considero la mappa $t\mapsto x$ che lascia fisso $\bbC$, si ha che è ben definito poiché l'immagine di una funzione razionale $R(t)$ è una funzione definita tranne in un numero finito di punti, si ha che è surgettivo e omomorfismo di campi, inoltre siccome è non nullo è iniettivo.
\end{proof}

\begin{proposizione}
$\bbC(x,y)\cong_{\bbC}\bbC(t,u)$
\end{proposizione}

\begin{proof}
Considero la mappa $t\mapsto x, u\mapsto y$ che lascia fisso $\bbC$, per vedere che è ben definito bisogna mostrare che se $R(t,u)=S(t,u)$ allora $R(x,y)=S(x,y)$, ma sottraendo mi riduco a mostrare che se $R(t,u)=0$ allora $R(x,y)=0$, ma $R(t,u)$ è $0$ solo se è divisibile per $u^2-f(t)$, ma nell'immagine $y^2-f(x)$ è $0$. Per ESERCIZIO si verifichi che è un omomorfismo e che l'immagine sta dentro $\bbC(x,y)$
\end{proof}

\begin{osservazione}
Si ha che $\bbC(x)\subseteq\bbC(x,y)$ è di grado $2$ quindi gli elementi di $\bbC(x,y)$ si possono rappresentare come $\{ R(x)+y S(x) | R,S \in \bbC(x)\}$.
\end{osservazione}

Ci si chiede se $\bbC(x,y)$ è isomorfo a un certo $\bbC(w)$.
Se lo fosse allora si avrebbe che $x=f(w), y=g(w)$ con $f$ e $g$ razionali, quindi la curva sarebbe parametrizzabile. "Si noti che qui c'è un forte collegamento tra concetti algebrici (campo semplice) e geometrici (parametrizzabilità) (questo si vedrà meglio più avanti nel corso)".

\begin{osservazione}
Non si può usare il teorema dell'elemento primitivo per l'estensione $\bbC\subseteq\bbC(x,y)$ infatti $x$ è trascendente su $\bbC$.
\end{osservazione}

\begin{proposizione}Se la cubica proviene da un toro ( $a=g_2(L), b=g_3(L)$ ) allora $\bbC(x,y)\cong_{\bbC}\bbC(\wp(z),\wp'(z))$.
\end{proposizione}

\begin{proof}
Si noti che quest'ultimo è un campo di funzioni meromorfe, infatti i polinomi nelle $\wp(z)$ e $\wp'(z)$, se non sono nulli, si annullano solo in un numero finito di punti (altrimenti gli zeri hanno un punto di accumulazione). Quindi presi due polinomi $P$ $Q$ calcolati in $\wp(z)$ e $\wp'(z)$ si ha che il loro quoziente $P/Q$ è nullo solo se il denominatore è nullo quindi solo in un numero finito di punti. Per ESERCIZIO si costruisca l'isomorfismo $\bbC[x,y]\rightarrow\bbC[\wp(z),\wp'(z)]$ e poi si passi ai quozienti completando la dimostrazione.
\end{proof}


\section{Estensioni di mappe razionali}
Sia $E'$ un'altra cubica dello stesso tipo di $E$.

\begin{definizione}
Una mappa razionale $E\rightarrow E'$ è una mappa $E\rightarrow E'$ definita su "$\tilde{E}$ meno un numero finito di punti" e che può scrivere in termini razionali di $x$ e $y$, cioè $\varphi(x_0,y_0)=(\varphi_1(x_0,y_0),\varphi_2(x_0,y_0) )$ con $\varphi_1$ e $\varphi_2$ razionali.
\end{definizione}

\begin{osservazione}
Per finire in $E'$ si deve avere che $\varphi_2^2=4\varphi_1^3-a'\varphi_1-b'$, quindi non è facile trovarne nemmeno dalla curva in se stessa. Qualche volta non ce ne sono (a parte le costanti, ovviamente) (vedi quanto scritto sotto per i tori, per aiutarti ad avere un esempio)
\end{osservazione}

Se considero una funzione da un toro in $\bbC$ meromorfa (quindi non definita sul tutto il toro, la chiama così per abbreviare) allora si può estendere ad una funzione olomorfa dal toro verso $\bbP^1(\bbC)$.
Quindi se la curva $E$ proviene da un toro e $E'=\bbC$ posso estendere la mappa razionale a tutta $\tilde{E}$, a valori in $\bbP^1(\bbC)$.
Questo si può fare anche se in arrivo non c'è $\bbC$, ma una qualsiasi $E'$, considerando come codominio della funzione dalla curva estesa $\tilde{E'}$.

\begin{proposizione}
Una mappa razionale $E\rightarrow E'$ si estende ad una mappa olomorfa $\tilde{E} \rightarrow \tilde {E'}$.
\end{proposizione}

\begin{proof}
Basta dimostrare che una mappa razionale $E\rightarrow \bbC$ si estende a $\tilde{E}\rightarrow\bbP^1(\bbC)$ (il caso generale è analogo, ha detto).\\
Ho quindi una mappa da $\tilde{E}$ meno un numero finito di punti a $\bbC$ olomorfa e devo dimostrare che in questi punti NON ha una singolarità essenziale. Siccome i punti mancanti sono in numero finito riesco a trovare per ogni punto un intorno $U$ dove la funziona è definita su tutto $U-\{ p_0\}$ e una carta $\varphi:U\rightarrow \Omega\subseteq\bbC$, in cui suppongo che $\varphi(p_0)=0$ per semplicità. Allora esiste $\psi:\Omega-\{ 0\}\rightarrow U$ con $\varphi(0)=p_0$ tale che $\varphi\circ\psi:\Omega-\{ 0\}\rightarrow \bbC$ è olomorfa. ( $\psi$ è ottenuta con il Teorema del Dini come inversa delle funzioni coordinate, nel modo visto in altre lezioni).\\
Ci basta dimostrare che la mappa $\varphi\circ\psi$ ha al più un polo in $0$.
Supponiamo per assurdo che $0$ sia una singolarità essenziale, allora per il teorema di Weierstrass ogn intorno bucato di $0$ ha immagine densa in $\bbC$. Sia quindi $V_n$ una successione numerabile di dischi aperti bucati di centro zero e raggio $2^{-n}$. Per il Teorema di Baire si ha che l'intersezione delle immagini dei $V_n$ (le mappe olomorfe sono aperte) è un denso, che chiamo $A$.\\
Sia quindi $\xi\in A$, si ha che $\forall n \exists z_n\in V_n - \{ 0 \} : \varphi ( \psi(z_n ))=\xi$, quindi gli $z_n$ formano una successione tendente a $0$. Ne segue che gli $z_n$ sono almeno una quantità numerabile e siccome $\psi$ è una bigezione anche gli $\psi(z_n)$ sono una quantità almeno numerabile. Ne segue che la fibra $\psi^{-1} (\xi )$ è almeno numerabile, ma ciò è assurdo, poichè $\varphi$ è una mappa razionale (nota che se $\varphi$ è costante non c'è un assurdo, ma sappiamo estendere le mappe costanti). Quindi non ci può essere una singolarità essenziale.
\end{proof}

\notamargine{
Teorema di Baire: In uno spazio metrico completo l'intersezione numerabile di aperti densi è densa. 
Oppure: if a non-empty complete metric space is the countable union of closed sets, then one of these closed sets has non-empty interior. (entrambe da Wikipedia) }

\begin{osservazione}
Nota che la dimostrazione fatta vale per qualsiasi curva algebrica che sia una Superficie di Riemann.\\
Nota che l'abbiamo dimostrato in 2 modi, uno usando che le curve ellittiche vengono dai tori (in modo apparentemente semplice, ma che richiede tanta teoria sotto) e uno usando Baire, che è un teorema semplice.\\
Si potrebbe dimostrare in un altro modo puramente algebrico usando che certi anelli di coordinate sono principali, quest'ultima dimostrazione è complicata ma ha il pregio di valere anche in ambiti in cui la caratteristica è un primo $p$.
\end{osservazione}

Come abbiamo già detto è difficile trovare mappe razionali, osserviamo che se due cubiche $E$ e $E'$ provengono da due tori $T$ e $T'$ si ha che ogni mappa razionale $E\rightarrow E'$ induce una mappa olomorfa sui completamenti quindi induce una mappa olomorfa $T\rightarrow T'$. Nota che (come visto alla "lezione del 24 Ottobre") queste mappe sono, a meno di una traslazione, delle moltiplicazioni per costanti, viste su $\bbC$. E che deve valere la relazione tra i reticoli $\mu L\subseteq L'$, condizione molto restrittiva.

\begin{proposizione}
Esiste un algoritmo per dire se esiste una mappa razionale non costante tra due cubiche, se si conoscono stime a priori sul grado della funzione razionale.
\end{proposizione}

\begin{proof} La mappa razionale è una mappa del tipo $(\varphi_1,\varphi_2)$ con $\varphi_i=r_i(x)+y s_i(x)$ $i=1,2$  che deve soddisfare $(r_2(x)+y s_2(x))^2=4(r_1(x)+y s_1(x))^3-a'(r_1(x)+y s_1(x))-b'$, separando i termini in $1$ e $y$ si ricavano due identità nella sola variabile $x$. Qualcosa del tipo $A(x,r_1,r_2,s_1,s_2)=0$ e $B(x,r_1,r_2,s_1,s_2)=0$. Che ci sono algoritmi per capire se questo sistema ha soluzione, se si conoscono delle stime sui gradi.
\end{proof}

\begin{osservazione} Esiste un algoritmo recente per stimare a priori i gradi delle $r_i$ e $s_i$.\\
Esiste un teorema(profondo): Se ne esiste una non costante allora ne esiste una con grado limitato.\\
Nota che le condizioni sui reticoli sono condizioni con quantità trascendenti, difficili da verificare (equivale a chiedersi se delle espressioni che riportano serie infinite (la $\wp$) sono numeri razionali (???))
\end{osservazione}
