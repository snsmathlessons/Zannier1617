\chapter{23 Gennaio 2017 - Biliardi ellittici e Funzioni Modulari}
\justify

\section{Biliardi ellittici}
\newthought{Introduciamo un altro modo} in cui si ottengono le curve
ellittiche dalle ellissi \notamargine{Abbiamo infatti già visto che si
  possono ottenere come integrali della lunghezza d'arco di un'ellisse}.

Prendiamo un'ellisse di equazione $ax^2 + by^2 = 1$ e supponiamo di
giocare a biliardo sull'ellisse: facendo partire la pallina da un punto
la lanciamo contro il bordo dell'ellisse su cui rimbalza secondo la nota
legge della riflessione \notamargine{Ovvero rispetto alla tangente
  all'ellisse nel punto rimbalza via con lo stesso angolo, come indicato
  in figura}

\begin{center}
  \includegraphics[width=8cm]{lezione-170123-fig1}
\end{center}
  
Una cosa che è nota da tempo è che le rette che compongono la
traiettoria sono tutte tangenti ad un'altra ellisse ``caustica'' che ha
gli stessi fuochi della prima: abbiamo quindi una famiglia ad un
parametro di ellissi che descrive tutte le possibili traiettorie.

\newthought{Vediamo allora che succede} quando prendiamo un punto $P$
sul bordo dell'ellisse ed una retta $l$ con $P \in l$ e tangente alla
caustica:

\includegraphics[width=3cm]{lezione-170123-fig2}

Possiamo definire un'applicazione $\phi$ dalle coppie punto-retta in sè
che è la funzione di ``evoluzione'' della traiettoria sul biliardo,
ovvero manda la coppia $(P, l)$ in $(P', l')$ con $P'$ l'altro punto di
intersezione della retta $l$ con l'ellisse e $l'$ la retta passante per
$P'$ che segue la legge della riflessione con $l$.

\newthought{È noto} che le tangenti in $\bbP^2$ ad una conica sono
parametrizzate da un'altra conica: la conica duale.
\notamargine{Tutto ciò non è difficile da verificare: se la conica $\cC$
  ha equazione $f = ax^2 + by^2 + cz^2$ e $(x_0, y_0, z_0) = P \in \cC$
  allora $(\nabla f)_P = (2ax_0, 2by_0, 2cz_0)$ e, ricordando che tutti
  i punti/vettori considerati sono in $\bbP^2$ si ha che dare la retta
  tangente in $P$ è uguale a fornire il vettore $(\nabla f)_P$. D'altra
  parte si riesce ovviamente a recuperare il punto $P$ dato $(\nabla
  f)_P$ a cui è tangente (basta vedere la formula scritta sopra in
  coordinate, visto che i coefficienti della conica sono noti)}
Allora il luogo di punti su cui la $\phi$ agisce è una sottovarietà
(algebrica) di $C_1 \times \hat{C_2}$, con $C_1 = \text{punti della
  conica}$ e $\hat{C_2} = \text{conica duale delle rette tangenti}$.

Il luogo di punti è dato dalle coppie $(P, l) \in C_1 \times \hat{C_2}$
tali che $P \in l$ (che è una condizione chiusa, ovvero dà luogo ad una
sottovarietà algebrica). Questa è anche una superficie di Riemann.

Scrivendo l'equazione si ottiene una curva ellittica e l'operazione
$\phi$ si rivela essere una traslazione sulla cubica detta ``gioco di
Poncelèt''.

Il gioco ``finisce'' se e solo se la traslazione $\phi(x) = x + \tau$ ha
un punto di ordine finito, ovvero $\exists n$
$\phi^n (x) = x + n\tau = x$ se e solo se $n\tau \in L$, il
reticolo. Ovvero si avrebbe $\phi^n(x) = x$ per ogni punto. Allora se il
gioco finisce per una traiettoria finisce per tutte le altre, cosa che
non è per nulla banale.

\notamargine{Come curiosità, se il gioco non finisce, le traiettorie del
  biliardo sono dense nello spazio tra le due caustiche}

\section{Funzioni Modulari}

\notamargine{Il nome ``modulari'' è riferito ai moduli, parametri che
  comparivano negli integrali ellittici. Oggi ci si riferisce a moduli
  per indicare uno spazio di parametri per una famiglia di curve
  algebriche.

  Esempio ``stupido'': $y - a x^2 = 0$ al variare di $a \in \bbC$ sono
  una famiglia di parabole (o per $a=0$ una retta). In questo caso lo
  spazio dei parametri è $\bbC$ (nel quale $a$ può variare)}

\begin{osservazione}
  Ricordiamo che conosciamo già un parametro delle cubiche:
  $j$. Infatti, se la cubica viene da un toro allora è della forma
  $y^2 = 4 x^3 - g_2 x - g_3$ e sappiamo che
  $j = 1728 \frac{g_2^3}{g_2^3 - 27 g_3^2}$ è un'invariante per
  trasformazioni algebriche delle cubiche.
\end{osservazione}

Siamo allora autorizzati a riscalare il reticolo $L$ pur restando nella
stessa classe di isomorfismo delle cubiche. Possiamo quindi supporre che
$L = \bbZ \tau + \bbZ 1$ con $\tau \in \cH = \{ z \in \bbC | \Img \tau >
0 \}$. In questo modo $g_2 = 60 \sum_{\omega \in L^*} \omega^{-4}$ e
$g_3 = 140 \sum_{\omega \in L^*} \omega^{-6}$ diventano funzioni
olomorfe di $\tau$ come parametro nel semipiano superiore e quindi pure
$j$ è una funzione di $\tau$

\begin{osservazione}
  Se vedessimo che $j$ assume tutti i valori in $\bbC$ ciò dimostrerebbe
  che tutte le cubiche provengono da un toro, poiché sappiamo già che
  due cubiche sono affinemente equivalenti se e solo se hanno lo stesso $j$.
\end{osservazione}

\begin{divagazione}
  Si può dimostrare che $\tau$ è immaginario quadratico su $\bbQ$ allora
  $j(\tau)$ è un numero algebrico. Di seguito diamo un'idea della dimostrazione
  \notamargine{ Immaginario quadratico vuol dire che soddisfa
    un'equazione di secondo grado a coefficienti in $\bbQ$, ovvero $x$ è
    tale che $\exists b, c \in \bbQ$ con $x^2 + bx + c = 0$}

  Quando $\tau$ è un immaginario quadratico il reticolo ha infatti degli
  automorfismi non banali. Se $j$ fosse trascendente, visto che gli
  automorfismi sono funzioni razionali delle coordinate e avremmo
  $\bbQ(j, funz.raz.)$ come campo finitamente generato su $\bbQ$, che ha
  però grado di trascendenza uno.

  Allora si può specializzare $j$, visto che il campo è isomorfo ad una
  cosa con una variabile. Specializzandolo ad ogni altro numero
  trascendente ottengo un campo isomorfo e quindi tutte le curve
  ellittiche avrebbero degli automorfismi non banali (poiché hanno
  uguali campi) e ciò è impossibile poiché le cubiche con automorfismi
  sono in numero numerabile.
\end{divagazione}

\section{Costruzione di un dominio fondamentale}

Vogliamo costruire un dominio fondamentale per lo spazio dei reticoli,
ovvero su cui agiranno le funzioni modulari.
\notamargine{Per dominio fondamentale intendiamo uno spazio in cui è
  presente esattamente un rappresentante per ogni reticolo. Nel nostro
  caso portiamo ogni reticolo nella forma $L = \bbZ + \bbZ \tau$}

Descriviamo innanzitutto la forma del dominio fondamentale:
$ F = \cH \cap \{ \Re z \in [ -\frac{1}{2}, \frac{1}{2} ) \} $
tolto l'insieme $\{ \abs{z} < 1 \} \cup \{ \abs{z} = 1 \mid \Re z > 0 \}$

e definiamo le due applicazioni
$$ S = \lbr{\begin{array}{cc} 0 & 1 \\ -1 & 0 \\ \end{array}} $$
$$ T = \lbr{\begin{array}{cc} 1 & 1 \\ 0 & 1 \\ \end{array}} $$
tra cui si hanno le relazioni $S^2 = (TS)^3 = \Id$

\begin{center}
  \includegraphics[width=8cm]{lezione-170123-fig3}
\end{center}

