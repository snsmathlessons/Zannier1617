\chapter{12 Dicembre 16 - burlesque}
\justify

\newthought{La lezione comincia} con un richiamo delle nozioni recentemente acquisite: il gruppo dei divisori della cubica completata proiettivamente $\Div(\widetilde{E})$, del morfismo $\deg$, del suo nucleo $\Div^0(\widetilde{E})$, del divisore di una funzione meromorfa sul toro $\div(f)\in\Div^0(\widetilde{E})$ e infine del gruppo di Picenni $\Pic^0(\widetilde{E}) := \frac{\Div^0(\widetilde{E})}{\text{divisori principali}}$.

Ricordiamo che per parlare di divisori stiamo assumendo di sapere già che la cubica $\widetilde{E}$ viene da un toro.

\section{Sulla legge di gruppo su $\widetilde{E}$}
Costruiamo la mappa seguente intesa come morfismo di gruppi:
$$
\Theta: \ \Div^0(\widetilde{E})\rar \widetilde{E},
$$
$$
\Theta\left(\sum_P a_P \cdot (P)\right)=\sum_P a_P \cdot P
$$
dove la somma a destra è fatta rispetto alla legge di gruppo della cubica.
Osserviamo che questa mappa è surgettiva: $\Theta((P)-(O))=P$.

\notamargine{La mappa $\Theta$ è la cosa più naturale da considerare:
  sappiamo già sommare (non solo formalmente) gli elementi di $\widetilde{E}$}

\begin{lemma}
Il nucleo della mappa $\Theta$ è dato esattamente dai divisori principali.
\end{lemma}
\begin{proof}
Prendiamo un generico divisore principale e calcoliamoci sopra la mappa:
$$
\Theta(\div f)=\Theta\left(\sum_P \ord_P(f) \cdot (P)\right)=\sum_P \ord_P(f)\cdot P=0
$$
poichè avevamo visto che era vero in $\bbC/L$ usando il T3.

Viceversa la scorsa volta vedemmo (\ref{teorema_divisori_semplici}) che ogni divisore di grado zero $d$ si può scrivere nella forma
$$ d = \mbox{divisore principale} + \sigma(R) = (R) - (O) $$
applicando ora $\Theta$ si ottiene che $R=O$ e ci siamo.
\end{proof}

\notamargine{Visto quanto fa schifo la legge di gruppo sulle cubiche, l'affermazione che facciamo è abbastanza forte. Per ogni combinazione su $\bbZ$ di punti (sommati sulla cubica) $\sum_i \alpha_i P_i = 0$ c'è una funzione $f$ razionale per cui vale $\forall i \quad \ord_{P_i}(f) = \alpha_i$}

Grazie a questo lemma e al primo teorema di isomorfismo si ottiene l' isomorfismo
$$
\widetilde{\Theta}: \Pic^0 \rar \widetilde{E}
$$
che ci dà una nuova interpretazione della legge di gruppo sulla cubica: essa è indotta dalla somma di divisori dentro al Picard. Per gli amanti delle sequenze esatte riassumiamo il tutto così:
$$
0\rar\mbox{divisori principali}\rar \Div^0(\widetilde{E})\rar \widetilde{E}\rar 0.
$$
Prima di procedere diamo un insight di ciò che si potrebbe dimostare; prendiamo l' anello:
$$
\bbC[\wp(z),\wp'(z)]\simeq\frac{\bbC[x,y]}{(y^2 - ax^3 - bx - c)}
$$
quello che si può far vedere è che i suoi ideali (o meglio il modulo dei suoi ideali principali) sono in naturale corrispondenza con il Picard della cubica. All' occhio esperto non sfuggirà l' analogia con gli anelli quadratici che si riassume nei simboli seguenti:
$$
\bbC[x] \leftrightarrow \bbZ\, \ \ , \ \ \ \bbC\left[x,\sqrt{ax^3+bx+c}\right] \leftrightarrow \bbZ\left[\sqrt{\alpha}\right].
$$ 
\section{logaritmo ellittico}
Lo scopo di questa chiacchera è dare una idea di come si possa {\it invertire} la funzione $\wp: \bbC/L \rar \widetilde{E}$, o meglio costruirne delle determinazioni.\\
L' analogia con le quadriche $x^2+y^2=1$ e la funzione $\sin$ è forte: stiamo cercando l' equivalente del logaritmo complesso.\\
Accettiamo il fatto che si possa dare una nozione di forma differenziale sulla cubica (che in fondo è una onesta varietà). Allora definiamo una forma differenziale $\omega$ su $\widetilde{E}$ dandone l' espressione in coordinate locali $(x,y)$:
$$
\omega=\frac{dx}{y(x)}=\frac{dx}{\pm\sqrt{ax^3+bx+c}}
$$
questa forma è definita su $\bbC\setminus\{\alpha, \beta, \gamma\}$ le tre radici della cubica ma ha un problema di scelta della determinazione della radice; perchè sia ben definita occorre operare dei {\it tagli} $\gamma_1$ e $\gamma_2$ che collegano le radici $\alpha,\beta$ e $\gamma,\infty$. Se rimuoviamo questi cammmini dal nostro piano una scelta del segno nella forma $\omega$ è coerente ovunque e si può felicemente integrare. Inolre aver tolto l' $\infty$ ci permette di passare dal piano complesso alla sfera di Riemann.\\
Alla fine abbiamo due forme $\omega_+$ ed $\omega_-$ definite su $\widehat{\bbC}\setminus\{\gamma_1,\gamma_2\}$. Adesso avviene il passaggio critico: dobbiamo incollare queste due sfere tramite l' identificazione rispettiva dei cammini che abbiamo rimosso:
$$
\gamma_1^+\sim \gamma_1^-\ \ , \ \ \gamma_2^+\sim\gamma_2^-;
$$
quest' incollamente a priori è topologico ma può essere fatto in modo da mantenere la struttura complessa, e questo è non ovvio. però dato che topologicamente il risultato è un toro la struttura complessa su di lui, per il teorema di classificazione di Riemann, deve essere quella standard.\\
Questo toro che abbiamo trovato è naturalmente quello corrispondente alla nostra cubica, anche se neanche questo sembra ovvio.\\
Le nostre forme differenziali nell' incollamento non si perdono, anzi si uniscono coerentemente dando luogo quindi ad una unica $\Omega$ definita sul toro.\\
Questa forma agisce naturalmente, tramite integrazione, sui cammini sul toro, o meglio sulle loro classi di equivalenza omotopica, o meglio sull' abelianizzato del gruppo fondamentale:
$$
\int_{c_1 * c_2} \Omega=\int_{c_2 * c_1} \Omega
$$ 
Insomma abbiamo costruito l' integrale sulle classi di omologia del toro (che ha primo gruppo di omotopia/omologia $\bbZ\times \bbZ $), fissati due generatori $\{c_1,c_2\}$ si ha:
$$
\int: \bbZ\times \bbZ \rar \bbC \ \ \ \ \ (m_1,m_2)\mapsto m_1 \int_{c_1} \Omega+m_2 \int_{c_2} \Omega  
$$
L' immagine di questa mappa, per delle ragioni da chiarire, dovrebbe proprio essere il reticolo $L$ soggiacente al toro.\\
Si arriva quindi alla definizione di logaritmo ellittico, fissato $O\in \widetilde{E}$:
$$
\widetilde{E}\rar \bbC/L\ \ \ \ P\mapsto \int_O^P \Omega  
$$
questi ultimi passaggi esulano persistentemente dalla mia comprensione.
\section{mappe tra curve ellittiche diverse}
Avevamo visto che una mappa olomorfa tra due tori $T_1$ e $T_2$, a meno di traslazioni, era per forza moltiplicativa. La domanda che ci poniamo è come questa mappa si legga a livello di cubiche: può essere una funzione trascendente?
\begin{proposizione}
Se $T_1$ e $T_2$ sono tori, $\widetilde{E}_1$ e $\widetilde{E}_2$ le corrispondenti cubiche, $\mu$ complesso non nullo; allora la mappa $\phi_\mu:\widetilde{E}_1\rar\widetilde{E}_2$, indotta dalla moltiplicazione per $\mu$ sui tori, è una funzione razionale delle coordinate.
\end{proposizione}
\begin{proof}
Sappiamo che, se i tori sono indotti da reticoli $L_1$ ed $L_2$, vale:
$$
\mu L_1\subseteq L_2;
$$
utilizzando le funzioni di Weierstrass abbiamo le parametrizzazioni (usiamo carte affini sulla cubica):
$$
\wp_i:T_i\rar \widetilde{E}_i \qquad \ \ z \mapsto (\wp_i(z),\wp_i'(z))\ \ \ \ i=1,2
$$
quindi possiamo scrivere la relazione di commutazione nel modo più efficiente:
$$
\phi_\mu(\wp_1(z),\wp_1'(z))=(\wp_2(\mu z),\wp_2'(\mu z))
$$
a questo punto osserviamo che, grazie alla condizione $\mu L_1\subseteq L_2$, la funzione $\wp_2(\mu z)$ (e la sua derivata) è periodica anche per $L_1$ e dunque appartiene a $\bbC(\wp_1,\wp_1')$; questo implica che la funzione $\phi_\mu$ sia razionale.
\end{proof}
A questo punto è obbligatorio un esempio di reticoli che abbiano morfismi in sè stessi non dati da moltiplicazione per interi:
\begin{fatto}
Per $L=\bbZ[i]$ la corrispondente cubica è $y^2=x^3-x$ e un' isogenia qui è data da $(x,y)\mapsto (-x,iy)$.
\end{fatto}
\section{Grado di una funzione razionale su una curva ellittica}
Diamo la definizione di grado. 
\begin{definizione}
Consideriamo una mappa $f:\widetilde{E}\rar\ \bbC$ che sia funzione razionale delle coordinate. Una tale $f$ verrà detta {\it funzione razionale}. Se è noncostante il suo grado è definito come:
$$
\deg(f,\alpha)=\#\{x\in \widetilde{E}:f(x)=\alpha\}
$$
dove $\alpha$ è un qualsiasi valore regolare di $f$.
\end{definizione}
Il fatto vero e a priori non ovvio è che questo numero sia finito e non dipenda da $\alpha$.\\
Osserviamo subito che per il T2 sulle funzioni ellittiche $f$ ed $f-\alpha$ hanno stesso numero di poli (ovvero di zeri) e quindi:
$$
\deg(f,\alpha)=\#\{f=\alpha\}=\#\{f-\alpha=0\}=\#\{\mbox{poli di } f-\alpha\}=\#\{\mbox{poli di } f\}
$$
e abbiamo eliminato la dipendenza da $\alpha$.\\
Il legame tra grado e curve ellittiche è dato dal seguente
\begin{esercizio}
Sia $f$ funzione ellittica non costante, allora vale:
$$
\deg f= [\bbC(\wp(z),\wp'(z)):\bbC(f(z))].
$$
\end{esercizio}
Con questa sfida si chiude, a Dio piacendo, la lezione di Umberto.

