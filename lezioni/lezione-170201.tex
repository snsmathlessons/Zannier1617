\chapter{01 Febbraio 2017 - Tutte le curve vengono dai tori}

\section{Una relazione tra forme modulari e ordini di annullamento}

\begin{teorema}
	Sia $f$ una forma modulare di peso $2k$, $F$ il dominio fondamentale delle lezioni precedenti,
	e $v_p(f)=\ord_p(f)\geq0$ (poiché $f$ è olomorfa). Allora si ha
	\begin{equation*}
	\v_\infty(f)+\frac{1}{2}\v_i(f)+\frac{1}{3}\v_{\rho}(f)+{\sum_{p\in F}}^{*}\v_p(f)=\frac{k}{6},
	\end{equation*}
	dove con $\dst {\sum_{p\in F}}^{*}$ si intende la somma per tutti i $p \neq i, \rho, \infty$,
	e con $\v_\infty(f)$ si intende $\ord_0(\ot{f})$.
	
	Se $e_p=\abs{\Stab(p)}$, allora l'uguaglianza sopra si può anche scrivere
	\begin{equation*}
		\sum_{p\in F \cup \{\infty\}} \frac{1}{e_p}\v_p(f)=\frac{k}{6}
	\end{equation*}
\end{teorema}

\begin{proof}
	\begin{figure}
		%\centering
		\definecolor{ffqqqq}{rgb}{1,0,0}
\definecolor{qqqqff}{rgb}{0,0,1}
\begin{tikzpicture}[line cap=round,line join=round,>=triangle 45,x=3.0cm,y=3.0cm]
\clip(6.66,0.62) rectangle (8.34,2.71);
\draw(7.5,0) circle (3cm);
\draw (7,0.62) -- (7,2.71);
\draw (8,0.62) -- (8,2.71);
\draw [color=ffqqqq] (7,2.43)-- (8,2.43);
\draw [shift={(7,0.87)},color=ffqqqq]  plot[domain=0.46:1.57,variable=\t]({1*0.13*cos(\t r)+0*0.13*sin(\t r)},{0*0.13*cos(\t r)+1*0.13*sin(\t r)});
\draw [shift={(7.5,1)},color=ffqqqq]  plot[domain=-0.04:3.18,variable=\t]({1*0.08*cos(\t r)+0*0.08*sin(\t r)},{0*0.08*cos(\t r)+1*0.08*sin(\t r)});
\draw [shift={(8,0.87)},color=ffqqqq]  plot[domain=1.57:2.68,variable=\t]({1*0.13*cos(\t r)+0*0.13*sin(\t r)},{0*0.13*cos(\t r)+1*0.13*sin(\t r)});
\draw [shift={(7.5,0)},color=ffqqqq]  plot[domain=1.18:1.49,variable=\t]({1*1*cos(\t r)+0*1*sin(\t r)},{0*1*cos(\t r)+1*1*sin(\t r)});
\draw [shift={(7.5,0)},color=ffqqqq]  plot[domain=1.65:1.96,variable=\t]({1*1*cos(\t r)+0*1*sin(\t r)},{0*1*cos(\t r)+1*1*sin(\t r)});
\draw [color=ffqqqq] (7,2.43)-- (7,1);
\draw [color=ffqqqq] (8,2.43)-- (8,1);
\begin{scriptsize}
\fill [color=qqqqff] (6.5,0) circle (1.5pt);
\draw[color=qqqqff] (6.55,0.09) node {$P$};
\fill [color=ffqqqq] (7.5,0) circle (1.5pt);
\draw[color=ffqqqq] (7.44,0.16) node {$X$};
\fill [color=qqqqff] (8,0) circle (1.5pt);
\draw[color=qqqqff] (8.06,0.09) node {$Q$};
\fill [color=qqqqff] (7,0) circle (1.5pt);
\draw[color=qqqqff] (7.06,0.09) node {$R$};
\draw[color=ffqqqq] (7.09,1.33) node {$\gamma$};
\fill [color=black] (7,0.87) circle (1.5pt);
\draw[color=black] (6.93,0.90) node {$\rho$};
\fill [color=black] (8,0.87) circle (1.5pt);
\draw[color=black] (8.08,0.91) node {$-\ol{\rho}$};
\fill [color=black] (7.5,1) circle (1.5pt);
\draw[color=black] (7.5,0.92) node {$i$};
\fill [color=black] (7,2.43) circle (1.5pt);
\draw[color=black] (7.06,2.52) node {$A$};
\fill [color=black] (7,1) circle (1.5pt);
\draw[color=black] (7.06,1.09) node {$B$};
\fill [color=black] (7.42,1) circle (1.5pt);
\draw[color=black] (7.37,1.07) node {$C$};
\fill [color=black] (7.88,0.92) circle (1.5pt);
\draw[color=black] (7.86,1.02) node {$D$};
\fill [color=black] (8,2.43) circle (1.5pt);
\draw[color=black] (8.06,2.52) node {$E$};
\fill [color=black] (7.12,0.93) circle (1.5pt);
\draw[color=black] (7.19,1.02) node {$B'$};
\fill [color=black] (7.58,1) circle (1.5pt);
\draw[color=black] (7.64,1.08) node {$C'$};
\fill [color=black] (8,1) circle (1.5pt);
\draw[color=black] (8.08,1.09) node {$D'$};
\end{scriptsize}
\end{tikzpicture}
	\end{figure}
	Integriamo $\dst \frac{\de f}{f}$ nei posti giusti: Sia $\gamma$ il cammino mostrato in rosso in figura,
	dove i tratti curvi sono archi di circonferenza e $B$ e $D'$ sono uno il traslato dell'altro.
	
	Possiamo supporre che $f$ non abbia zero al di fuori della regione delimitata da $\gamma$.
	Infatti, $f$ (o $\ot f$, se siamo in $\infty$) è olomorfa, dunque gli zeri non possono accumularsi;
	consideriamo ad esempio $\rho$: se $f(\rho)=0$, allora esiste un intorno in cui $\rho$ è l'unico zero,
	mentre se $f(\rho)\neq0$ per continuità $f$ sarà diversa da $0$ in tutto un intorno.
	
	Inoltre possiamo supporre che $f$ non abbia zeri su $\gamma$: se li avesse sui lati $AB$ e $D'E$
	sarebbe possibile modificare $\gamma$ attraverso una ``piccola'' deviazione in modo che i due pezzi del percorso
	continuino ad essere l'uno il traslato dell'altro, e la zona interna rimarrebbe la stessa perché 
	a due punti traslati corrisponde lo stesso valore di $f$; sugli archi $B'C$ e $C'D$ si agisce in modo analogo,
	e sugli altri tratti è possibile non avere zeri per lo stesso ragionamento di prima.
	
	Dunque per il teorema dei residui si ha
	\begin{equation}
		\frac{1}{2\pi i} \int_{\gamma} \frac{\de f}{f}={\sum_{p\in F}}^{*}v_p(f)
	\end{equation}
	
	Scriviamo l'integrale in un altro modo, spezzandolo nei vari pezzi.
	
	\begin{equation}
		\frac{1}{2\pi i}\int_E^A\frac{\de f}{f}=\frac{1}{2\pi i}\int_{\de\Omega}\frac{\de \ot{f}}{\ot{f}}=-v_0(\ot{f})=-v_\infty(f),
	\end{equation}
	dove $\Omega$ è un disco intorno all'origine, di raggio $e^{2\pi \mbox{Im}(A)}$, e $\de\Omega$ è percorso in senso orario.
	
	\begin{equation}
		\frac{1}{2\pi i}\left( \int_A^B\frac{\de f}{f} + \int_{D'}^E\frac{\de f}{f} \right) =0
	\end{equation}
	
	Si ha che, poiché $f$ è una forma modulare, $f(Sz)=f(-\frac{1}{z})=z^{2k}f(z)$, da cui, prendendo la derivata logaritmica,
	$\dst\frac{\de f}{f}(Sz)=2k\cdot\frac{\de z}{z}+\frac{\de f}{f}(z)$. Dunque
	\begin{multline}
		\frac{1}{2\pi i}\left\{ \int_{B'}^C\frac{\de f}{f} + \int_{C'}^D\frac{\de f}{f} \right\} \overset{*}{=} %
		\frac{1}{2\pi i}\left\{ \int_{B'}^C\frac{\de f}{f} + \int_{C}^{B'} \left[2k \cdot \frac{\de z}{z} + \frac{\de f}{f}\right] \right\} = \\
		= \frac{1}{2\pi i}\int_{C}^{B'} 2k \cdot \frac{\de z}{z} \longrightarrow \frac{2k}{2\pi i}\int_i^\rho \frac{\de z}{z} = \frac{\pi}{6},
	\end{multline}
	dove l'uguaglianza $*$ è valida perché $S(C'D)=CB'$, dove conservo anche il verso di percorrenza
	
	Supponendo che $f(z)=(z-i)^mg(z)$, dove $g$ è olomorfa e non si annulla in $i$, cioè $m=v_i(f)$, possiamo scrivere
	\begin{equation}
		\frac{1}{2\pi i}\int_C^{C'}\frac{\de f}{f} = \frac{1}{2\pi i}\int_C^{C'}\left[ m\cdot\frac{\de z}{z-i} + %
		\frac{\de g}{g}(z) \right] \longrightarrow -\frac{m}{2},
	\end{equation}
	perché $\int_C^{C'}\frac{\de g}{g}(z)$ tende a $0$ se $C$ tende a $i$, e se $C$ tende a $i$
	l'arco di circonferenza $CC'$ tende ad essere un semicerchio, percorso in senso orario.
	
	Analogamente si ottiene 
	\begin{equation}
		\frac{1}{2\pi i}\int_B^{B'}\frac{\de f}{f}=\frac{1}{2\pi i}\int_D^{D'}\frac{\de f}{f}=-\frac{v_\rho(f)}{6}
	\end{equation}
	
	Uguagliando gli integrali si ottiene la tesi.
\end{proof}
