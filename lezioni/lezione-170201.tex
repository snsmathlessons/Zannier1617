\chapter{01 Febbraio 2017 - Tutte le curve vengono dai tori}

\section{Una relazione tra forme modulari e ordini di annullamento}

\begin{teorema}
    Sia $f$ una funzione modulare di peso $2k$, $F$ il dominio fondamentale delle lezioni precedenti. Allora si ha
    \begin{equation*}
	\ord_\infty(f)+\frac{1}{2}\ord_i(f)+\frac{1}{3}\ord_{\rho}(f)+{\sum_{p\in F}}^{*}\ord_p(f)=\frac{k}{6},
    \end{equation*}
    dove con $\dst {\sum_{p\in F}}^{*}$ si intende la somma per tutti i $p \neq i, \rho, \infty$.
    
    Se $e_p=\abs{\Stab(p)}$, allora l'uguaglianza sopra si può anche scrivere
    \begin{equation*}
     	\sum_{p\in F \cup \{\infty\}} \frac{1}{e_p}\ord_p(f)=\frac{k}{6}
    \end{equation*}
\end{teorema}
\begin{proof}
    Integriamo $\dst \frac{\de f}{f}$ nei posti giusti
\end{proof}
