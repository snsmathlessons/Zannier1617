\chapter{01 Febbraio 2017 - Tutte le curve vengono dai tori}

\section{Una relazione tra forme modulari e ordini di annullamento}

\begin{teorema} \label{170201-forme-ordini}
	Sia $f$ una funzione modulare di peso $2k$, $F$ il dominio fondamentale delle lezioni precedenti. Allora si ha
	\begin{equation*}
	\ord_\infty(f)+\frac{1}{2}\ord_i(f)+\frac{1}{3}\ord_{\rho}(f)+{\sum_{p\in F}}^{*}\ord_p(f)=\frac{k}{6},
	\end{equation*}
	dove con $\dst{\sum_{p\in F}}^{*}$ si intende la somma per tutti i $p \neq i, \rho, \infty$, e con $\ord_\infty(f)$ si intende $\ord_0(\ot{f})$.
	
	Se $e_p=\abs{\Stab(p)}$, allora l'uguaglianza sopra si può anche scrivere
	\begin{equation*}
		\sum_{p\in F \cup \{\infty\}} \frac{\ord_p(f)}{e_p}=\frac{k}{6}
	\end{equation*}
\end{teorema}

\begin{proof}
	\begin{figure}
		%\centering
		\definecolor{ffqqqq}{rgb}{1,0,0}
\definecolor{qqqqff}{rgb}{0,0,1}
\begin{tikzpicture}[line cap=round,line join=round,>=triangle 45,x=3.0cm,y=3.0cm]
\clip(6.66,0.62) rectangle (8.34,2.71);
\draw(7.5,0) circle (3cm);
\draw (7,0.62) -- (7,2.71);
\draw (8,0.62) -- (8,2.71);
\draw [color=ffqqqq] (7,2.43)-- (8,2.43);
\draw [shift={(7,0.87)},color=ffqqqq]  plot[domain=0.46:1.57,variable=\t]({1*0.13*cos(\t r)+0*0.13*sin(\t r)},{0*0.13*cos(\t r)+1*0.13*sin(\t r)});
\draw [shift={(7.5,1)},color=ffqqqq]  plot[domain=-0.04:3.18,variable=\t]({1*0.08*cos(\t r)+0*0.08*sin(\t r)},{0*0.08*cos(\t r)+1*0.08*sin(\t r)});
\draw [shift={(8,0.87)},color=ffqqqq]  plot[domain=1.57:2.68,variable=\t]({1*0.13*cos(\t r)+0*0.13*sin(\t r)},{0*0.13*cos(\t r)+1*0.13*sin(\t r)});
\draw [shift={(7.5,0)},color=ffqqqq]  plot[domain=1.18:1.49,variable=\t]({1*1*cos(\t r)+0*1*sin(\t r)},{0*1*cos(\t r)+1*1*sin(\t r)});
\draw [shift={(7.5,0)},color=ffqqqq]  plot[domain=1.65:1.96,variable=\t]({1*1*cos(\t r)+0*1*sin(\t r)},{0*1*cos(\t r)+1*1*sin(\t r)});
\draw [color=ffqqqq] (7,2.43)-- (7,1);
\draw [color=ffqqqq] (8,2.43)-- (8,1);
\begin{scriptsize}
\fill [color=qqqqff] (6.5,0) circle (1.5pt);
\draw[color=qqqqff] (6.55,0.09) node {$P$};
\fill [color=ffqqqq] (7.5,0) circle (1.5pt);
\draw[color=ffqqqq] (7.44,0.16) node {$X$};
\fill [color=qqqqff] (8,0) circle (1.5pt);
\draw[color=qqqqff] (8.06,0.09) node {$Q$};
\fill [color=qqqqff] (7,0) circle (1.5pt);
\draw[color=qqqqff] (7.06,0.09) node {$R$};
\draw[color=ffqqqq] (7.09,1.33) node {$\gamma$};
\fill [color=black] (7,0.87) circle (1.5pt);
\draw[color=black] (6.93,0.90) node {$\rho$};
\fill [color=black] (8,0.87) circle (1.5pt);
\draw[color=black] (8.08,0.91) node {$-\ol{\rho}$};
\fill [color=black] (7.5,1) circle (1.5pt);
\draw[color=black] (7.5,0.92) node {$i$};
\fill [color=black] (7,2.43) circle (1.5pt);
\draw[color=black] (7.06,2.52) node {$A$};
\fill [color=black] (7,1) circle (1.5pt);
\draw[color=black] (7.06,1.09) node {$B$};
\fill [color=black] (7.42,1) circle (1.5pt);
\draw[color=black] (7.37,1.07) node {$C$};
\fill [color=black] (7.88,0.92) circle (1.5pt);
\draw[color=black] (7.86,1.02) node {$D$};
\fill [color=black] (8,2.43) circle (1.5pt);
\draw[color=black] (8.06,2.52) node {$E$};
\fill [color=black] (7.12,0.93) circle (1.5pt);
\draw[color=black] (7.19,1.02) node {$B'$};
\fill [color=black] (7.58,1) circle (1.5pt);
\draw[color=black] (7.64,1.08) node {$C'$};
\fill [color=black] (8,1) circle (1.5pt);
\draw[color=black] (8.08,1.09) node {$D'$};
\end{scriptsize}
\end{tikzpicture}
	\end{figure}
	Dimostriamo solo il caso in cui $f$ è una forma modulare (cioè $f$ è olomorfa ovunque), e sia $v_p(f)=\ord_p(f)\geq0$.
	
	Integriamo $\dst \frac{\de f}{f}$ nei posti giusti: Sia $\gamma$ il cammino mostrato in rosso in figura, dove i tratti curvi sono archi di circonferenza e $B$ e $D'$ sono uno il traslato dell'altro.
	
	Possiamo supporre che $f$ non abbia zero al di fuori della regione delimitata da $\gamma$.
	Infatti, $f$ (o $\ot f$, se siamo in $\infty$) è olomorfa, dunque gli zeri non possono accumularsi; consideriamo ad esempio $\rho$: se $f(\rho)=0$, allora esiste un intorno in cui $\rho$ è l'unico zero, mentre se $f(\rho)\neq0$ per continuità $f$ sarà diversa da $0$ in tutto un intorno.
	
	Inoltre possiamo supporre che $f$ non abbia zeri su $\gamma$: se li avesse sui lati $AB$ e $D'E$ sarebbe possibile modificare $\gamma$ attraverso una ``piccola'' deviazione in modo che i due pezzi del percorso continuino ad essere l'uno il traslato dell'altro, e la zona interna rimarrebbe la stessa perché a due punti traslati corrisponde lo stesso valore di $f$; sugli archi $B'C$ e $C'D$ si agisce in modo analogo (considerando anziché l'azione di $T$, quella di $S$), e sugli altri tratti è possibile non avere zeri per lo stesso ragionamento di prima.
	
	Dunque per il teorema dei residui si ha
	\begin{equation*}
		\frac{1}{2\pi i} \int_{\gamma} \frac{\de f}{f}={\sum_{p\in F}}^{*}v_p(f)
	\end{equation*}
	
	Scriviamo l'integrale in un altro modo, spezzandolo nei vari pezzi.
	
	\begin{equation*}
		\frac{1}{2\pi i}\int_E^A\frac{\de f}{f}=\frac{1}{2\pi i}\int_{\partial\Omega}\frac{\de \ot{f}}{\ot{f}}=-v_0(\ot{f})=-v_\infty(f),
	\end{equation*}
	dove $\Omega$ è un disco intorno all'origine, di raggio $e^{2\pi \Im(A)}$, e $\partial\Omega$ è percorso in senso orario.
	
	\begin{equation*}
		\frac{1}{2\pi i}\left( \int_A^B\frac{\de f}{f} + \int_{D'}^E\frac{\de f}{f} \right) =0
	\end{equation*}
	
	Si ha che, poiché $f$ è una forma modulare, $f(Sz)=f(-\frac{1}{z})=z^{2k}f(z)$, da cui, prendendo la derivata logaritmica,
	$\dst\frac{\de f}{f}(Sz)=2k\cdot\frac{\de z}{z}+\frac{\de f}{f}(z)$. Dunque
	\begin{multline}
		\frac{1}{2\pi i}\left\{ \int_{B'}^C\frac{\de f}{f} + \int_{C'}^D\frac{\de f}{f} \right\} \overset{*}{=} %
		\frac{1}{2\pi i}\left\{ \int_{B'}^C\frac{\de f}{f} + \int_{C}^{B'} \left[2k \cdot \frac{\de z}{z} + \frac{\de f}{f}\right] \right\} = \\
		= \frac{1}{2\pi i}\int_{C}^{B'} 2k \cdot \frac{\de z}{z} \longrightarrow \frac{2k}{2\pi i}\int_i^\rho \frac{\de z}{z} = \frac{k}{6},
	\end{multline}
	dove l'uguaglianza $*$ è valida perché $S(C'D)=CB'$, dove conservo anche il verso di percorrenza.
	
	Supponendo che $f(z)=(z-i)^mg(z)$, dove $g$ è olomorfa e non si annulla in $i$, cioè $m=v_i(f)$, possiamo scrivere
	\begin{equation*}
		\frac{1}{2\pi i}\int_C^{C'}\frac{\de f}{f} = \frac{1}{2\pi i}\int_C^{C'}\left[ m\cdot\frac{\de z}{z-i} + %
		\frac{\de g}{g}(z) \right] \longrightarrow -\frac{m}{2},
	\end{equation*}
	perché $\int_C^{C'}\frac{\de g}{g}(z)$ tende a $0$ se $C$ tende a $i$, e se $C$ tende a $i$ l'arco di circonferenza $CC'$ tende ad essere un semicerchio, percorso in senso orario.
	\notamargine{Se avessi percorso l'intera circonferenza, al limite per il teorema dei residui avrei ottenunto $m$. Poiché ne ho percorsa solo metà, è ``ragionevole'' aspettarsi di ottenere, sempre al limite, un valore che è la metà del valore intero.}
	
	Analogamente si ottiene 
	\begin{equation*}
		\frac{1}{2\pi i}\int_B^{B'}\frac{\de f}{f}=\frac{1}{2\pi i}\int_D^{D'}\frac{\de f}{f}=-\frac{v_\rho(f)}{6}
	\end{equation*}
	
	Uguagliando gli integrali si ottiene la tesi.
\end{proof}

\section{Struttura delle forme modulari}

\begin{corollario}
	Sia $M_k$ l'insieme delle forme modulari di peso $2k$. Allora $M_k$ è uno spazio vettoriale su $\bbC$ di dimensione finita.
\end{corollario}

\begin{proof}
	Per induzione: il passo base è per $k\leq6$.
	\begin{itemize}
		\item Per $k<0$ si ha che $M_k=\{0\}$;
		
		\item Per $k=0$ si ha $M_0=\bbC$: infatti, per il teorema precedente, ogni $f\in M_0$ o è $0$ o la somma pesata degli ordini di annullamento è $0$. Poiché gli ordini sono positivi, la funzione non si deve mai annullare, ed è quindi una costante diversa da $0$;
		
		\item Per $k=1$ si ha $M_1=\{0\}$. Infatti non è possibile scrivere $\frac{1}{6}$ come combinazione lineare intera positiva di $1, \frac{1}{3}, \frac{1}{2}$;
		
		\item Per $k=2$ si ha $M_2=\bbC G_2$. Infatti le forme $\neq0$ in $M_2$ si annullano solo in $\rho$ con ordine 1,
		e $G_2$ soddisfa questa proprietà. Se $f\in M_2$, sia $x\neq\rho$. Allora considero $f-\frac{f(x)}{G_2(x)}G_2$. Questa funzione si annulla in $x$ e in $\rho$, ed è quindi $0$, dunque $f=\frac{f(x)}{G_2(x)}G_2$
		
		\item Per $k=3, 4, 5$, allo stesso modo, si ha $M_3=\bbC G_3, M_4=\bbC G_2^2, M_5= \bbC G_2G_3$
		
		\item Per $k=6$, si ha $M_6=\bbC G_2^3 + \bbC G_3^2$. Infatti, con lo stesso ragionamento di prima, otteniamo che $G_2^3, G_3^2 \in M_6$, e, con combinazioni lineari di queste, possiamo trasformare una funzione $f \in M_6$ in un'altra $f'$ con $v_\rho(f')=3, v_i(f')=2$.
		Dunque, per la formula del teorema precedente, $f'$ non può essere in $M_6$ a meno che non sia $0$.
		Inoltre $G_2^3$ e $G_3^2$ sono linearmente indipendenti, perché si annullano in punti diversi, e dunque sono una base di $M_6$.
		
		Notiamo inoltre che $\Delta\in M_6$, poiché combinazione di $G_2^3$ e $G_3^2$.
		Poiché $\Delta(z)\neq0$ in $\cH$, si deve avere che $\Delta(\infty)=0$ con ordine $1$.
	\end{itemize}
	
	Per $k>6$ abbiamo che $G_k \in M_k$, e inoltre $G_k(\infty)\neq0$.
	Sia allora $f \in M_k$, e $c=\frac{f(\infty)}{G_k(\infty)}$. $f-cG_k$ si annulla in $\infty$, dunque $\frac{f-cG_k}{\Delta}$ è ancora una forma modulare, di peso $2(k-6)$, e dunque appartiene a $M_{k-6}$.
	\notamargine{$\frac{f-cG_k}{\Delta}$ è olomorfa in $\infty$ perché $\ord_\infty(\Delta)=1$, e $\ord_\infty(f-cG_k)\geq1$, e altrove perché $\Delta$ non assume il valore $0$ su $\cH$}
	Dunque $f$ si scrive come $cG_k+\Delta f_{k-6}$, per un qualche $f_{k-6} \in M_{k-6}$, e di conseguenza si ha $M_k=\bbC G_k + \Delta M_{k-6}$.
	
	Per ipotesi induttiva $M_{k-6}$ è di dimensione finita, dunque lo è anche $M_k$, e in particolare $\dim(M_k)=\dim(M_{k-6})+1$.
	Da questo e dalle dimensioni dei casi base si deduce la formula per la dimensione degli $M_k$:
	
	\begin{equation*}
		\begin{cases}
			0 						&	\text{se $k<0$}\\
			\floor{\frac{k}{6}} +1	&	\text{se $k\geq0$, $k\not\equiv1\pmod{6}$}\\
			\floor{\frac{k}{6}}		&	\text{se $k\geq0$, $k\equiv1\pmod{6}$}
		\end{cases}
	\end{equation*}

\end{proof}

Come ulteriore corollario abbiamo che $M_k$ è generato dai monomi $G_2^\alpha G_3^\beta$, con $2\alpha+3\beta=k$.
Infatti, siano $a$ e $b$ fissati tali che $2a+3b=k$.
Allora si ha $G_2^a G_3^b\in M_k$, e $(G_2^a G_3^b)(\infty)\neq0$.
Ripercorrendo la dimostrazione, usando $G_2^a G_3^b$ al posto di $G_k$, si ottiene $M_k=\bbC G_2^a G_3^b + \Delta M_{k-6}$, ma $M_{k-6}$ è generato da $G_2^x G_3^y$ con $2x+3y=k-6$, e $\Delta$ si scrive come combinazione lineare di $G_2^3$ e $G_3^2$, dunque $\Delta M_{k-6}$ è generato da $G_2^x G_3^y$ con $2x+3y=k$, e dunque lo è anche $M_k$.

\notamargine{Le serie di Fourier delle $G_k$ sono ``interessanti'', perché i coefficienti di Fourier sono delle funzioni aritmetiche notevoli.}

\section{L'invariante $j$ e la corrispondenza curve-tori}

\notamargine{Ricordiamo che $j$ è definito come $1728\frac{g_2^3}{\Delta}$.}
$j$ è una funzione modulare di peso $0$, dunque è invariante per trasformazioni in $\SL_2(\bbZ)$; ha un polo semplice in $\infty$ e nessun altro polo.

Ora, sia $c \in \bbC$. Applicando il teorema \ref{170201-forme-ordini} alla funzione $j-c$, otteniamo che questa, avendo un solo polo, deve avere anche un solo zero (di ordine opportuno se in $\rho$ o in $i$), e dunque facendo variare $c$ si ottiene che la funzione $j$ assume ogni possibile valore complesso. Dunque, per l'osservazione \ref{170123-j_suriettiva}, abbiamo effettivamente dimostrato che \emph{ogni cubica viene da un toro}.
\notamargine{In particolare $\quotient{\cH}{G} \overset{j}{\rar} \bbC$ è una bigezione}

Infine come nota finale, aggiungiamo che ogni funzione modulare di peso $0$ appartiene a $\bbC(j)$, cioè è una funzione razionale della funzione $j$ (non dimostrato). Per la dimostrazione vedere il Serre (A course in Arithmetic) pagina 90 (Sotto il 3. The space of modular function - Proposizione 6)

% Qui finiscono le avventure del terzo anno 2016/17 nella landa di Zannier. 