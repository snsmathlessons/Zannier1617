\chapter{Appendice B - Domande degli Orali}
\justify

\newthought{In preparazione al giorno del giudizio}, riportiamo di seguito le domande che sono state fatte agli orali.
Esse hanno la piacevole tendenza a ripetersi con particolare frequenza, ragion per cui conviene considerare questa appendice come parte integrante del corso.

\section{Punti di torsione}
Per punti di $n$-torsione si intendono i punti $P$ di una curva ellittica $E$ tali che $nP = 0$ (ovvero $P + \ldots + P$ con le operazioni sulla cubica).

\subsection{$n$-torsione sul toro}
Notate che sul toro la $n$-torsione è una scemenza: infatti si ha che i punti di $n$-torsione sono $p_{h, k} = \omega_1 \frac{k}{n} + \omega_2 \frac{h}{n}$ con $1 \le k, h \le n$.
Questi sono sicuramente di $n$-torsione: $n p_{h, k} \in L$. D'altra parte visto che la moltiplicazione sul toro coincide con la moltiplicazione sui numeri complessi risulta immediato anche il contrario.

Allora se sappiamo che la cubica viene da un toro si ha immediatamente che i punti di $n$-torsione sono $n^2$ e corrispondono a $(\wp(p_{h, k}), \wp'(p_{h, k})$ in coordinate affini.

Sulla cubica (senza assumere la provenienza da un toro, ovvero anche su campi diversi da $\bbC$) la questione è molto più complicata, visto che la legge di gruppo non ha una descrizione così bella come sui complessi.
\notamargine{Anche se rimane vero che i punti di ordine che divide $n$ sono un sottogruppo isomorfo a $\frac{\bbZ}{n\bbZ} \times \frac{\bbZ}{n\bbZ}$}

\subsection{Punti di $2$-torsione}
Notiamo che per le cubiche provenienti da tori si ha che i punti di due torsione sono le immagini dei punti di $\frac{L}{2}$, ovvero i punti dove $\wp'$ si annulla, ovvero le coppie $(0, \alpha)$ con $\alpha$ radice del polinomio $p(x)$ nella forma di Weierstrass $y^2 = p(x)$.

Ciò vale anche per le cubiche generiche: infatti calcolando la retta proiettiva tangente in quei punti si ha che essa incontra la cubica nel punto all'infinito, che è l'elemento neutro del gruppo.
Inoltre 

\subsection{Calcolo dei punti di $3$-torsione}
\paragraph{Domanda} Trovare un polinomio che si annulla esattamente sulle coordinate dei punti di $3$-torsione

\paragraph{Soluzione}
\squared{1} Notiamo che se $P = (x_0, y_0)$ è di $3$-torsione, allora anche $-P = (x_0, -y_0)$ lo è
\squared{2} Prima di lanciarvi in lunghi e boriosi conti per calcolare le coordinate di $3P$ notate che si ha $3P = 0 \sse 2P = -P$. Allora i punti di tre torsione sono precisamente (per l'osservazione precedente ed il fatto che ad $x_0$ fissato si hanno al più due punti della cubica affine con $x_0$ come prima coordinata) quelli tali che $x_{2P} = x_{-P} = x_{P}$. Lestamente ricordando la formula di duplicazione si giunge alla formula $\left( \frac{12 x_P^2 - g_2}{2 y_P} \right)^2 - 2 x_P = x_P$. Ricordando $y_P^2 = 4 x_P^2 - g_2 x_P - g_3$ si giunge ad un polinomio di quarto grado nella sola $x_P$.
\squared{3} Bisogna dimostrare che il polinomio ottenuto non ha radici multiple. In questo modo abbiamo otto punti che sono di $3$-torsione sull'affine, a cui bisogna aggiungere quello all'infinito, per un totale di $9$.

\subsection{Coincidenza con i flessi della cubica}
\paragraph{Domanda} Dimostrare che i punti di $3$-torsione coincidono con i flessi della cubica

\paragraph{Soluzione}
\squared{1} Per un semplice computo della cardinalità delli due insiemi considerati (entrambi con $9$ punti) basta dimostrare che tutti i flessi sono punti di $3$-torsione.
\squared{2} Se $(x_0, y_0)$ è un flesso allora anche $(x_0, -y_0)$ lo è
\squared{3} Se una retta è tangente in un flesso (si annulla di ordine 2) allora è la tangente di flesso (si annulla di ordine 3) poiché i punti sono non singolari
\squared{4} Il punto all'infinito è di flesso
\squared{5} Si prenda un flesso $P$. Vogliamo calcolare $2P = P + P$: tracciamo allora la retta tangente al flesso e cerchiamo il terzo punto di intersezione: esso è il punto di flesso stesso per l'osservazione (3). Allora si ha che $2P$ è l'opposto del flesso stesso, ovvero se $P = (x_0, y_0)$, sappiamo che il terzo punto di intersezione con la retta passante per il punto all'infinito è $(x_0, -y_0) = -P$. Allora abbiamo appena mostrato che $2P = -P$ e quindi $3P = 0$.

\subsection{Proprietà delle rette}
\paragraph{Domanda} Mostra che i punti di $3$-torsione hanno la seguente proprietà: se $P$ e $Q$ sono punti di tre torsione, e traccio la retta tra di essi, la interseco con la cubica, anche il terzo punto è di tre torsione.
\notamargine{A quanto pare questa proprietà delle rette per punti nel piano reale implica che siano tutti allineati, quindi possiamo dire che almeno un punto di tre torsione non ha coordinate entrambe reali}

\paragraph{Soluzione}
\squared{1} ho che $3P = 0$ e $3Q = 0$. Allora (per abelianità del gruppo) $3(P + Q) = 0$, ovvero anche $3(- P - Q) = 0$.
\squared{2} Notare che la terza intersezione della retta con la cubica è proprio $- P - Q$ (infatti se ne prende l'inverso per ottenere $P + Q$).

\section*{Altre domande}
Le domande sono state fatte nell'ordine a 
\begin{itemize}
\item Clara Antonucci
\item Nicola Picenni
\item Matteo Migliorini
\item Umberto Pappalettera
\item Federico Franceschini
\item Dario Balboni
\item Andrea Marino
\item Giovanni Italiano
\item Andrea Caberletti
\item Manuele Cusumano
\item Gianluca Tasinato
\end{itemize}

\bigskip
\bigskip

\begin{enumerate}
%Clara
\item Definizione di razionalità e unirazionalità. Le curve ellittiche lo sono? (ha chiesto solo di rispondere, non di motivare la risposta).
\item Sia $\tilde E$ una cubica in forma di Weierstrass ($\tilde E= \{ (x:y:z)\in \bbP^{2}\bbC \mid 0=A(x, y, z)=-zy^{2}+4x^{3}+\alpha xz^{2} + \beta z^{3} \}$). Supponiamo che esistano $a(t), b(t) \in \bbC(t)$ tali che soddisfino l'equazione affine della cubica: $A(a(t), b(t), 1) \equiv 0$. \'E vero che 
\begin{alignat*}{2}
	\varphi:\bbC(x,y)&\longrightarrow	&\bbC(t)\\
		x	&\longmapsto		&a(t)\\
		y	&\longmapsto		&b(t)\\
\end{alignat*}
è un isomorfismo di campi?
\item Di che grado è la funzione: $f \equiv \wp '\wp - \wp^{4}$?
\bigskip
\hrule
\bigskip

%Nick
\item Quanti sono i punti di ordine che divide n su una cubica non singolare? (con dimostrazione).
\item Formula di dupilcazione sulla cubica (che valga anche su campi a caratteristica non nulla).
\item Calcolare esplicitamente la tangente in un punto.
\item $x_{2P}$ dipende solo da $x_{P}$ e non da $y_{P}$. Perchè?
\bigskip
\hrule
\bigskip

%Miglio
\item Dimostrare che $x_{nP}$ dipende solo da $x_{P}$.
\item Quando $a_{n}=2^{n}P$ diventa periodica?
\item Qualcosa di teoria sulle forme modulari (facile).
\bigskip
\hrule
\bigskip

%Prof
\item Cos'è una funzione modulare? $j$ lo è? A cosa serve $j$?
\item Quando ci sono mappe razionali tra due cubiche?
\item Definizone di grado di una isogenia tra tori. Come si calcola?
\bigskip
\hrule
\bigskip

%Fred
\item Definizione di funzione modulare. \'E vero che $j^{2}$ è una funzione modulare?
\item Quanti sono i punti di 3-torsione di una cubica non singolare? Calcolarli. 
\bigskip
\hrule
\bigskip

%Balbo
\item Finire la dimostrazione della non modularità di $j^2$.
\item Caratterizzare il sottogruppo degli automorfismi di $\bbC$ per cui $j^2$ è invariante.
\item Altra caratterizzazione dei punti di 3-torsione. Mostrare che soddisfano la seguente proprietà: "Dati due punti nell'insieme, considero la retta passante per i due punti. Allora questa interseca l'insieme in almeno un altro punto distinto dai primo due". 
\item Definizione di grado di una funzione razionale.
\bigskip
\hrule
\bigskip

%Marinelo
\item Come agisce $\bbP SL_2(\bbZ)$ su $\cH$?
\item Trovarne un dominio fondamentale.
\item Dimostrare che agisce in maniera propriamente discontinua.
%\item Trovare i punti fissi ({\it nota dell'interrogato:} ad una certa si è dimenticato di aver fatto questa domanda).
\item Definizione di grado di una funzione razionale in una variabile. Cosa significa valore regolare? Perchè il grado non dipende dal punto scelto?
\item Mostrare che deg$f=[\bbC(t):\bbC(f)]$
\bigskip
\hrule
\bigskip

%Giova
\item Trovare i punti fissi dell'azione di $\bbP SL_{2}(\bbZ)$ su $\cH$.
\item Sia $f$ una funzione meromorfa ellittica e sia $f'$ la sua derivata. Allora $\exists p\in \bbC[x, y]$ tale che $p(f, f')\equiv 0$.
\item Data $f$ meromorfa, mostrare che esiste un polinomio $a\in\bbC[x, y]$ tale che $a(f, \wp)\equiv 0$.
\item Mostrare che, date $f, g$ funzioni meromorfe ellittiche, $\exists q\in \bbC[x, y]$ tale che $q(f, g)\equiv 0$.
\bigskip
\hrule
\bigskip

%Cabe
\item A proposito di $\Gamma(2)$:
\begin{enumerate}[label=(\alph*)]
	\item Quale sequenza esatta soddisfa? Mostrare la surgettività alla fine.
	\item Che indice ha $\Gamma(2) < SL_{2}(\bbZ)$?
	\item Che indice ha $\Gamma(p) < SL_{2}(\bbZ)$ ($\Gamma(p)$ definito con la sequenza esatta di sopra, sostituendo 2 con un primo $p$ generico)?
	\item Quale può essere un dominio fondamentale per l'azione di $\Gamma(2)$ su $\cH$?
	\item Trovare i punti fissi dell'azione.
\end{enumerate}
\item Caratterizzare le funzioni olomorfe tra tori.
\bigskip
\hrule
\bigskip

%Cusu
\item Sia $L\subseteq \bbC$ reticolo. Sia $f: \bbC \rar \bbC$ funzione olomorfa tale che $\forall \lambda \in L$ $\exists c_{\lambda}\in\bbC$ tale che $\forall z\in\bbC$ si abbia $f(z+\lambda)=c_{\lambda}f(z)$. Cosa si sa dire?
\item Sia $\tilde E$ cubica che viene da un toro $T=\faktor{\bbC}{L}$, $P\in \tilde E$ punto di ordine due. Chi è $\faktor{\tilde E}{\langle P\rangle}$?
\item Scrivere la mappa razionale tra le due curve ellittiche trovate sopra.
\bigskip
\hrule
\bigskip

%Tasi
\item Finire di scrivere la mappa razionale tra le due curve.
\item Finire di trovare i punti fissi dell'azione di $\Gamma(2)$ su $\cH$.
\item Non birazionalità delle cubiche.
\end{enumerate}
\newpage

\section*{Risposte}
Non sono necessariamente dimostrazioni complete, potrebbero essere solo idee di dimostrazione [in alcuni punti non è nemmeno chiaro quale fosse la domanda]
\bigskip
\begin{enumerate}

%Clara
\item vedi dispense
\item Sì. La cosa non banale è mostrare che se $q\in \bbC[x, y]$ è un polinomio non nullo allora anche $q(a(t), b(t))$ non è identicamente nullo. Una possibile dimostrazione è che se si annulla in infiniti punti ho un assurdo per Bézout. Una seconda dimostrazione è considerare $q$ come polinomio in $\bbC(x)[y]$ e fare la divisione euclidea tra $q$ e $A(x, y)$. Si ottiene un resto $R\equiv r(x)+s(x)y$  che deve essere identicamente nullo $\Rar$ $y=\frac{r(x)}{s(x)}$. Imponendo ora $A(x, y)\equiv 0$ si ottiene un assurdo.
\item Contare i poli, sono 8.
\bigskip
\hrule
\bigskip

%PicNic
\item Contarli sul toro: se $\omega_1$ e $\omega_2$ sono generatori, allora tutti i punti del tipo $P=\frac{k_1 \omega_1 +k_2 \omega_2}{n}$ con $k_1, k_2\in \bbN$, $k_i \leq n$ sono tutti e soli i punti con ordine che divide $n$. Sono $n^2$.
\item Usare la legge geometrica di gruppo sulla cubica (l'equazione della tangente intuitiva, quella che vale in $\bbR^3$ con le derivate funziona anche sui campi finiti). Vedere le dispense/punto sotto per una trattazione più completa
\item La tangente in un punto $P=(a:b:c)$ alla cubica $A(x, y, z)=0$ è $\{ (x:y:z)\in \bbP^2\bbC$ | $ (\partial_x A|_{P})x + (\partial_y A|_{P})y + (\partial_z A|_{P})z=0\}$. Per teorema di Eulero sulle funzioni omogenee (che per i polinomi vale anche su campi a caratteristica qualsiasi) ho una relazione tra le derivate parziali e il valore nel punto. Sfruttando questa relazione si vede che la retta tangente calcolata in carta affine (con metodi classici di geometria cartesiana) è la retta scritta sopra e passata in carta.
\item L'osservazione chiave è che la coordinata $x$ è una funzione pari del punto. Quindi, se vi fosse una dipendenza da una potenza dispari di $y$ nella formula, avrei $x(2P)\neq x(-2P)$. Tuttavia, sfruttando la relazione tra $x$ e $y$ so eliminare la dipendenza dalle potenze pari di $y$ $\Rar$ la formula non dipende da $y$. 
\bigskip
\hrule
\bigskip

% Miglio
\item Ripetere per induzione la dimostrazione sopra.
\item ({\it nota del Tasi:} così a botta non lo so fare, dovrei pensarci e non c'ho troppa voglia; se qualcuno lo risolve, è liberissimo di mandare la soluzione e qualche anima pia provvederà ad aggiungerla. Grazie cari)
\item Vedere dispense.
\bigskip
\hrule
\bigskip

% Prof
\item Vedere dispense.
\item Quando i tori da cui vengono ammettono un'isogenia.
\item Vedere dispense.
\bigskip
\hrule
\bigskip

%Fred
\item Vedere dispense. No, $j^2$ non lo è. Infatti, se per assurdo lo fosse, dovrebbe essere invariante anche per la mappa $z\mapsto 4z$. Di conseguenza avrebbe degli zeri che si accumulano ({\it nota del Tasi:} all'orale è bastato dire che sull'asse immaginario ci sono infiniti zeri che si accumulano in 0. Tuttavia 0 è fuori dal dominio di definizione di $j^2$, quindi mi pare non abbiamo ancora raggiunto un assurdo. Di nuovo, dovrei rifletterci con calma...).
\item Sono i punti fissi della formula di duplicazione: infatti, $3P=0 \Rar -P=2P \Rar x_{-P}=x_{2P}$. Da questo si ha che la coordinata x di un punto di 3-torsione deve risolvere un polinomio opportuno. 
\bigskip
\hrule
\bigskip

%Balbo
\item Vedere sopra
\item \'E il gruppo $G=\langle 4\rangle \rtimes \bbZ$ ($\langle 4 \rangle = \{ 4^i \in \bbC^* | i \in \bbZ\}$).
\item Sono i flessi della cubica. L'osservazione chiave è che, dati $P$ e $Q$ di 3-torsione, anche $P+Q$ ha ordine 3. Di conseguenza, $P$, $Q$ e $-P-Q$ sono allineati e tutti distinti.
\item Controimmagine della fibra generica.
\bigskip
\hrule
\bigskip

%Marinelo
\item Vedere dispense.
\item Vedere dispense.
\item Vedere dispense.
\item Vedere dispense.
\item Se $f=\frac{p(t)}{q(t)}$ con $p, q\in \bbC[t]$, si definisce deg($f$)=$|f^{-1}(y)|$ con $y\in \bbC$ valore regolare (ossia $f'\neq 0$ $\forall x\in f^{-1}(y)$.
	\begin{enumerate}[label=(\alph*)]
	\item Si può fare perchè i valori regolari sono in numero finito: sono le immagini degli zeri di $f'$ (che sono in numero finito).
	\item Non dipende dal valore regolare scelto perchè deg($f$)$=\\
		\sum_{x\in f^{-1}(y)}ord_{x}(f)=|\{\mbox{poli di } f-y\}|=|\{\mbox{poli di } f\}|$.
	\item Per le funzioni razionali vale deg$(f)=\mbox{max}(\mbox{deg}(p), \mbox{deg}(q))$.
	\end{enumerate}
\item Mostriamo le due disuguaglianze:
\begin{itemize}
	\item \framebox{deg($f$)$\geq[\bbC(t):\bbC(f)]$}\\
		L'estensione è primitiva, perciò il grado dell'estensione è il grado del polinomio minimo di $t$ su $\bbC(f)$. Se esibiamo un polinomio che fa zero su $t$ di grado deg($f$) abbiamo vinto. $A(x)=p(x)-f*q(x)$ funziona.
\bigskip
	\item \framebox{deg($f$)$\leq [\bbC(t):\bbC(f)]$}\\
\end{itemize}
\bigskip
\hrule
\bigskip
%Giova
\item Vedere dispense.
\item Vedere punto sotto. 
\item $f$ meromorfa $\Rar$ $f\in \bbC(\wp, \wp')$ $\Rar$ $f=R(\wp)+\wp'S(\wp)$. Ora isolo $\wp'$ e elevo la quadrato: ora ho $\wp'^{2}=$ funzione razionale di $f$ e $\wp$...
\item L'idea dovrebbe essere: $\faktor{\bbC(f)}{\bbC}$ ha dimensione di trascendenza 1. Anche $\faktor{\bbC(\wp, \wp')}{\bbC}$ ha dimensione 1; per cui, se esistesse una funzione meromorfa ellittica $g$ tale che $\faktor{\bbC(f,g)}{\bbC(f)}$ non sia un'estensione algebrica, avrei un sottocampo con dimensione trascendente maggiore di quella del campo totale. ({\it Nota del Tasi:} non ne so abbastanza di teoria dei campi per dire con certezza che la dimensione di trascendenza si comporta come la dimensione degli spazi vettoriale; tuttavia è l'unica "dimostrazione" che abbiamo trovato).
\bigskip
\hrule
\bigskip
%Cabe
\item 
\begin{enumerate}[label=(\alph*)]
	\item 
\begin{diagram}
	0 &	\rTo&	\Gamma(2)&	\rTo&	\bbP SL_{2}(\bbZ)&	\rTo&	\bbP SL_{2}(\bbF_{2})&	\rTo&	0\\
\end{diagram}
	\item $6(= 3*2)$
	\item $\frac{p^{2}(p^2-p)}{p-1}$ ({\it nota del Tasi:} questo numero è quello che ha trovato Cusu al volo, bisognerebbe controllare che sia davvero giusto).
	\item Si è accontentato di dire che doveva essere più grande di $F$ (quello di tutto $\bbP SL_{2}(\bbZ)$). 
	\item Non ce ne sono. La prima osservazione da fare è che se c'è un punto fisso, allora ne trovo uno anche dentro $F$ (uso che $\Gamma(2)$ è normale in $\bbP SL_{2}(\bbZ)$). Da qui si conclude esattamente come nelle dispense, tenedo a mente come sono fatte le matrici di $\Gamma(2)$.
\end{enumerate}
\item Vedere dispense.
\bigskip
\hrule
\bigskip

%Magico Cusu
\item Intanto si osserva che, o $c\equiv 0$ oppure $c:L\rar \bbC^*$ è un omomorfismo di gruppi. Inoltre, a meno di sostituire $f(z)$ con $f(z)e^{kz}$, posso supporre che $c(\omega_1)=1$ (con $\omega_1\in L$ un generatore). A questo punto non si è più capito quale fosse la domanda... 
\item \'E ancora una cubica perchè quozientando il toro per un elmento di ordine due ottengo un altro toro, e la proiezine al quoziente è una isogenia di grado 2.
\item Questa è veramente mistica: Cusu ha smanettato un po' con la $\wp$ ma non ne ha cavato fuori troppo. Tasi ha provato con la $\wp'$ e a scrivere in qualche forma decente i coefficienti della cubica associata al toro quoziente: Nada...

\bigskip
\hrule
\bigskip
%Tasi
\item Vedere sopra.
\item Vedere sopra.
\item Vedere dispense.

\end{enumerate}


