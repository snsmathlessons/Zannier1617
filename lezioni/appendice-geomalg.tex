\chapter{``Prerequisiti''}
\justify

\newthought{Qui raccogliamo le principali nozioni} di geometria algebrica che nel corso vengono date per scontate.
Forse sarebbe più opportuno metterla come introduzione ma forse anche no.

\vskip 2.0em
{\itshape Per mancanza di tempo e di volontari i prerequisiti non sono stati scritti.}
\vskip 2.0em

Mi sento però di consigliarvi di leggere il libro ``\usebibentry{Reid88}{title}'' di \citet{Reid88} per un'introduzione ``veloce'': il libro ha circa centotrenta pagine, che non sono poche, ma lo trovo abbastanza informale e con sufficienti disegni tali da poter comunicare le idee in maniera veloce.
Inoltre vi sarà ovvio leggendolo che alcune parti proprio non vi servono.

\vskip 2.0em
\noindent Argomenti da mettere nei prerequisiti:
\begin{itemize}
\item Calcolo della retta tangente in un punto in coordinate affini e proiettive.
\item Teorema di Bèzout.
\item Relazione tra Flessi ed Hessiano.
\item Quanti sono i flessi di una cubica? Mostrare che sono tutti distinti.
\end{itemize}

