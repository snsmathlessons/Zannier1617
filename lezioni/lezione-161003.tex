\chapter{3 Ottobre 2016 - Automorfismi delle superfici di Riemann semplicemente connesse}
\justify

\begin{definizione}[Superficie di Riemann (speriamo)]
Si dice superficie di Riemann una varietà complessa connessa di dimensione $1$, cioè uno spazio topologico di Hausdorff (T2)
connesso tale che ogni suo punto abbia un intorno $U_\alpha$ omeomorfo a un aperto $V_\alpha$ di $\mathbb{C}$
tramite l'omeomorfismo $\varphi_\alpha$, e che per ogni $\alpha$ e $\beta$ valga che $\varphi_\beta \circ {\varphi_\alpha}^{-1}$
sia una funzione olomorfa in $\varphi_\alpha \left( U_\alpha \cap U_\beta \right)$.
\end{definizione}

\begin{osservazione}
La richiesta di essere T2 non si deduce dalle altre condizioni. Infatti se consideriamo il "disco con due origini",
cioè il quoziente $S/\!\!\sim$ dove $S=D\times\{0\} \cup D\times\{1\}$ e $x\sim y \Longleftrightarrow x=y$ oppure $x=(u,i),\ y=(u,j)$ con $\{i,j\}=\{0,1\}$,
si osserva che ogni punto ha un intorno omeomorfo a un aperto di $\mathbb{C}$ ma $(0,0)$ e $(0,1)$ non hanno intorni disgiunti.
\end{osservazione}

\begin{esercizio}
$S/\!\!\sim$ è semplicemente connessa.
\end{esercizio}

Determiniamo ora i gruppi degli automorfismi delle superfici di Riemann semplicemente
connesse: $\mathbb{P}^1(\mathbb{C})$, $\mathbb{C}$, $D$.


\section{Automorfismi di $\mathbb{P}^1(\mathbb{C})=\hat{\mathbb{C}}$}
Vogliamo dimostrare che $Aut( \hat{\mathbb{C}} )= \mathbb{P}GL_2 (\mathbb{C} )=\left\{z\mapsto \displaystyle{\frac{az+b}{cz+d}} \ ,\ \  ad\neq bc\right\}$

\begin{definizione}[Grado di una funzione razionale]
Si definisce grado di $\frac{p(x)}{q(x)}$ il massimo tra i gradi di $p(x)$ e $q(x)$.
Si può osservare che, come per i polinomi, il grado è il numero di controimmagini di ogni punto contate con molteplicità.
\end{definizione}

\begin{osservazione}
$\varphi \in Aut( \hat{\mathbb{C}} ) \Longrightarrow deg(\varphi )=1$, perché deve essere un omeomorfismo e quindi è iniettivo. 
\end{osservazione}


\begin{lemma}
$G= \mathbb{P}GL_2 (\mathbb{C} )$ è triplamente transitivo, cioè date due terne
$(x_1 , x_2 , x_3 ) \in \hat{\mathbb{C}}^3$ e $(y_1 , y_2 , y_3 ) \in \hat{\mathbb{C}}^3$
con $x_i \neq x_j$ e $y_i \neq y_j$ se $i \neq j$, $\exists ! g \in G$ t.c. $g(x_i )=y_i$ per $i=1,2,3$.
\end{lemma}
\begin{proof}
Basta farlo per $(x_1 , x_2 , x_3 ) = (0,1,\infty )$ e $y_i \neq \infty$, a meno di comporre con qualche elemento di $G$.
Devo quindi trovare $a,b,c,d \in \mathbb{C}$ tali che:
$$
\left\{
\begin{array}{l}
\medskip
\displaystyle{\frac{0a+b}{0c+d}=\frac{b}{d}=y_1} \\
\medskip
\displaystyle{\frac{1a+b}{1c+d}=\frac{a+b}{c+d}=y_2} \\
\displaystyle{\frac{\infty a+b}{\infty c+d}=\frac{a}{c}=y_3}
\end{array}
\right.
$$
da cui sostituendo si ottiene $y_3 c + y_1 d=y_2 (c+d)$ e si vede che, se $y_i \neq y_j$ quando $i \neq j$, esistono $c$ e $d$ con $c+d \neq 0$ che risolvono l'equazione.

Resta quindi da mostrare l'unicità. Per farlo basta verificare che l'unico elemento di $G$ che fissa una data terna
(prendiamo ancora $(0,1,\infty )$) è l'identità. Imponiamo quindi che:
$$
\left\{
\begin{array}{l}
\medskip
\displaystyle{\frac{0a+b}{0c+d}=\frac{b}{d}=0} \\
\medskip
\displaystyle{\frac{1a+b}{1c+d}=\frac{a+b}{c+d}=1} \\
\displaystyle{\frac{\infty a+b}{\infty c+d}=\frac{a}{c}= \infty}
\end{array}
\right.
$$
da cui si ricava subito $a=d=1$, $b=0$ e $c=0$, cioè la trasformazione considerata è l'identità.
\end{proof}

\begin{esercizio}
I sottogruppi finiti di $\mathbb{P}GL_2 (\mathbb{C} )$ sono:
\begin{itemize}
\item Tutti i gruppi ciclici
\item Tutti gruppi diedrali
\item $S_4$ e $A_5$
\end{itemize}
\end{esercizio}


\begin{proposizione}
$Aut (\hat{\mathbb{C}} ) = G = \mathbb{P}GL_2 (\mathbb{C} )$
\end{proposizione}
\begin{proof}
Sia $\varphi \in Aut (\hat{\mathbb{C}} )$. Sappiamo già che $Aut (\hat{\mathbb{C}} ) \supseteq G$.
Quindi, essendo $G$ triplamente transitivo, possiamo assumere $\varphi( \infty ) = \infty$.
Ora osserivamo che, essendo $\varphi$ un automorfismo olomorfo, detto $D=D(0,1)$, $\varphi (D)$ conterrà
un certo disco $D_1$. Quindi $\varphi$ non può avere una sigolarità essenziale all'infinito poiché in tal caso,
per il teorema di Weierstrass, dovrebbe assumere valori densi in $\mathbb{C}$ in ogni intorno di $\infty$
e quindi anche valori in $D_1$ contraddicendo l'iniettività di $\varphi$.
Allora $\varphi$ ha un polo all'infinito ed è olomorfa su $\mathbb{C}$, quindi deve essere un polinomio (basta considerare la sua espansione in serie) e,
essendo anche iniettiva, deve avere grado $1$. Quindi $\varphi \in G$.
\end{proof}


\section{Automorfismi di $\mathbb{C}$}
Studiamo ora gli automorfismi olomorfi di $\mathbb{C}$.
Sia $g \in Aut(\mathbb{C})$. Allora, come prima, per il teorema di Weierstrass $g$ ha un polo all'$\infty$,
quindi è un polinomio e ha grado $1$ per iniettività. Quindi
$Aut(\mathbb{C} )= \left\{ z\mapsto az+b \right\} = Stab_{\mathbb{P} GL_2 (\mathbb{C})} (\infty )$.


\section{Automorfismi di $D$}
\begin{definizione}[Trasformazioni di Möbius]
Si chiamano Trasformazioni di Möbius le trasformazioni $D \rar D$ del tipo $z\mapsto c \displaystyle{\frac{z-a}{1- \bar{a}z}}$ con $|c|=1$ e $a \in D$.
\end{definizione}

\notamargine{A volte invece si chiamano così tutte le trasformazioni di $\mathbb{P} GL_2 (\mathbb{C})$}
%Esce troppo in alto. Come faccio a metterla vicino alla definizione?

\begin{esercizio}
Le trasformazioni di Möbius sono un sottogruppo di $\mathbb{P} GL_2 (\mathbb{C})$.
\end{esercizio}

\begin{osservazione}
Le trasformazioni di Möbius mandano D in se stesso. Infatti: 
$$\left|c \frac{z-a}{1- \bar{a}z} \right|^2 = 1 \frac{(z-a)(\bar{z}-\bar{a})}{(1- \bar{a}z)({1- a\bar{z}})}=
\frac{|z|^2 + |a|^2 - 2Re(a \bar{z})}{1+ |a|^2 |z|^2 - 2Re(a \bar{z})} <1
\Longleftrightarrow \left( 1-|z|^2 \right) \left( 1-|a|^2 \right) >0$$
Questa è vera $\forall z \in D$.
\end{osservazione}

\begin{osservazione}
Il gruppo delle Trasformazioni di Möbius G ha dimensione topologica 3 (è come $S^1 \times D$),
quindi ci aspettiamo che sia transitivo ma non doppiamente transitivo.
\end{osservazione}

\begin{lemma}
$G= \left \{ z \mapsto \displaystyle{c \frac{z-a}{1- \bar{a}z}} \right \}$ è transitivo.
\end{lemma}
\begin{proof}
È evidente che si può mandare $0$ in qualsiasi punto di $D$ (anche ponendo $c=1$).
\end{proof}

\begin{osservazione}
$Stab_G (0) = \{z \mapsto cz \}$ cioè tutte e sole le rotazioni ($|c| = 1$).
\end{osservazione}

\begin{lemma}[di Schwarz]
Sia $f:D\rightarrow D$ una funzione olomorfa tale che $f(0)=0$. 
Allora $|f(z)| \leq |z|\ \ \forall z \in D$ e $|f'(0)| \leq 1$. Inoltre, se vale la prima uguaglianza anche solo in un punto, $f$ è una rotazione.
\end{lemma}
\begin{proof}
$f(0)=0 \Rightarrow \frac{f(z)}{z}$ è olomorfa in $D$. Detto $D_r$ il disco di raggio $r$,
per il principio del massimo modulo si ha che, per $z \in D_r$
$$\left| \frac{f(z)}{z} \right| \leq \sup_{|z|=r} \left| \frac{f(z)}{z} \right| \leq \frac{1}{r} \ \  \forall r \in (0,1)
\Longrightarrow \left| \frac{f(z)}{z} \right| \leq 1 \ \ \forall z \in D$$
Facendo tendere $z$ a $0$ si ottiene anche la tesi sulla derivata.
Infine, se si verifica un'uguaglianza, sempre per il principio del massimo modulo risulta che
$\left| \frac{f(z)}{z} \right|$ è costante in $D$, quindi $f$ è una rotazione.
\end{proof}
\notamargine{In realtà $f$ è una rotazione anche se vale solo $|f'(0)| = 1$ (sviluppare in serie attorno all'origine).
Infatti poi usiamo questo enunciato...}

\begin{proposizione}
$Aut(D)=G$
\end{proposizione}
\begin{proof}
Sappiamo che $Aut(D) \supseteq G$. Sia quindi $\varphi \in Aut(D)$. Essendo $G$ transitivo posso supporre che $\varphi(0)=0$.
Sia $\psi$ l'inversa di $\varphi$. $\psi (0)=0$. Quindi
$\psi (\varphi (z))=z \stackrel{deriviamo}{\Longrightarrow} \psi ' (\varphi (z)) \varphi '(z) =1 \Rightarrow \psi'(0) \varphi'(0) =1$
Ma per il lemma di Schwarz $|\psi'(0)| \leq 1$ e $|\varphi'(0)| \leq 1$. Quindi $|\varphi'(0)|=1$ e, sempre per il lemma di Schwarz,
$\varphi$ è una rotazione e quindi appartiene a $G$. 
\end{proof}

\subsection {$D \simeq \mathcal{H}$}

Vediamo ora che il disco $D$ è biolomorfo al semipiano $\mathcal{H} = \left \{ z \in \mathbb{C} | \ Im (z)>0 \right \}$
Presi $w \in \mathcal{H}$ e $z \in D$, consideriamo le mappe:
$$w \mapsto z=\frac{i-w}{i+w} \qquad z \mapsto w= i\frac{1-z}{1+z}$$
Si può verificare sono una l'inversa dell'altra e che sono effettivamente omeomorfismi tra $\mathcal{H}$ e $D$.
Quindi $D \simeq \mathcal{H}$ come superfici di Riemann.

Questo ci permette anche di determinare che $Aut( \mathcal{H}) = \left \{ z \mapsto \displaystyle{\frac{az+b}{cz+d}} \mid ad > bc \right \} = \GL^+ (2, \bbR)$

Ora quindi dovremo capire quali sottogruppi di $Aut (\hat{\mathbb{C}})$, $Aut(\mathbb{C})$ e $Aut(D)$ agiscono in modo propriamente discontinuo
e non hanno punti fissi, per classificare le superfici di Riemann.