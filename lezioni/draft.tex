%% Lezione di esempio. Copiate questo file nella lezione che dovete creare
%% per avere già uno scheletro di come scrivere le lezioni

%% Diamo un nome al capitolo. Idealmente mettiamo la data della lezione ed
%% una sua breve descrizione / argomenti trattati
\chapter{DD/MM/YY - Lezione di esempio}

%% Le parole dentro a \newthought vengono scritte in maiuscoletto. Usatele
%% per sottolineare meglio alcune parole ad inizio paragrafo
\newthought{Queste} poche righe di esempio servono per dare uno scheletro
al modo in cui si dovrebbe scrivere una lezione: useremo un capitolo per
ogni lezione e, all'interno di questi divideremo il testo in vari paragrafi
e/o varie sezioni come si può vedere da questo file.

%% Le cose scritte dentro a \footnotetext appaiono come note al margine
\notamargine{Le scritte dentro all'elemento notamargine appaiono al fianco
  della pagina, come questa scritta. Le si può utilizzare per dare alcune
  spiegazioni su punti che sono poco chiari, o per mettere dei riferimenti
  ad alcuni libri che spiegano meglio l'argomento}

Inoltre potete scrivere le lettere accentate tranquillamente dentro al
sorgente, infatti ho importato il pacchetto opportuno che ci consente di
inserire àèìòùy come se fosse antani e verranno visualizzate bene nel file.

%% Si può poi inserire una section per scrivere degli argomenti specifici
\section{Argomento Rilevante}
Qui siamo ad esempio all'interno di una section e possiamo scrivere anche
alcune formule $\forall x \in X \quad \varphi(x) = x^2$. Oppure anche su
una linea a parte per le cose significative
    $$ \int_a^b \cos(x)\sin^2(x) dx $$

\notamargine{Notate come per separare un $\forall$ dalla parte successiva
  è stato utilizzato un quad, per lasciare un po' di spazio.}

\paragraph{Idea della Dimostrazione} Paragrafetti come questo si possono
utilizzare per dare un'idea della dimostrazione, oppure per far notare un
fatto importante o subdolo che potrebbe sfuggire.

Si possono inoltre inserire delle tabelle come la seguente nel caso ci sia
bisogno di distinguere un po' di casi in maniera ordinata:

\begin{table}[h]
  \caption{Nota a margine della tabella senza senso, che potrebbe contenere
  il titolo della stessa ed anche qualche spiegazione se fosse il caso}
  \begin{center}
    \footnotesize % Gli diciamo di farla in carattere piccolo
    \begin{tabular}{lcr}
      \toprule % Per poter avere la linea sopra la tabella
      Prima colonna & Seconda colonna & Terza Colonna \\
      \midrule % Per la linea a metà, subito sotto gli header
      Misure & Gruppi & Senza senso \\
      Grafi & Integrali & Efelanti \\
      Numeri & Operazioni & Schifo \\
      \bottomrule % Mettiamo una riga anche sotto alla tabella
    \end{tabular}
  \end{center}
\end{table}

\paragraph{Nota} Utilizzate gli elementi teorema, definizione, lemma tra i
tag begin ed end in modo da avere un modo uniforme di scriverli. Sono presenti
gli elementi seguenti: teorema, definizione, lemma, osservazione, remark,
proof, corollario.

\begin{osservazione}
  Scriviamo qualche osservazione importante tipo $2 = 1 + 3$
\end{osservazione}

\begin{lemma}[Lemma del Grande Puffo]
  Potete dare un nome ai lemmi e scriverci davvero dei lemmi dentro tipo
  $a = 1$ che è un importantissimo lemma
\end{lemma}

\begin{teorema}[del Gelato]
  $1 = 1$
\end{teorema}
\begin{proof}
  Ovvia
\end{proof}

%% Le subsection sono, ovviamente, le sottosezioni. State attenti però che
%% con il tipo di testo che stiamo usando le subsubsection non esistono
\subsection{Prime Definizioni}
\begin{Verbatim}
    Potreste voler scrivere qualcosa esattamente come appare, ad esempio
    se ci fosse bisogno di inserire del codice, ma non penso proprio che
    ce ne sia bisogno.
\end{Verbatim}

\subsection{Teorema Principale}


\section{Altro Argomento}
Ormai non so più cosa scrivere, e dovreste aver imparato i principali comandi
utili. Nel caso qualcosa non fosse chiaro non esitate a chiedere.

\notamargine{E non esitate nemmeno a cercare su Google}
