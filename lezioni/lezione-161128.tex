\chapter{28 Novembre 2016 - Mappa tra il Toro e la cubica corrispondente}
\justify

\newthought{La volta scorsa}, dato un reticolo $L \subseteq \bbC$, avevamo
costruito la funzione di Weierstrass associata $\wp_L (z)$, che soddisfa
un'equazione del tipo $\wp'^2(z) = 4 \wp^3(z) - g_2 \wp(z) - g_3$ dove
$g_2 = 60 s_4$ e $g_3 = 140 s_6$ con $s_n = \sum_{\omega \in L^*}
\omega^{-n}$.

Sia $E: y^2 = 4x^3 - g_2 x - g_3$ (il luogo di zeri in $\bbC^2$) e
consideriamone la chiusura proiettiva $\ol{E} = E \cup \{ (0 : 1 : 0)
\}$.

\begin{notazione}
  Con $\ol{E}(\bbK)$ si intendono i punti $\bbK$-razionali di $\ol{E}$,
  ovvero i punti di $\bbP^2 \bbC$ per le quali il rapporto tra le
  coordina è un numero di $\bbK$, ovvero
  $\ol{E}(\bbK) = \{ (x : y : z) \in \ol{E} \mid \frac{x}{y},
  \frac{y}{z} \in \bbK \}$
\end{notazione}

\newthought{Consideriamo la mappa} $F: T = \sfrac{\bbC}{L} \rar \ol{E}$
dal toro relativo al reticolo $L$ alla chiusura della cubica. $F$ è
definita da $z \mapsto (\wp(z) : \wp'(z) : 1)$ dove questa espressione
ha senso. Quando invece $z$ è un polo di $\wp$, si può usare un'altra
espressione per la mappa, definita sul polo, ma che coincida con
quella fornita sui punti in cui si intersecano gli aperti di definizione.

\notamargine{Ad esempio si può considerare l'espressione
  $$\lbr{1 : \frac{\wp'(z)}{\wp(z)} : \frac{1}{\wp(z)}}$$

  Notiamo che stiamo trattando $T$ come una varietà olomorfa e ciò che
  stiamo facendo è definire una mappa tra varietà su alcune carte
}

\begin{proposizione}
  $F$ è una biggezione tra $T$ ed $\ol{E}(\bbC)$ (ed è olomorfa)
\end{proposizione}
\begin{proof}
  Sia $p \in \ol{E}(\bbC)$. Distinguiamo in due casi:
  \begin{itemize}
  \item Se $p = (0 : 1 : 0)$ ok per costruzione (significa che la $\wp$
    ha un polo e quindi come unico punto abbiamo $z = 0 \in T$)
  \item Se $p \in E(\bbC)$ (nella parte affine della cubica) sia $p =
    (x_0, y_0)$.

    Consideriamo allora $\wp(z) - x_0$ che è una funzione ellittica pari
    non costante e quindi ha due zeri (con molteplicità). Se $z_0$ è uno
    zero lo è anche $- z_0$ (dove i punti si intendono sul
    toro). Distinguiamo in due casi a seconda della derivata nel punto:
    \begin{itemize}
    \item Se $\wp'(z_0) = 0$ allora $z_0$ è uno zero doppio e deve
      quindi coincidere con $-z_0$ sul toro, ovvero nel piano ho
      solamente quattro possibilità: sono la metà dei generatori del
      parallelogramma fondamentale del toro.
      % TODO: Disegno del parallelogramma fondamentale con i punti evidenziati

      Allora si ha un solo punto tale che $4x_0^3 - g_2 x_0 - g_3 = 0 (=
      \wp'(z_0))$ e quindi anche questo caso è ok
    \item Se $\wp'(z_0) \neq 0$ allora si ha $z_0 \neq - z_0$ in $T$ e
      quindi $\wp'(z_0) = - \wp'(-z_0)$ poiché la $\wp'$ è
      dispari. Siamo allora nel caso $4x_0^3 - g_2 x_0 - g_3 = \alpha
      \neq 0$ e per avere $y = \pm \sqrt{\alpha}$ posso usare $z_0$ ed
      anche $-z_0$, ovvero ancora una volta la tesi è verificata.
    \end{itemize}
  \end{itemize}

  Per quanto riguarda l'olomorfia, basta aggiungere che la mappa fornita
  $z \mapsto (\wp(z) : \wp'(z) : 1)$ è ovviamente olomorfa tra $T$ e
  $\bbP^2 \bbC$
\end{proof}

\begin{osservazione}
  Notiamo però che se mi viene dato un polinomio di terzo grado nella
  forma $y^2 = 4x^3 - g_2 x - g_3$ non so ancora dire se venga da un
  toro oppure no. Se viene da un toro allora ho la corrispondenza
  biunivoca (proposizione precedente) $\sfrac{\bbC}{L} \leftrightarrow T$
\end{osservazione}

\begin{osservazione}
  Se abbiamo un polinomio in due variabili $f(x, y) = 0$ abbiamo una
  superficie di Riemman, ma non avevamo dimostrato (nel caso delle
  cubiche) la connessione, che in generale non è ovvia.

  Nel caso delle cubiche ciò segue dalla proposizione appena dimostrata.
\end{osservazione}

\paragraph{Altra dimostrazione della connessione} Dimostriamo ancora una
volta che $\ol{E}(\bbC)$ è connesso, iniziando dalla connessione di
$E(\bbC)$ (spazio dei punti affini)

La funzione $\bbC \rar \bbC$ definita da $x \mapsto 4x^3 - g_2 x - g_3$
diventa un rivestimento tolte le tre radici $\alpha, \beta, \gamma$ del
polinomio $4x^3 - g_2 x - g_3$.

\notamargine{L'idea è che, tolte le radici, abbiamo un rivestimento di
  grado due. I punti del rivestimento si riescono a connettere ``facendo
  dei giri attorno alle radici''}

Ovviamente si ha che, avendo il rivestimento grado due, o è connesso,
oppure ha due componenti connesse. Ora, se facciamo un cammino tra due
punti $x_0$ ed $x_1$ del $\bbC$ ``che viene ricoperto'' posso sollevare
il cammino, perché ho un rivestimento. Per mostrare la connessione del
rivestimento basta quindi dimostrare che la fibra di $x_0 \neq \alpha,
\beta, \gamma$ ``è connessa'', ovvero ho un cammino che connette i due
elementi della fibra. E ciò è molto semplice da fare con la radice
quadrata:

Prendiamo infatti un punto $x_0$ vicino ad $\alpha$. Allora si ha che
$\sqrt{f(x)} = (x - \alpha)^{\frac{1}{2}} \cdot g(x)$ con $g(x)$
olomorfa vicino ad $\alpha$ (poiché siamo lontani da $\beta$ e da
$\gamma$)

% TODO: Fare i conti espliciti o motivarlo meglio
% TODO: Aggiungere disegno di Zannier

Scegliamo una determinazione della radice e facciamo un giro attorno ad
$\alpha$ tornando sulla fibra di $x_0$ ma con valore del segno cambiato.

\section{Cubiche di $\bbP^2\bbC$}
Consideriamo tutte le cubiche in $\bbP^2\bbC$: esse sono definite da
un'equazione omogenea di grado tre $F(x, y, z) = 0$. Mostreremo ora che
nel caso la cubica è parametrizzabile se e solo se è singolare.

\notamargine{Ricordiamo che per parametrizzazione intendiamo una
  isomorfismo birazionale con $\bbP^1$, ovvero una funzione razionale
  $f: \bbP^1 \rar E$ che si intende definita da quasi tutti i punti di
  $\bbP^1$ a quasi tutti i punti di $E$. (Quasi tutti significa tutti
  tranne al più un numero finito)}

\subsection{Caso non singolare}
In ogni cubica c'è sempre un punto di flesso (ovvero dove l'hessiano si
annulla). Usando una proiettività si può mandare il flesso nel punto
all'infinito $(0 : 1 : 0)$.

\notamargine{L'Hessiano in questo caso è il determinante della matrice
  hessiana del polinomio che definisce la cubica, ovvero
  $\Det (\frac{\partial F}{\partial x_i \partial x_j})_{i, j = 1, 2, 3}$

  Il polinomio hessiano ha, per un facile conto, grado $3 (d-2)$, dove
  $d$ è il grado della curva $F$, oppure è identicamente nullo.
  Nel primo caso, per il teorema di Bèzout, esistono punti di $\bbP^2$
  su cui esso si annulla assieme all'equazione di $F$, ovvero esistono
  punti di $F$ di flesso (e sappiamo anche che sono $3 (d - 2) d$
  contati con molteplicità)}

Facendo i conti con il flesso all'infinito e con tangente al flesso la
retta $\{ z = 0 \}$ ottengo un'equazione simile a quella di Weierstrass
a cui si arriva con poche manipolazioni algebriche:
$y^2 = 4 x^3 - a x - b$.

Se poi effettuiamo la trasformazione $x \mapsto \lambda^2 x$ e $y
\mapsto \lambda^3 y$ con $\lambda \neq 0$ allora si ottiene $y^2 = 4 x^3
- \frac{a}{\lambda^4} x - \frac{b}{\lambda^6}$.

Posso quindi scegliere $\lambda$ in modo da far scomparire un parametro.

\begin{osservazione}
  Non sappiamo ancora che ogni cubica viene da un toro, quindi non
  potevamo dire a priori che lo spazio delle cubiche ha un solo
  parametro.
\end{osservazione}

Che succede se al posto di un reticolo $L$ ne prendiamo uno omotetico
$\lambda L$ con $\lambda \in \bbC^*$? Le funzioni $g_2$ e $g_3$ si
trasformano nel seguente modo:
$ g_2 \mapsto \frac{g_2}{\lambda^4} ,\quad g_3 \mapsto
\frac{g_3}{\lambda^6}$ e quindi l'equazione della cubica diventa:
$E_{\lambda L}: y^2 = 4x^3 - \frac{g_2}{\lambda^4} x -
\frac{g_3}{\lambda^6}$

Quali sono le funzioni razionali di $a$ e di $b$ che rimangono
invarianti per trasformazioni omotetiche? Sicuramente vi è
$\frac{a^3}{b^2}$.
\notamargine{Per esercizio si può dimostrare che ogni funzione
  invariante per omotetia di $a$ e di $b$ si scrive come funzione
  razionale di $\frac{a^3}{b^2}$}

Dato un reticolo $L$, lo scriviamo come $\bbZ \tau + \bbZ$ con $\tau \in
\cH$ (semipiano superiore) a meno di rotomotetie del piano.

\begin{proposizione}
  $\frac{g_2^3}{g_3^2}$ è una funzione di $\tau$ non costante e meromorfa
\end{proposizione}

Per quanto riguarda la meromorfia, si può notare che, definita la somma
$$S_n = \sum_{(r, s) \in \bbZ^2 \setminus \{(0, 0)\}} \frac{1}{(r \tau
  + s)^n}$$ essa converge per $n \ge 2$ e definisce una funzione olomorfa
di $\tau$. Segue quindi che, se $S_3$ non è identicamente nulla, si ha
che $\frac{g_2^3}{g_3^2} = \frac{S_2^3(\tau)}{S_3^2(\tau)}$ per
definizione e quindi la succitata funzione è meromorfa.

\notamargine{Per verificare la convergenza assoluta della sommatoria per
  $n \ge 2$ si può maggiorare la serie dei valori assoluti con
  l'opportuno integrale in due variabili}

Inoltre, se trovassimo due reticoli $L_1$ ed $L_2$ in cui valga
rispettivamente
$$g_3(L_1) = 0, g_2(L_1) \neq 0$$
$$g_2(L_2) = 0, g_3(L_2) \neq 0$$
ne seguirebbe che la funzione non è costante.

Si possono a tal proposito considerare i reticoli $\tau = i$ e
$\tau = \zeta_3$:
\begin{itemize}
\item ($\tau = i$) Sia $L_1 = \bbZ[i]$. Allora si nota che $i L_1 = L_1$
  e quindi si ha $g_2(i L_1) = g_2(L_1)$ e per omogeneità
  $g_3(L_1) = g_3(i L_1) = i^{-6} g_3(L_1)$ da cui segue $g_3(L_1) = 0$.

  Inoltre $g_2 \neq 0$, altrimenti il polinomio definente avrebbe radici
  multiple.
  % TODO: Da controllare l'affermazione

\item ($\tau = \zeta_3$) Sia $L_2 = \bbZ[\zeta_3]$ e notiamo che
  $\zeta_3 L_2 = L_2$ che implica $g_2(L_2) = g_2(\zeta_3 L_2) =
  \zeta_3^4 g_2(L_2) = \zeta_3 g_2(L_2)$ e quindi $g_2(L_2) = 0$
\end{itemize}

