\chapter{28/09/16 - Non ho ancora ben capito su cos'è questa lezione}
\justify
%Balbo sappi che ti sto odiando per tutti quei comandi definiti un po' in modo libertino.

\newthought{Eravamo rimasti} a questo enunciato:
\begin{teorema}
	Sia $M$ una varietà, e sia $G$ un gruppo di omeomorfismi di $M$.
 	Allora la proiezione al quoziente $p:M\rar \quoziente{M}{G}$ è un rivestimento se e solo se $G$ agisce in modo propriamente discontinuo.
\end{teorema}
\notamargine{$\Rar$ è ovvia e $\Lar$ sta sul Manetti}

Qui, seguendo il Manetti, intendo che $G$ agisce in modo propriamente discontinuo se $\forall\ x\in X\ \exists\ U\subset X$ aperto tale che $\forall\ g\in G, g(U)\cap U=\emptyset$. 

Se ho qualche piccola ipotesi su X (essere varietà basta e avanza), questo è equivalente a chiedere che $G$ sia libero, cioè che $\forall\ x \in X, \forall\ g\in G, g(x)\neq X$, e che $\forall\ K \subset X$ compatto, $g(K)\cap K \neq \emptyset$ solo per un numero finito di $g$.
\notamargine{$\Lar$ sta sul Manetti, $\Rar$ ha una parte ovvia e una che usa compattezza per successioni}
%Nota: Capire perché le note a margine sono impaginate schifo

A questo aggiungo un altro teorema:

\begin{teorema} \label{rivuniv}
 	Sia $p:E\rar X$ un rivestimento universale. Allora esiste $G$ gruppo di omeomorfismi di $E$ propriamente discontinuo tale che $X\isom\quoziente{E}{G}$.
\end{teorema}
\notamargine{Ricordo che $p:E\rar X$ rivestimento si dice universale se $E$ è semplicemente connesso}
\notamargine{Anche qui la dimostrazione sta sul Manetti, in posti a caso}

Adesso, sia $X$ una \sdR, e sia $\pi:X\rar\ot X$ il suo rivestimento universale (che ricordo essere unico a meno di omeomorfismi). Attraverso $\pi$, $\ot X$ eredita una struttura di \sdR\ da quella di $X$.

Infatti, prendendo un raffinamento $U_\alpha$ del ricoprimento su $X$ dato dalla definizione di \sdR, posso pensare che gli $U_\alpha$ sono aperti banalizzanti per $\pi$, ottenendo dunque un ricoprimento su $\ot X$ dato dagli insiemi $V_\alpha^r$, dove $\pi^{-1}(U_\alpha)=\sqcup_r V_\alpha^r$, in cui le carte locali sono le funzioni $\phi_\alpha^r=\phi_\alpha\circ\pi$. Il fatto che i cambi di carta siano olomorfi su $\ot X$ discende dal fatto che lo siano su $X$.

Ora, per il teorema \ref{rivuniv}, esiste $G$ fatto nel modo giusto tale che $\quoziente{\ot X}{G}\isom X$.
\begin{lemma}
 	%Balbo ti odio, perché non potevi mettere tra i similteoremi anche proposizione?
 	Sia $g\in G$. Allora $g$ è olomorfa come applicazione tra \sdR.
\end{lemma}
\begin{proof}
 	Osservo che ovviamente $g \in \mbox{Aut}\left(\quoziente{\ot X}{X}\right)$, cioè $g$ è un automorfismo di rivestimento (in inglese \emph{deck transformation}), cioè $g$ preserva le fibre. Questa condizione si può scrivere anche come $\pi\circ g=\pi$.
 	
 	Sia ora $x\in \ot X$. Sia $V_\alpha^r$ uno degli aperti della condizione di \sdR\ tale che $x\in V_\alpha^r$, e sia $\phi_\alpha^r$ la sua carta locale. Si ha $g(x)\in V_\alpha^s$ per un qualche $s$, e dunque bisogna dimostrare $\phi_\alpha^s\circ g\circ (\phi_\alpha^r)^{-1}$ olomorfa in $\phi_\alpha^r(x)$. Ma questa funzione si può scrivere anche come $\phi_\alpha\circ\pi\circ g\circ \pi^{-1}\circ \phi_\alpha^{-1}$, che poiché $g$ è automorfismo di rivestimento è uguale a $\phi_\alpha\circ\pi\circ\pi^{-1}\circ\phi_\alpha^{-1}=id$. Poiché l'identità è olomorfa, abbiamo dimostrato che $g$ è olomorfa.
\end{proof}
\notamargine{$\pi^{-1}$ dovrebbe presentare un pedice che si riferisce al fatto che ha codominio $V_\alpha^r$, ma non l'ho ritenuto fondamentale}

Ora, per classificare le \sdR\ posso dunque studiare tutti i quozienti opportuni di \sdR\ semplicemente connesse. Per il Teorema di Riemann, queste sono solo $\bbP_1(\bbC)=\oc \bbC, \bbC, \cD\isom\cH$, dove su quest'ultimo ho la struttura complessa data dall'inclusione.

Dunque il prossimo passo è classificare i gruppi di isomorfismi propriamente discontinui e olomorfi delle tre \sdR\ semplicemente connesse.

\newthougth{Ora iniziano} cose di cui non ho esattamente capito il senso, per ora.







