\chapter{19/09/16 - Breve Descrizione della lezione}


Funzioni con un periodo.
Arriviamo a studiare le funzioni con periodo $\pi$
Si può vedere topologicamente che è un cilindro. A meno di rinormalizzazioni potrò
considerare $\mathbb{C}/\mathbb{Z}$

\begin{definizione}
Si dice una superficie di Riemann una varietà complessa connessa di dimensione $1$.
\end{definizione}

\begin{proposizione}
Le funzioni meromorfe con periodo $z_0$ sono tutte e sole quelle della forma $g\left(e^{2\pi i/z_0}\right)$
\end{proposizione}


Proposizione:

Sia $f: \mathbb{C}\rightarrow\mathbb{C}$ una funzione meromorfa. Allora l'insieme $L$ dei periodi di $f$
forma un sottogruppo additivo di $\mathbb{C}$.

Dimostrazione: esercizio.
