\chapter{19/09/16 - Introduzione}

In questo corso studieremo le funzioni meromorfe periodiche.
Partiamo dalle funzioni con un solo periodo, che a meno di rinormalizzazioni posso considerare essere $1$.
Lo spazio topologico quoziente $\bbC / \bbZ$ (ovvero $\bbC$ quozientato per la relazione di
equivalenza $a\textit{R}b \sse a-b \in \bbZ$) è un cilindro, e non è compatto.

\begin{definizione}
Si dice una superficie di Riemann una varietà complessa connessa di dimensione $1$.
Si intende che ogni punto deve avere un intorno $U_\alpha$ omeomorfo al disco unitario aperto di $\bbC$
tramite l'omeomorfismo $\phi_\alpha$, e che per ogni $\alpha$ e $\beta$ valga che $g:=\phi_\beta \circ {\phi_\alpha}^{-1}$
sia una funzione olomorfa.
\end{definizione}

\notamargine{La regolarità nei complessi è molto più forte che non nei reali. Essere olomorfe è davvero tanta roba in più che non essere $C^{\infty}$.}

\begin{osservazione}
Più periodi richiedo, più è difficile che la funzione sia anche solo continua. Ad esempio una funzione
$f: \bbR \rar \bbR$ con due periodi incommensurabili è necessariamente costante.
\end{osservazione}

\begin{lemma}
Le funzioni meromorfe con periodo $z_0$ sono tutte e sole quelle della forma $g\left(e^{2\pi i/z_0}\right)$
\end{lemma}

\begin{osservazione}
$\bbC / \bbZ$ è una superficie di Riemann, ma non è omeomorfa a $\bbC$, ad esempio perché $\bbC$ non è semplicemente connesso.
\end{osservazione}

\begin{lemma}
Sia $f: \bbC \rar \bbC$ una funzione meromorfa. Allora l'insieme $L$ dei periodi di $f$
forma un sottogruppo additivo di $\bbC$.
\end{lemma}
\begin{proof}
Esercizio (facile).
\end{proof}

\section{Reticoli}
Sia $f: \bbC \rar \bbC$ una funzione meromorfa non costante.
Sia $L_f$ il gruppo dei periodi di $f$. Allora:
\begin{enumerate}
 \item $L_f$ è discreto
 \item $L_f$ può essere isomorfo solo al gruppo banale, a $\bbZ$ o a $\bbZ^2$ 
\end{enumerate}

\begin{definizione}[Reticolo]
$L_f$ si chiama reticolo di $f$.
\end{definizione}

\begin{definizione}[Funzione ellittica]
Una funzione $f$ si dice ellittica se il suo reticolo ha rango $2$.
\end{definizione}

\notamargine{Tutti i reticoli di rango $2$ sono isomorfi come gruppi, ma la loro
struttura complessa vedremo che sarà completamente diversa.}

\begin{osservazione}
Se $L_f$ ha rango 2, $\bbC / L_f$ è omeomorfo ad un toro, e quindi è compatto.
\end{osservazione}

\paragraph{Finestra sul futuro}
Le funzioni ellittiche "provengono" da equazioni (???)


\begin{definizione}
Dato un polinomio in due (o più) variabili $p\left(x,y\right)$, si dice polinomio omogenizzato il
polinomio $z^{deg\left(p\right)}\cdot p\left(x/z,y/z\right)$
\end{definizione}

\begin{osservazione}
Ho ottenuto un polinomio omogeneo in tre variabili, con soluzioni in $\bbP^2 \left( \bbC \right)$, che è compatto.
\end{osservazione}

\section{Curve ellittiche}
\begin{definizione}[Curva ellittica]
Si dice curva ellittica un sottoinsieme di $\bbC^2$: $E= \left\{\left(x,y \right) \in \bbC^2 | y^2=p \left( x \right) \right\}$, dove $p$ è un polinomio a coefficienti complessi di terzo grado con radici distinte.
\end{definizione}

Le soluzioni di questa equazione coincidono con 
gli zeri della funzione $f \left( x,y \right) = y^2 - p \left( x \right)$.
Per il Teorema del Dini, intorno ad ognuno di questi zeri la curva si riesce ad esprimere come un grafico in almeno una delle due variabili.
(le ipotesi del Teorema sono soddisfatte grazie all'assenza di radici multiple di $p$ ).

\notamargine{ Si chiamano curve ellittiche, perché sono collegate con la lunghezza di archi di ellisse.
Data un'ellisse $y^2 = 1- \alpha x^2 $, per calcolare la lunghezza di un arco si giunge a:
$\int \frac{1-b^2 x}{\sqrt{\left( 1-b^2 x\right)\left(1-a^2 x\right)}}$ }, che con un'opportuna sostituzione...

\section{Le parametriche}
Buco

\section{Trasformazione razionale della curva in sé}
Fissiamo un punto $P$ appartenente alla curva algebrica, sarà la nostra origine. Fissato un qualsiasi altro punto $Q$
appartenente alla curva, consideriamo la retta che passa per $P$ e $Q$. Intersecherà la curva in esattamente un altro punto $R$.
Abbiamo quindi associato al punto $Q$ il punto $R$. Tale trasformazione è chiaramente iniettiva, e si può dimostrare (esercizio)
che è anche razionale (cioè le coordinate di $R$ sono una funzione razionale delle coordinate di $Q$).

\section{Grupi algebrici}
\begin{definizione}[Gruppo algebrico]
Si dice gruppo algebrico un luogo definito da equazioni algebriche 
su un qualche luogo, dotato di una struttura di gruppo razionale.
\end{definizione}

\begin{definizione}
$\bbG_a$ è la retta affine (???).
\end{definizione}

\begin{osservazione}
Sono gruppi algebrici non compatti di dimensione $1$.
\end{osservazione}

\begin{definizione}
$\bbG_m$ è $\bbG_a$ meno un punto, e l'operazione è il prodotto (??? componente per componente?
In $\bbR$ o in $\bbC$??).
\end{definizione}


\paraghaph{Finestra sul futuro}:
Le curve ellittiche saranno tutti e soli i gruppi algebrici di dimensione $2$. Saranno compatti
(moralmente, provengono da dei tori, che sono compatti).


