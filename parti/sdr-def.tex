\chapter{Definizione e primi esempi}

\section{Preliminari ed esempi}
In questa prima parte del corso ci occuperemo di superfici di Riemann e inizieremo una classificazione, che ci sarà utile soprattutto per i tori, ma anche per l'ultima parte sulle funzioni modulari.\\
Cominciamo quindi con la
\begin{definizione}[Superficie di Riemann]
Si dice superficie di Riemann una varietà complessa connessa di dimensione $1$, cioè uno spazio topologico di Hausdorff (T2)
connesso tale che ogni suo punto abbia un intorno $U_\alpha$ omeomorfo a un aperto $V_\alpha$ di $\mathbb{C}$
tramite l'omeomorfismo $\varphi_\alpha$, e che per ogni $\alpha$ e $\beta$ valga che $\varphi_\beta \circ {\varphi_\alpha}^{-1}$
sia una funzione olomorfa in $\varphi_\alpha \left( U_\alpha \cap U_\beta \right)$.
\end{definizione}

\begin{osservazione}
La richiesta di essere T2 non si deduce dalle altre condizioni. Infatti se consideriamo il "disco con due origini",
cioè il quoziente $S/\!\!\sim$ dove $S=D\times\{0\} \cup D\times\{1\}$ e $x\sim y \Longleftrightarrow x=y$ oppure $x=(u,i),\ y=(u,j)$ con $\{i,j\}=\{0,1\}$,
si osserva che ogni punto ha un intorno omeomorfo a un aperto di $\mathbb{C}$ ma $(0,0)$ e $(0,1)$ non hanno intorni disgiunti.
\end{osservazione}
\begin{esercizio}
$S/\!\!\sim$ è semplicemente connessa.
\end{esercizio}

In questa lezione discuteremo di alcuni esempi di superfici di Riemann, e cominceremo ad introdurre alcuni risultati che ci aiuteranno a classificare specifiche classi di superfici di Riemann a meno di isomorfismo.

\textbf{Primi esempi di superfici di Riemann:}
\begin{enumerate}
  \item $\bbC$ (o un qualsiasi aperto connesso di $\bbC$). In questo caso l'unica carta è tutto l'insieme, con la mappa di immersione (non compatta).
  \item La sfera di Riemann $\bbP^1(\bbC)$, anche detta $\hat\bbC$. $\bbP^1(\bbC)=U_0\cup U_1$ dove $U_0=\{(x_0:x_1)\in \bbP^1(\bbC) | \ x_0\neq 0\}$ e $U_1=\{(x_0:x_1)\in \bbP^1(\bbC) | \ x_1\neq 0\}$, con le mappe $\varphi_0(x_0:x_1)=\frac{x_1}{x_0}$ e $\varphi_1(x_0:x_1)=\frac{x_0}{x_1}$. La funzione di transizione è $z\mapsto 1/z$ che è olomorfa nell'intersezione, ovvero $\bbC^\ast$. Questa superficie di Riemann è compatta.
  \item Curve algebriche non singolari in $\bbP^2(\bbC)$ ($\{(x:y:z)\in \bbP^2(\bbC) | \ f(x,y,z)=0\}$, dove $f\in \bbC[x,y,z]$ è un polinomio irriducibile (è compatta).\\
  Infatti data la non singolarità, in ogni punto vale $\nabla f(p)\neq0$, quindi per il teorema della funzione implicita (versione complessa) posso esprimere una coordinata come funzione olomorfa delle altre e uso questa come carta.
\end{enumerate}
\notamargine{ACHTUNG! Manca la dimostrazione che le curve algebriche non singolari sono superfici di Riemann.}
\begin{osservazione}
    In $\bbP^2(\bbC)$ vale che ogni curva definita da un polinomio riducibile (che non sia una potenza di un irriducibile) è singolare. Infatti, detto $f(x,y,z)=p(x,y,z)\cdot q(x,y,z)$, se $p$ e $q$ non sono una potenza di uno stesso polinomio irriducibile, per Bézout deve esistere un punto isolato in $\{p(x,y,z)=0\}\cap\{q(x,y,z)=0\}\subseteq \bbP^2(\bbC)$. In questo punto la curva è formata da due "bracci" che si intersecano, e quindi non è localmente esprimibile come grafico, pertanto necessariamente entrambe le derivate parziali si annullano, dunque è singolare. %TODO qui ci starebbe molto bene un disegnino (magari nelle note).
\end{osservazione}

Come curiosità diciamo anche che:
\begin{divagazione}[di Chow]
    Gli esempi precedenti costituiscono tutti gli esempi di superfici di Riemann compatte immerse in $\bbP^2(\bbC)$.
\end{divagazione}
\begin{osservazione}
    Una superficie di Riemann meno un numero finito di punti resta una superficie di Riemann. Questo è dovuto al fatto che un aperto connesso di $\bbC$ meno un punto resta un aperto connesso.
\end{osservazione}
%TODO Curve algebriche singolari meno i punti singolari


\textbf{Esempi di superfici di Riemann come quozienti di $\bbC$:}
\begin{enumerate} %TODO Far partire l'indice da 4
  \item $\bbC/\bbZ$ (non compatta).
  \item $\bbC/L$, dove $L$ è un reticolo (discreto) di rango 2 (compatta).
\end{enumerate}
\begin{proof} \textit{(che sono superfici di Riemann)}
    Dimostriamo solo che $\bbC/\bbZ$ lo è, la dimostrazione per $\bbC/L$ è analoga. Considero $\pi:\bbC\rightarrow\bbC/\bbZ$ la proiezione al quoziente, e $Y:=\{z\in\bbC|\ 0\leq Re(z)<1\}$ una striscia verticale di rappresentanti. Ricopro $Y\subseteq\bbC$ con dischi $D_\alpha$ (aperti) di raggio $1$ (in modo che $D_\alpha$ non contenga mai due punti che al quoziente sono uguali). Si osserva facilmente che $\pi_{|D_\alpha}$ è un omeomorfismo, scegliamo $\varphi_\alpha=\pi_{|D_\alpha}^{-1}$, si verifica facilmente soddisfare le proprietà richieste dalla definizione.

    Nel caso del reticolo di rango $2$, la dimostrazione si fa prendendo come $Y$ un parallelogrammo, e raggio dei dischi abbastanza piccolo da impedire che ci possano essere due punti equivalenti nello stesso disco.
\end{proof}

\begin{definizione}
Siano $X$ e $Y$ due superfici di Riemann, $x_0\in X$, $y_0=f(x_0)$. $f:X\rightarrow Y$ si dice olomorfa in $x_0$ se esiste un intorno $A$ di $x_0$ tale che $A\subseteq U_\alpha$ e detto $V_\beta$ un aperto del ricoprimento di $Y$ che contiene $y_0$, vale che $\psi_\beta \circ f \circ \varphi_\alpha^{-1}: \varphi_\alpha(A)\rightarrow \bbC$ è una funzione olomorfa.
\end{definizione}
\begin{osservazione}
    Quella sopra è una buona definizione, ovvero non dipende da $u_\alpha$ e $V_\beta$, grazie alla proprietà di compatibilità sulle intersezioni delle $\varphi_\alpha$ e $\psi_\beta$.
\end{osservazione}


\begin{definizione}
Due superfici di Riemann $X$ e $Y$ si dicono isomorfe (o conformemente equivalenti) se esiste $f:X\rightarrow Y$ invertibile, olomorfa con inversa olomorfa. Una tale $f$ viene definita biolomorfismo.
\end{definizione}

\begin{osservazione}
Sia $X$ una superficie di Riemann, $Y$ uno spazio topologico, $f:X\rightarrow Y$ un omeomorfismo. Allora posso trasportare su $Y$ la struttura complessa di $X$, ricoprendolo con aperti $V_\alpha=f(U_\alpha)$ e mappe $\psi_\alpha=\varphi_\alpha \circ f^{-1}_{|f(U_\alpha)}:f(U_\alpha)\rightarrow\bbC$
\notamargine{Con questa struttura di varietà, chiaramente $Y$ è isomorfo ad $X$}
\end{osservazione}

Finiamo la sezione con un importante teorema di classificazione, grazie al quale riusciremo ad avere una parziale classificazione di tutte le superfici di Riemann usando la teoria dei rivestimenti.
\begin{teorema}[di Riemann]\label{teo_riemann}
Ogni superficie di Riemann semplicemente connessa è biolomorfa ad uno dei seguenti tre modelli:
\begin{enumerate}
  \item La sfera di Riemann $\bbP^1(\bbC)=:\widehat{\bbC}$.
  \item Il piano complesso $\bbC$.
  \item Il disco di Poincaré $D$.
\end{enumerate}
\end{teorema}
\begin{proof}
La dimostrazione non verrà trattata in questo corso a causa dell'eccessiva difficoltà.
Nella prossima lezione vedremo che queste tre superfici non sono biolomorfe (è una conseguenza del teorema di Liouville).
\end{proof}



\chapter{Classificazione delle superfici di Riemann, parte I}
Cerchiamo di muoverci verso un risultato riguardo la classificazione delle superfici di Riemann. Lo schema con cui affronteremo il problema consiste nel considerare un rivestimento universale della superficie di Riemann e cercare di esprimere la superficie di partenza come quoziente dello spazio rivestente per un gruppo di automorfismi. In questo modo sposteremo il problema sullo studio dei sottogruppi del gruppo di automorfismi delle tre superfici semplicemente connesse date dal teorema di Riemann.

\section{Rivestimenti olomorfi}


Cominciamo con l'osservare che dato un rivestimento in cui lo spazio base è una \sdR, questo induce una struttura complessa anche sullo spazio ambiente.
\begin{proposizione}\label{sdr_strut_riv}
Sia $X$ una superficie di Riemann, se $\pi:Y\rightarrow X$ è un rivestimento, allora $Y$ è in modo naturale una superficie di Riemann, con una definizione che rende $\pi$ olomorfa.
\end{proposizione}
\begin{proof}[Idea della dimostrazione]
È possibile fare un raffinamento degli $U_\alpha\subseteq X$ in modo da renderli "compatibili" con gli aperti banalizzanti del rivestimento (per esempio, posso considerare le intersezioni con essi). In questo modo, avendo gli $U_\alpha$ inclusi in un aperto banalizzante $A$, è possibile "tirarli su" sullo spazio ricoprente: se $\pi^{-1}(A)=\bigcup A_i$ con $\pi\restriction_{A_i}:A_i\to A$ omeomorfismi, metto nell'atlante per $Y$ tutte le carte $(A_i,\phi_\alpha\circ\pi)$. La verifica che sono rispettate le proprietà della definizione è banale. Per vedere che $\pi$ è olomorfa basta osservare che prendendo le carte giuste si ottiene l'identità su $\bbC$.
\end{proof}

Tutto il nostro interesse per i rivestimenti (quelli universali in particolari) è dovuto alla seguente
\begin{osservazione}
Sia $\pi:\widetilde{X}\rightarrow X$ un rivestimento universale, $x_0\in X$. Allora la fibra $\pi^{-1}(x_0)$ è discreta in $\widetilde{X}$, ed esiste un gruppo di omeomorfismi di $\widetilde{X}$ che preserva le fibre ed agisce in modo transitivo su di esse. Chiamato $G$ tale gruppo, si ha che $X \simeq \widetilde{X}/G$.
\end{osservazione}

Quindi il nostro studio si sposta ora sui sottogruppi di $G=\Aut(\widetilde X,\pi)$.\\
Ricordiamo che $g\in G$ vuol dire $g:\widetilde X\to\widetilde X$ tale che $\pi\circ g=\pi$, ovvero $g$ preserva le fibre.
\begin{proposizione}
 	Sia $g\in G$. Allora $g$ è olomorfa come applicazione tra \sdR.
\end{proposizione}
\begin{proof}
 	Sia $x\in \ot X$. Sia $V_\alpha^r$ uno degli aperti della condizione di \sdR\ tale che $x\in V_\alpha^r$, e sia $\phi_\alpha^r$ la sua carta locale. Si ha $g(x)\in V_\alpha^s$ per un qualche $s$, e dunque bisogna dimostrare $\phi_\alpha^s\circ g\circ (\phi_\alpha^r)^{-1}$ olomorfa in $\phi_\alpha^r(x)$. Ma questa funzione si può scrivere anche come $\phi_\alpha\circ\pi\circ g\circ \pi^{-1}\circ \phi_\alpha^{-1}$, che poiché $g$ è automorfismo di rivestimento è uguale a $\phi_\alpha\circ\pi\circ\pi^{-1}\circ\phi_\alpha^{-1}=id$. Poiché l'identità è olomorfa, abbiamo dimostrato che $g$ è olomorfa.
\end{proof}
\notamargine{$\pi^{-1}$ dovrebbe presentare un pedice che si riferisce al fatto che ha codominio $V_\alpha^r$, ma non l'ho ritenuto fondamentale}


\section{Azioni di gruppi propriamente discontinui}

Vogliamo capire come sono fatti i sottogruppi di $\Aut(\widetilde X,\pi)$ che inducono un rivestimento sulla nostra superdicie di Riemann di partenza $X$.

\begin{definizione}
Sia $G$ un gruppo che agisce su uno spazio topologico $X$. L'azione di $G$ si dice \textit{propriamente discontinua} se per ogni compatto $K\subseteq X$ esiste solo un numero finito di elementi di $G$ tali che $g(K)\cap K =\varnothing$.
\end{definizione}

Allora vale il seguente
\begin{teorema}
Sia $X$ una varietà con un gruppo di automorfismi $G<Aut(X)$. Allora $\pi:X\to X/G$ è un rivestimento se e solo se valgono le seguenti due proprietà:
\begin{enumerate}
  \item L'azione di $G$ su $X$ è propriamente discontinua.
  \item Gli elementi di $G\setminus\{\Id\}$ agiscono senza punti fissi.
\end{enumerate}
\end{teorema}
\begin{proof}$ $\\
\begin{itemize}
    \item[$\Longleftarrow$]Consideriamo $g\in G$ che ha un punto fisso $g(x)=x$; prendo un aperto banalizzante $A\ni \pi(x)$, ovvero $\pi^{-1}(A)=\bigcup B_i$ con $\pi\restriction_{B_i}$ omeomorfismo. Considero ora il $B$ tale che $x\in B$; dato che $g$ è olomorfa, esiste un intorno $U\ni x$ tale che $g(U)\subset B$; inoltre vale $\pi\circ g=\pi$, per cui $\forall u\in U$ si ha $\pi(g(u))=\pi(u)$; ma dato che $g(u)\in B$ e che $\pi\restriction_B$ è bigettivo, abbiamo infine $g(u)=u$. Dunque i punti fissi di $g$ sono un aperto; ma sono anche un chiuso in quanto $(g-\Id)^{-1}(0)$, quindi abbiamo $g=\Id$ in quanto $X$ è connesso.\\
    Supponiamo ora per assurdo che esista un $K$ compatto per cui $H=\{ g\in G : g(K)\cap K\neq\emptyset \}$ è infinito; anche $H'=H\setminus\{\Id\}$ è infinito. Allora $\forall x$ possiamo trovare una successione $g^x_n\in H'$ per cui $g^x_n(x)$ converga a qualche punto $\tilde x$; se considero ora un aperto banalizzante $A\ni\pi(\tilde x)$ e il $B$ per cui $\pi\restriction_B$ sia un omeomorfismo, per $n\ge N$ ho anche $g^x_n(x)\in B$; ma $\pi(g^x_n(x))=\pi(x)$ e dato che $\pi$ è bigettiva su $B$, deve essere $g^x_n(x)=x\forall n\ge N$. Ma dato che $G$ agisce senza punti fissi, ho $g^x_n=\Id$, che è assurdo poichè $\Id\not\in H'$.\\
    \item[$\implies$] Sia $x\in X$, e $K$ un suo intorno compatto; l'insieme $H=\{ g : g(K)\cap K\neq\emptyset\}$ è finito e ha elementi $h_0=\Id,h_1,\dots,h_n$. Vediamo che per ogni $i\neq j$ vale $h_i(x)\neq h_j(x)$ perché altrimenti $h_ih_j^{-1}$ avrebbe un punto fisso; ma allora possiamo trovare intorni $U_i\ni h_i(x)$ disgiunti. Come aperto banalizzante intorno a $\pi(x)$ prendo allora $V=\bigcap \pi(U_i)$; infatti $\pi^{-1}(V)=\bigcup{g(U'_0) : g\in G}$ dove $U'_0$ è una restrizione di $U_0$, e tutti questi aperti sono disgiunti per costruzione.
\end{itemize}
\end{proof}

Come avevamo accennato in precedenza, vale poi anche:

\begin{teorema} \label{rivuniv}
 	Sia $p:E\rar X$ un rivestimento universale. Allora esiste $G$ gruppo di omeomorfismi di $E$ propriamente discontinuo tale che $X\isom\quotient{E}{G}$.
\end{teorema}
\notamargine{Ricordo che $p:E\rar X$ rivestimento si dice universale se $E$ è semplicemente connesso}
\notamargine{Anche qui la dimostrazione sta sul Manetti, in posti a caso}


Adesso, sia $X$ una \sdR, e sia $\pi:\ot X\rar X$ il suo rivestimento universale (che ricordo essere unico a meno di omeomorfismi). Attraverso $\pi$, $\ot X$ eredita una struttura di \sdR\ da quella di $X$, come visto in \ref{sdr_strut_riv}.\\
Ora, per il teorema \ref{rivuniv}, esiste $G$ fatto nel modo giusto tale che $\quotient{\ot X}{G}\isom X$.


Ora, per classificare le \sdR\ posso dunque studiare tutti i quozienti opportuni di \sdR\ semplicemente connesse. Per il Teorema di Riemann, queste sono solo $\bbP_1(\bbC)=\oc \bbC, \bbC, \cD\isom\cH$, dove su quest'ultimo ho la struttura complessa data dall'inclusione.

Dunque il prossimo passo è classificare i gruppi di isomorfismi propriamente discontinui e olomorfi delle tre \sdR\ semplicemente connesse.

\section{Cusumano}


\newthought{Ora iniziano} cose di cui non ho esattamente capito il senso, per ora.

\begin{definizione}
	Sia $A\subset\bbC$. Una funzione $f:A\rar\bbC$ si dice \emph{algebrica} se $\exists\ p\in\bbC[x,y], p\neq0 \tc p(x,f(x))=0\ \forall\ x\in A$.
\end{definizione}

\begin{esercizio}
 	$f(x)=e^x$ non è una funzione algebrica.
\end{esercizio}

Ora si sta parlando di rivestimenti indotti da funzioni algebriche.

Per esempio, consideriamo la funzione $p(t)=t^2, p:\bbC\rar\bbC$. Questa funzione non è bigettiva, dunque non ha un'inversa globale, ma comunque localmente ha un'inversa. Devo però effettuare una scelta per ottenere questa inversa locale, in paricolare la scelta del segno. Inoltre vorrei che quest'inversa locale sia quantomeno continua. Si dimostra però che in ogni intorno di $0$ non esiste nessuna funzione continua tale che $f(x)^2=x$ per ogni $x$.

Questo si può vedere tramite i rivestimenti. Infatti la funzione $p(t)$ non è un rivestimento di $\bbC$ su sè stesso, proprio perché l'immagine inversa di ogni intorno di $0$ non è mai omeomorfa all'intorno tramite la mappa $p$.

Se però considero $p:\bbC^*\rar\bbC^*$, ho che questo è un rivestimento di grado $2$, e pertanto, poiché $\bbC^*$ è connesso, non può avere sezioni globali, ma solamente sezioni locali.
\notamargine{Se $p:E\rar X$ è rivestimento, una sua sezione $\phi$ è una mappa continua da $U\subset X$ a $E$ t.c. $p\circ \phi=id_U$}

È facile osservare che $z\rar-z$ è un automorfismo di rivestimento olomorfo per $p$. È però possibile avere rivestimenti senza automorfismi olomorfi. Un esempio è dato di seguito.
\notamargine{lo stesso argomento è (abbastanza alla lontana) trattato dal Manetti, capitolo 13, sezione 5}

Sia dunque $\phi:\bbC\rar\bbC$ definita da $\phi(z)=z^3-3z$. Voglio capire se induce un rivestimento su $\bbC$ o perlomeno su un suo sottoinsieme.
Essendo un polinomio di grado $3$, l'eventuale rivestimento indotto avrà grado $3$, dunque devono essere scartati tutti i punti $\ol z$ tali che, se $\phi(z)=a$ ha soluzione doppia, $\ol z$ è una delle soluzioni di questa equazione, anche non doppia.
\notamargine{Questo significa che devono essere scartati tutti i punti che stanno in una fibra di cardinalità minore di $3$, perché in un rivestimento la cardinalità delle fibre è costante}

Questi punti si possono identificare tramite gli zeri della derivata, che è $3z^2-3$. In particolare, gli zeri della derivata sono $\pm1$, e gli $a$ corrispondenti sono $\mp2$. Gli $z$ soluzione di $z^3-3z\pm2=0$ sono $\pm1,\pm2$, dunque il rivestimento indotto da questa $\phi$ sarà $\phi:\bbC\minus\{\pm1,\pm2\}\rar\bbC\minus\{\pm2\}$.

%TODO: capire per bene quella cosa di Galois, intanto metto due parole

Sembra che questa cosa sia collegata in qualche modo al fatto che, se considero il campo delle funzioni razionali $\bbC(x)$, e $y$ è radice di $y^3-3y-x$, allora l'estensione $\quotient{\bbC(x,y)}{\bbC(x)}$ non è di Galois.

Ora supponiamo per assurdo che il rivestimento di $\phi$ abbia un automorfismo olomorfo non banale $f$. Allora posso estenderlo per continuità a tutto $\bbC$, perché è limitato jin quanto $f$ soddisfa $f(z)^3-3f(z)=z^3-3z$, e questo porta a $f(z)$ limitato se $z$ limitato.

Inoltre, per $\abs{z}$ abbastanza grande, si può scrivere $\dst 1=\frac{f(z)^3-3f(z)}{z^3-3z}$, e per $\abs{z}\rar+\infty$ devo avere $\abs{f(z)/z}\rar1$, per cui, essendo $f$ olomorfa, si ha $f(z)=az+b$. Chiedendo che $f(z)$ sia automorfismo di rivestimenti, si scopre che l'unica possibilità è $f(z)\equiv z$.
