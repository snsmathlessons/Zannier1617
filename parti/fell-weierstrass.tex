\chapter{Funzioni ellittiche}

Prendiamo $T_1=\bbC/L_1$ e $T_2=\bbC/L_2$ tori complessi.
Ogni morfismo (omomorfismo di gruppi che sia anche olomorfo) $T_1\rar T_2$ si solleva ad un unico morfismo $\bbC\rar\bbC$, che è dato da una funzione affine $z \mapsto az + b$.
Se rappresentiamo $T_1$ e $T_2$ come quozienti di $\bbC$ in un modo diverso, il sollevamento può cambiare (quindi non è canonico).
Tuttavia, se $T_1=T_2$ allora la sollevata si può prendere della forma $z \mapsto az$ ed $a$ è determinato in modo canonico (quello ottenuto rappresentando $T_1$ e $T_2$ nello stesso modo come quozienti di $\bbC)$.

\section{Formule di annullamento}

\begin{teorema}
  \label{residui-funzioni-toriche}
$T$ un toro e $f:T\rar\bbC$ meromorfa (non costante). Allora
\begin{enumerate}
    \item $\sum_{x\in T}\res_xf=0$ $\qquad$ (uguaglianza di numeri complessi)

    \item $\sum_{x\in T}\ord_xf=0$ $\qquad$ (uguaglianza di numeri interi)

    \item $\sum_{x\in T}x\cdot\ord_xf=0$ $\qquad$ (uguaglianza di elementi del gruppo $T$)
\end{enumerate}

\end{teorema}

\notamargine{$\res_xf$ indica il residuo di $f$ in $x$ (e $0$ se $f(x)\in\bbC$)}

\notamargine{$\ord_xf$ indica la molteplicità di $x$ come radice di $f(x)=0$, o l'opposto dell'ordine del polo in $x$ (e $0$ se $f(x)\in\bbC\setminus\{0\}$)}

\begin{proof}
$f^{-1}(0)$ è finita: se fosse infinita, allora dovrebbe accumularsi da qualche parte su $T$ (per compattezza), e quindi $f$ sarebbe costante (per prolungamento analitico).
Analogamente l'insieme dei punti in cui $f$ non è definita è finito.

Scegliamo una rappresentazione $T=\bbC/L$ con $L=\{a\omega_1+b\omega_2 : a,b\in\bbZ\}$. Prendiamo la funzione biperiodica $F:\bbC\rar\bbP^1$ data da $\bbC\rar T\xrightarrow{f}\bbP_1$.

Sia $\gamma:[0,1]\rar\bbC$ un cammino che percorre il bordo di un parallelogrammo del reticolo (un solo giro, verso antiorario):
possiamo supporre che in $\Img\gamma$ valga $F\in\bbC\setminus\{0\}$ (a meno di traslare $\gamma$).

(i) Uso il teorema dei residui sulla funzione $F$ con il gammino $\gamma$: ottengo

$2\pi i\sum_{x\in T}\res_xf=\int_\gamma F=(\int_0^{\omega_1}F-\int_{\omega_2}^{\omega_1+\omega_2}F)+(\int_{\omega_1}^{\omega_1+\omega_2}F-\int_{0}^{\omega_2}F)$

ed entrambe le parentesi si annullano per biperiodicità di $F$.

\notamargine{Un po' di libertà di notazione}

(ii) Identico ad (i), usando la funzione $\frac{F'}{F}$ al posto di $F$ (osservo che $\frac{F'}{F}$ è ancora biperiodica) (osservo che $\res_x\frac{F'}{F}=\ord_xF$).

(iii) Uso il teorema dei residui sulla funzione $z\frac{F'(z)}{F(z)}$ con il cammino $\gamma$.

Notare che $\res_xz\frac{F'(z)}{F(z)}=x\cdot\res_x\frac{F'}{F}=x\cdot\ord_xF$.
\notamargine{In generale, data $g(z) = \sum_{k \in \bbZ} c_k (z - a)^k$ vale che $res_a (g(z)z) = c_{-2} + a c_{-1} = c_{-2} + a res_a(g(z))$
Se scegliamo $g(z) = \frac{f'(z)}{f(z)}$ effettivamente $c_{-2} = 0$}

Ottengo $2\pi i\sum_{x\in T}x\cdot\ord_xf=\int_\gamma z\frac{F'(z)}{F(z)}\ dz=(\int_0^{\omega_1}-\int_{\omega_2}^{\omega_1+\omega_2})+(\int_{\omega_1}^{\omega_1+\omega_2}-\int_{0}^{\omega_2})$

e per biperiodicità si ha $\int_{\omega_2}^{\omega_1+\omega_2}z\frac{F'z}{Fz}\ dz=\int_0^{\omega_1}(z-\omega_2)\frac{F'z}{Fz}\ dz$ quindi

$(\int_0^{\omega_1}-\int_{\omega_2}^{\omega_1+\omega_2})=\omega_2\int_0^{\omega_1}\frac{d}{dz}(\log F)\ dz=\omega_22\pi i m$ per qualche $m\in\bbZ$
(più precisamente, $m$ è il numero di giri che fa $\log F$ intorno a $0$ quando $z$ varia da $0$ a $\omega_1$).

Quindi $\sum_{x\in T}x\cdot\ord_xf$ è un elemento del reticolo $L$, da cui la tesi.
\end{proof}


\section{Funzione $\wp$ di Weierstrass}


Fissiamo un toro $T$ e rappresentiamolo come $\bbC/L$ ($L$ reticolo).
Consideriamo la serie di funzioni definite sull'aperto $\bbC\setminus L$ a valori in $\bbC$ data da $\frac{1}{z^2}+\sum_{\omega\in L\ \omega\not=0}(\frac{1}{(z-\omega)^2}-\frac{1}{\omega^2})$.

\begin{proposizione}
La serie sopra converge totalmente (e quindi uniformemente) sui compatti contenuti in $\bbC\setminus L$.
\end{proposizione}
\begin{proof}(sketch)

Fissiamo $K\subseteq\bbC\setminus L$ compatto.

Si ha $\sup_{z\in K}\abs{\frac{1}{(z-\omega)^2}-\frac{1}{\omega^2}}=\sup_{z\in K}\abs{\frac{z(2\omega-z)}{\omega^2(z-\omega)^2}}$ che per $\omega$ 'lontano' da $K$ è circa $\abs{\frac{1}{\omega^3}}$.

Ma $\sum_{\omega\in L\setminus\{0\}}\abs{\frac{1}{\omega^3}}$ converge, perchè
gli $\omega$ di modulo $R$ sono circa $R$, e quindi quella è circa $\sum R\cdot\frac{1}{R^3}=\sum\frac{1}{R^2}<+\infty$.
\end{proof}

Quindi il limite della serie sopra è una funzione olomorfa $P:\bbC\setminus L\rar\bbC$.

\'E inoltre facile verificare che $P$ è una funzione $L$-biperiodica.
Ad esempio, basta considerare $P(z+\tau)-P(z)$ con $\tau \in L$: per le proprietà di convergenza, posso raccogliere la sommatoria e ottenere
\[ P(z+\tau) - P(z) = \frac1{(z+\tau)^2} - \frac1{z^2} + \sum_{\omega \in L\ \omega\ne 0} \frac1{(z+\tau-\omega)^2}-\frac1{(z-\omega)^2}\]
e ora mettendo il caso $\omega=0$ nella sommatoria e riarrangiando si conclude.
\notamargine{Notiamo che gli addendi non vanno spezzati, vanno solamente sommati con la loro ``controparte simmetrizzata'', ovvero ciascun addendo con $\omega$ lo si somma con l'addendo $\tau-\omega$}

Se prendiamo la serie che definisce $P$ e togliamo il termine $\frac{1}{z^2}$, converge uniformemente in un intorno di $0$ (e analogo per ogni $\omega\in L$).
Quindi $P$ è meromorfa su tutto $\bbC$.

Fattorizzando $P$, otteniamo una funzione meromorfa $\wp:T\rar\bbC$, chiamata \textbf{funzione di Weierstrass}. Attenzione: tale definizione dipende dal reticolo $L$ scelto per rappresentare $T$! Però non cambia in modo sostanziale: infatti la $\wp$ dovrebbe trasformare, per un cambio di reticolo $L \mapsto \alpha L$:
\[
 \wp_L(z) = \alpha^2 \wp_{\alpha L}(\alpha z)
\]

In particolare, non vengono modificati né gli zeri né i poli (che saranno le cose che ci interesseranno maggiormente).


\begin{proposizione}
La funzione meromorfa $\wp:T\rar\bbC$ ha le seguenti proprietà:

(i) $\wp(-z)=\wp(z)$.

(ii) $\wp$ ha un polo doppio in $0$ e nessun altro polo.

(iii) Fissato $u\in T$ con $u\not=-u$, si ha $\ord_u(\wp-\wp(u))=\ord_{-u}(\wp-\wp(u))=1$ e $\ord_0(\wp-\wp(u))=-2$ e $\ord_x(\wp-\wp(u))=0$ per $x\not=0,u,-u$.

In altre parole, $\wp-\wp(u)$ ha uno zero semplice in $u$, uno zero semplice in $-u$, un polo doppio in $0$ e nessun altro zero o polo.

(iii') Fissato $u\in T$ con $u=-u\not=0$, si ha $\ord_u(\wp-\wp(u))=2$ e $\ord_0(\wp-\wp(u))=-2$ e $\ord_x(\wp-\wp(u))=0$ per $x\not=0,u$.

In altre parole, $\wp-\wp(u)$ ha uno zero doppio in $u$, un polo doppio in $0$ e nessun altro zero o polo.
\end{proposizione}
\begin{proof}
(i) Ovvio direttamente dalla definizione (riordinando gli addendi).

(ii) Ovvio (perchè sappiamo dove stanno i poli di $P$).

(iii) Prendiamo la funzione meromorfa $\wp-\wp(u):T\rar\bbC$ e usiamo il teorema \ref{residui-funzioni-toriche} (grazie a (ii) sappiamo dove questa ha poli e con che molteplicità).

(iii') Idem come (iii).
\end{proof}


\begin{teorema}
Non esiste un polinomio $A\in\bbC[x]$ tale che $A(\wp):T\rar\bbC$ sia la funzione costantemente nulla.
\end{teorema}
\begin{proof}
Sia $A(x)=(x-a_1)...(x-a_n)$. Se $A(\wp)\equiv0$, allora almeno una tra le $(\wp-a_i)$ dovrebbe essere $\equiv0$, ma $\wp-a_i$ ha un polo doppio in $0$...
\end{proof}

\section{Struttura del campo delle funzioni ellittiche}


\begin{teorema}
Il campo delle funzioni meromorfe pari $T\rar\bbC$ è $\bbC(\wp)$.
\end{teorema}
\begin{proof}
Prendiamo $f:T\rar\bbC$ meromorfa pari (non costante).

Prendiamo $u\not=-u$ con $\ord_uf\not=0$ (quindi uno zero o un polo). Allora $\ord_{-u}f=\ord_uf$ e possiamo considerare $f\cdot(\wp-\wp(u))^{-\ord_uf}$:
questa è meromorfa pari ed ha gli stessi zeri e poli di $f$ (con le stesse molteplicità) eccetto
in zero ed in $u,-u$ (in cui $\ord_uf\cdot(\wp-\wp(u))^{-\ord_uf}=\ord_{-u}f\cdot(\wp-\wp(u))^{-\ord_uf}=0$).
Sostituiamo $f$ con $f\cdot(\wp-\wp(u))^{-\ord_uf}$.

Prendiamo $u=-u\not=0$ con $\ord_uf\not=0$. Allora $\ord_uf$ è pari (se $u$ è uno zero, osserviamo che $f'$ è dispari e quindi $f'(u)=0$, e analogo per tutte le derivate di ordine dispari;
se $u$ è un polo, facciamo lo stesso ragionamento su $\frac{1}{f}$). Possiamo quindi considerare $f\cdot(\wp-\wp(u))^{-\frac{1}{2}\ord_uf}$:
questa è meromorfa pari ed ha gli stessi zeri e poli di $f$ (con le stesse molteplicità) eccetto in zero ed in $u$ (in cui $\ord_uf\cdot(\wp-\wp(u))^{-\frac{1}{2}\ord_uf}=0$).
Sostituiamo $f$ con $f\cdot(\wp-\wp(u))^{-\frac{1}{2}\ord_uf}$.

Reiterando le operazioni sopra descritte, otteniamo un'uguaglianza del tipo $f\cdot\frac{A(\wp)}{B(\wp)}=g$ con $g$ meromorfa pari senza zeri e poli eccetto eventualmente in $0$.
Ma dal teorema \ref{residui-funzioni-toriche} otteniamo che $g$ non ha uno zero o polo nemmeno in $0$. Ma allora $g$ deve essere costante
(sollevate $g:T\rar\bbC$ ad una $G:\bbC\rar\bbC$: vi viene che $G$ è olomorfa su tutto $\bbC$ e biperiodica, quindi limitata, e quindi costante per Liouville).

Quindi $f\in\bbC(\wp)$, come voluto.
\end{proof}


Fissiamo una funzione meromorfa dispari $h:T\rar\bbC$ (non costante) (ad esempio $\wp'$).
Essendo $h^2$ pari, si ha $h^2\in\bbC(\wp)$.
Ogni funzione meromorfa $g:T\rar\bbC$ si scrive (in modo unico) come $g_1+g_2$ con $g_1$ meromorfa pari e $g_2$ meromorfa dispari.
Ma allora $g=g_1+h(\frac{g_2}{h})$ e $\frac{g_2}{h}$ è pari e quindi sta in $\bbC(\wp)$.

Quindi abbiamo ottenuto che il campo delle funzioni meromorfe su $T$ è $\bbC(\wp,h)$ per una qualsiasi $h$ dispari non costante.


$(\wp')^2=\frac{A(\wp)}{B(\wp)}$ e ci chiediamo chi siano $A$ e $B$.

\begin{itemize}
 \item Contiamo i poli in 0. A sinistra ho un polo sestuplo, mentre a destra l'ordine del polo è dato da $2\cdot (\deg A - \deg B)$. Quindi $\deg A = \deg B + 3$.
 \item Contiamo ora gli zeri. A sinistra ho 6 zeri, a destra il numero di zeri è dato da $2*\deg A$: infatti il numeratore contribuisci con zeri in posti diversi da 0 e con poli in zero; il denominatore contribuisce con zeri in 0 (che però vengono mangiati dai poli del denominatore) e con poli altrove. Quindi $\deg A = 3$
\end{itemize}

In conclusione otteniamo che $(\wp')^2=A(\wp)$, con $A$ di grado 3.

\section{Relazione algebrica tra $\wp$ e $\wp'$}

Abbiamo dimostrato che sussite la seguente relazione differenziale tra la funzione di Weierstrass e la sua derivata:

$$\wp_L'(z)^2 = A_L(\wp_L(z)) $$

Dove $L$ è un reticolo di $\bbC$,  $A_L$ è un polinomio cubico. Questa lezione sarà dedicata a capire le caratteristiche di $A_L$; in particolare
\begin{itemize}
\item Proveremo che $A_L$ non ha radici multiple;
\item Determineremo i coefficienti di $A_L$ in termini del reticolo $L$.
\end{itemize}

Chiameremo $E_L$ il luogo di zeri dell'equazione $y^2=A(x)$ in $\bbC^2$, $\tilde{E}_L$ il suo completamento proiettivo.
\begin{proposizione}
Il polinomio $A_L$ non ha radici multiple.
\end{proposizione}

\begin{proof}[Dimostrazione 1]
Evitiamo di esplicitare la dipendenza dal reticolo in $\wp, E,A \dots$ per non appesantire la notazione. Supponiamo per assurdo che $A(t) = (t-c)^2(at+b)$. Allora, sostituendo $\wp, \wp'$ nell'equazione di $E$ si ottiene
$$\wp'(z)^2 = (\wp(z) - c)^2(a\wp(z)+b) $$
$$ \left ( \frac{\wp'(z)}{\wp(z)-c} \right )^2 = a\wp(z)+b $$

Contiamo il numero di poli con molteplicità a destra e a sinistra. A destra ho un polo doppio in $0$. A sinistra ho un polo doppio ogni volta che $\wp - c$ si annulla: infatti se $\ord_{z}(\wp -c) = m > 0$, vale $\ord_{z}(\wp') = m-1$, e perciò $\ord_{z}(LHS) = 2(m-1-m) = -2$. Ma $\wp-c$ ha un polo doppio in zero, dunque ha due zeri (con molteplicità). In definitiva, LHS ha almeno 4 poli mentre RHS ne ha solo 2, assurdo.
\end{proof}

\begin{proof}[Dimostrazione 2]
Riprendiamo l'equazione
$$\wp'(z)^2 = (\wp(z) - c)^2(a\wp(z)+b) $$
E sia $z_0$ uno zero di $\wp-c$. Contiamo gli ordini a destra e a sinistra, indicando con $m:=\ord_{z_0}(\wp-c) > 0$.
A sinistra ho $\ord_{z_0}( \wp'^2) = 2\ord_{z_0}( \wp') = 2(m-1)$. A destra ho $\ord_{z_0}(RHS) = 2m + \ord_{z_0}(a\wp+b) >= 2m$, perchè $a \wp(z_0)+b = ac+b \neq \infty$. Ma $2m> 2(m-1)$, assurdo.
\end{proof}

\begin{proof}[Dimostrazione 3]
Dimostriamo che se $L=\omega_1\bbZ + \omega_2 \bbZ$, gli zeri di $A$ sono $ \wp(\omega_1/2), \wp(\omega_2/2), \wp((\omega_1+\omega_2)/2)$. Questi sono distinti:
\begin{itemize}
\item Dato $z_0 \in \bbC$ diverso da 0, la funzione $\wp(z) - \wp(z_0)$ ha un polo doppio in 0. Visto che il numero di zeri e poli è uguale, avrà due zeri. Per parità di $\wp$, essi sono $\pm z_0$.
\item I complessi $\omega_1/2, \omega_2/2, (\omega_1+\omega_2)/2$ sono a due a due non opposti.
\end{itemize}
Dunque ci basta dimostrare che questi sono effettivamente zeri di $A$ (che ne avrà tre, essendo di grado 3). La funzione $\wp'$ è dispari e periodica di periodo $L$, perciò per ogni $x$ tale che $2x \in L$ si ha
$\wp'(x) = \wp'(x-2x) = \wp'(-x) = -\wp'(x)$
da cui $\wp'(x) = 0$. I numeri $x=\omega_1/2, \omega_2/2, (\omega_1+\omega_2)/2$ sono effettivamente tali che $2x \in L$, perciò zeri di $\wp'$. Per tali $x$ ottengo
$$ 0 = \wp'(x)^2 = A(\wp(x))$$
che $\wp(x)$ è uno zero di $A$, as desired.
\end{proof}

\begin{proposizione}
Il polinomio $A_L(x)$ è della forma $4x^3-g_2x-g_3$, dove
\begin{itemize}
\item $g_2=60s_4, g_3=140s_6$
\item $\displaystyle s_n = \sum_{0 \neq \omega \in L} \frac{1}{\omega^n} $
\end{itemize}

\end{proposizione}

\begin{proof}
Questo conto è lungo. Bisogna avere pazienza, alle volte anche mesi. Ma lo portiamo a casa: è una promessa. \newline
Sia $L^*$ il reticolo senza lo zero. Richiamiamo l'espressione in serie per la $\wp, \wp'$:
\begin{eqnarray*}
\wp(z) & = & \frac{1}{z^2} + \sum_{\omega \in L^*} \frac{1}{(\omega-z)^2} -\frac{1}{\omega^2} \\
\wp'(z) & = & -\frac{2}{z^3} + \sum_{\omega \in L^*} \frac{2}{(\omega-z)^3} \\
\end{eqnarray*}
E le seguenti identità per $|x| < 1$ \notamargine{la seconda e la terza si ottengono derivando le precedenti}:
\begin{eqnarray*}
\frac{1}{1-x} & = & \sum_{k \ge 0} x^k \\
\frac{1}{(1-x)^2} & = & \sum_{k \ge 0} (k+1)x^k \\
\frac{2}{(1-x)^3} & = & \sum_{k \ge 0} (k+1)(k+2) x^k \\
\end{eqnarray*}
Si noti inoltre che $s_n=0$ per $n$ dispari, perchè se $\omega \in L^*$ allora $-\omega \in L^*$. Vogliamo sviluppare le $\wp, \wp'$ per $|z| < |\omega|, \ \forall \ \omega \in L^*$. Poi calcoleremo i coefficienti espliciti delle potenze negative di $z$ che compaiono nella serie, e troveremo i coefficienti di $A_L$ comparando le due espressioni $(\wp')^2 = A(\wp)$. Vale, per $|z| < |\omega| \ \forall \omega \in L^*$ \notamargine{Tutte gli scambi di sommatoria sono leciti perchè tutte le convergenze sono assolute}:

\begin{eqnarray*}
\wp(z) & = & \frac{1}{z^2} + \sum_{\omega \in L^*} \frac{1}{\omega^2} \frac{1}{(1-z/\omega)^2} - \frac{1}{\omega^2} = \\
       & = & \frac{1}{z^2} + \sum_{\omega \in L^*} \frac{1}{\omega^2} \sum_{k \ge 1} (k+1) \left ( \frac{z}{\omega} \right ) ^k =  \text{attenzione al k=0 !}\\
       & = & \frac{1}{z^2} + \sum_{k \ge 1} (k+1) z^k s_{k+2} = \\
       & = & \frac{1}{z^2} + \sum_{\substack{k \text{ pari} \\ k \ge 2}} (k+1)s_{k+2} z^k \\
\wp'(z)& = & -\frac{2}{z^3}+ \sum_{\omega \in L^*} \frac{1}{\omega^3} \frac{2}{(1-z/\omega)^3}= \\
       & = & -\frac{2}{z^3}+ \sum_{\omega \in L^*} \frac{1}{\omega^3} \sum_{k \ge 0} (k+1)(k+2) \left ( \frac{z}{\omega} \right ) ^k =\\
       & = & -\frac{2}{z^3} + \sum_{k \ge 0} (k+1)(k+2) z^k s_{k+3} =\\
       & = & -\frac{2}{z^3} + \sum_{\substack{k \text{ dispari} \\ k \ge 1}} (k+1)(k+2) z^k s_{k+3}
\end{eqnarray*}
Notare che $\wp, \wp'$ si possono scrivere nella forma $\wp(z) = 1/z^2 + z^2 f(z), \ \ \wp'(z) = -2/z^3 + zg(z)$, per $f,g$ olomorfe. Calcoliamo ora le potenze di $\wp, \wp'$ a meno di un $O(z)$:
\begin{eqnarray*}
\wp'(z)^2 & = & \left ( \frac{-2}{z^3} + zg(z) \right )^2 = \\
          & = & \frac{4}{z^6} -4\frac{g(z)}{z^2} + z^2g^2(z) = \text{ il termine in } z^2 \text{ è } O(z)\\
          & = & \frac{4}{z^6} +\frac{-4 \cdot 6s_4 }{z^2} -4 \cdot 4 \cdot 5 s_6 + O(z) = \\
          & = & \frac{4}{z^6} - \frac{24s_4 }{z^2} -80s_6 + O(z) = \\
\wp(z)^2  & = & \left ( \frac{1}{z^2} + z^2f(z) \right )^2 = \\
          & = & \frac{1}{z^4} +2f(z) + z^4f^2(z) = \text{ il termine in } z^4 \text{ è } O(z)\\
          & = & \frac{1}{z^4} + 6s_4 + O(z) \\
\wp(z)^3  & = & \left ( \frac{1}{z^2} + z^2f(z) \right )^3 = \\
          & = & \frac{1}{z^6} +3\frac{1}{z^4}z^2f(z) + 3 \frac{1}{z^2} z^4 f^2(z) + z^6f(z)^3 = \\
          & = & \frac{1}{z^6} + \frac{9s_4}{z^2} + 15s_6 + O(z) \\
A(\wp(z)) = a\wp(z)^3+b\wp(z)^2 + c\wp(z)+d & = & \frac{a}{z^6} + \frac{b}{z^4} + \frac{9as_4+c}{z^2} + (15as_6+6bs_4+d) + O(z)
\end{eqnarray*}
Imponendo $A(\wp(z))=\wp'(z)^2$, si ottiene:
\begin{itemize}
\item Dal coefficiente di $1/z^6$, $a=4$;
\item Dal coefficiente di $1/z^4$, $b=0$;
\item Dal coefficiente di $1/z^2$, $c = -36s_4-24s_4 = -60s_4 = -g_2$;
\item Dal coefficiente di $z^0$, $d=-80s_6-60s_6 = -140s_6 = -g_3$.
\end{itemize}
\end{proof}
