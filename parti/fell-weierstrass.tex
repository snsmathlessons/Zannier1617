\chapter{Funzioni ellittiche}

Passiamo ora alla parte portante del corso, ovvero le funzioni ellittiche. Cominciamo con la

\begin{definizione}[Funzione ellittica]
    Una funzione ellittica è una $f:\bbC\to\bbC$ meromorfa doppiamente periodica. Possiamo dunque passare al quoziente e pensarla come funzione olomorfa (tra superfici di Riemann) da un toro alla sfera di Riemann $$ f: \quotient{\bbC}{L}\to \bbP_1(\bbC)$$
\end{definizione}

Prendiamo $T_1=\bbC/L_1$ e $T_2=\bbC/L_2$ tori complessi.
Ogni morfismo (omomorfismo di gruppi che sia anche olomorfo) $T_1\rar T_2$ si solleva ad un unico morfismo $\bbC\rar\bbC$, che è dato da una funzione affine $z \mapsto az + b$.
Se rappresentiamo $T_1$ e $T_2$ come quozienti di $\bbC$ in un modo diverso, il sollevamento può cambiare (quindi non è canonico).
Tuttavia, se $T_1=T_2$ allora la sollevata si può prendere della forma $z \mapsto az$ ed $a$ è determinato in modo canonico (quello ottenuto rappresentando $T_1$ e $T_2$ nello stesso modo come quozienti di $\bbC)$.

\section{Formule di annullamento}

\begin{teorema}
  \label{residui-funzioni-toriche}
  Sia $T$ un toro e $f:T\rar\bbP_1$ funzione ellittica. Allora
\begin{enumerate}
    \item $\sum_{x\in T}\res_xf=0$ $\qquad$ (uguaglianza di numeri complessi)

    \item $\sum_{x\in T}\ord_xf=0$ $\qquad$ (uguaglianza di numeri interi)

    \item $\sum_{x\in T}x\cdot\ord_xf=0$ $\qquad$ (uguaglianza di elementi del gruppo $T$)
\end{enumerate}

\end{teorema}

\notamargine{$\res_xf$ indica il residuo di $f$ in $x$ (e $0$ se $f(x)\in\bbC$)}

\notamargine{$\ord_xf$ indica la molteplicità di $x$ come radice di $f(x)=0$, o l'opposto dell'ordine del polo in $x$ (e $0$ se $f(x)\in\bbC\setminus\{0\}$)}

\begin{proof}
$f^{-1}(0)$ è finita: se fosse infinita, allora dovrebbe accumularsi da qualche parte su $T$ (per compattezza), e quindi $f$ sarebbe costante (per prolungamento analitico).
Analogamente l'insieme dei poli, ovvero $f^{-1}(\infty)$, è finito.

Scegliamo una rappresentazione $T=\bbC/L$ con $L=\{a\omega_1+b\omega_2 : a,b\in\bbZ\}$. Prendiamo la funzione biperiodica $F:\bbC\rar\bbP^1$ data da $\bbC\rar T\xrightarrow{f}\bbP_1$.

Sia $\gamma:[0,1]\rar\bbC$ un cammino che percorre il bordo di un parallelogrammo del reticolo (un solo giro, verso antiorario):
possiamo supporre che in $\Img\gamma$ valga $F\in\bbC\setminus\{0\}$ (a meno di traslare $\gamma$).

\begin{enumerate}
    \item Uso il teorema dei residui sulla funzione $F$ con il cammino $\gamma$: ottengo

    $2\pi i\sum_{x\in T}\res_xf=\int_\gamma F=(\int_0^{\omega_1}F-\int_{\omega_2}^{\omega_1+\omega_2}F)+(\int_{\omega_1}^{\omega_1+\omega_2}F-\int_{0}^{\omega_2}F)$

    ed entrambe le parentesi si annullano per biperiodicità di $F$.

    \notamargine{Un po' di libertà di notazione}

    \item Consideriamo la funzione $G=\frac{F'}{F}$ (derivata logaritmica) che è ancora una funzione ellittica: è biperiodica e meromorfa. Osserviamo ora che $\res_{z_0}\frac{F'}{F}=\ord_{z_0}F$; infatti se $F(z)=(z-z_0)^mh(z)$ con $h$ olomorfa e mai nulla, allora $\dfrac{F'}{F}=\dfrac{m}{z-z_0}+\dfrac{h'}{h}$ e il secondo addendo è olomorfo.\\
    Ma allora basta applicare il punto 1 su $G$ per avere $\sum_{x\in T}\res_x G=0$.\\
    Questo punto ci dice che una funzione ellittica ha tanti zeri quanti poli.

    \item Uso il teorema dei residui sulla funzione $z\frac{F'(z)}{F(z)}$ con il cammino $\gamma$.

    Notare che $\res_xz\frac{F'(z)}{F(z)}=x\cdot\res_x\frac{F'}{F}=x\cdot\ord_xF$.
    Infatti, data $g(z) = \sum_{k \in \bbZ} c_k (z - a)^k$ vale $g(z)z=g(z)(z-a)+ag(z)=\sum (z-a)^k(ac_k+c_{k-1})$ e dunque $\res_a (g(z)z) = c_{-2} + a c_{-1} = c_{-2} + a\res_a(g(z))$.
    Se scegliamo $g(z) = \frac{F'(z)}{F(z)}$ per la formula vista prima vale $c_{-2} = 0$.

    Ottengo allora $2\pi i\sum_{x\in T}x\cdot\ord_xf=\int_\gamma z\frac{F'(z)}{F(z)}\ dz=(\int_0^{\omega_1}-\int_{\omega_2}^{\omega_1+\omega_2})+(\int_{\omega_1}^{\omega_1+\omega_2}-\int_{0}^{\omega_2})$

    e per biperiodicità si ha $\int_{\omega_2}^{\omega_1+\omega_2}z\frac{F'(z)}{F(z)}\ dz=\int_0^{\omega_1}(z-\omega_2)\frac{F'(z)}{F(z)}\ dz$ quindi

    $(\int_0^{\omega_1}-\int_{\omega_2}^{\omega_1+\omega_2})=\omega_2\int_0^{\omega_1}\frac{d}{dz}(\log F)\ dz=\omega_22\pi i m$ per qualche $m\in\bbZ$
    (più precisamente, $m$ è il numero di giri che fa $\log F$ intorno a $0$ quando $z$ varia da $0$ a $\omega_1$).

    Analogamente $\int_{\omega_1}^{\omega_1+\omega_2}-\int_{0}^{\omega_2}=\omega_1\cdot 2\pi i n$ con $n\in\bbZ$.

    Semplificando un $2\pi i$ otteniamo $\sum_{x\in T}x\cdot\ord_xf=\omega_1n+\omega_2m\in L$ ovvero è un elemento del reticolo, che è la tesi.
\end{enumerate}
\end{proof}

Vediamo ora un'altra importante proprietà delle funzioni ellittiche, in particolare quelle pari.
\begin{proposizione}\label{fell-pari}
    Sia $f$ una funzione ellittica pari con uno zero in $u$; allora vale $\ord_uf=\ord_{-u}f$ e se $u\equiv-u$ (ovvero $2u\in L$) allora $\ord_u f$ è pari.
\end{proposizione}
\begin{proof}
    Scriviamo $f(z)=(z-u)^mg(z)$ con $g$ olomorfa e $g(u)\neq0$; vale anche $f(z)=f(-z)=(-1)^m(z+u)^mg(-z)$, ovvero anche $-u$ è zero di $f$ con ordine esattamente $m$.\\
    Per la seconda parte consideriamo $f(z)=f(-z)$ e deriviamo ripetutamente per ottenere $f^{(i)}(z)=(-1)^if^{(i)}(-z)$. Se $2u\in L$, allora dato che ogni $f^{(i)}$ è ellittica, vale $f^{(i)}(u)=f^{(i)}(-u)$. Ma questo implica che per $i$ dispari valga $f^{(i)}(u)=0$, e dunque $\ord_u f$ deve essere pari.
\end{proof}


\section{Funzione $\wp$ di Weierstrass}


Fissiamo un toro $T$ e rappresentiamolo come $\bbC/L$ ($L$ reticolo fissato).
\begin{definizione}
    La funzione $\wp$ di Weierstrass è definita dalla serie $$\wp_L(z):=\frac{1}{z^2}+\sum_{\omega\in L\ \omega\not=0}(\frac{1}{(z-\omega)^2}-\frac{1}{\omega^2})$$
    Inoltre la \emph{serie di Eisenstein di peso $k$} è data dalla serie $$G_k(L)=\sum_{\omega\in L\ \omega\not=0}\omega^{-2k}$$
\end{definizione}

Ci occupiamo ora delle questioni di convergenze delle serie nel seguente
\begin{teorema}
    Sia $L$ un reticolo fissato. Allora valgono
    \begin{enumerate}
        \item La serie di Eisenstein $G_k(L)$ è assolutamente convergente per $k>1$
        \item La serie che definisce la funzione $\wp_L(z)$ converge assolutamente e uniformemente sui compatti di $\bbC\setminus L$. La serie definisce quindi una funzione meromorfa su $\bbC$ avente solo poli di ordine $2$ e residuo nullo nei punti del reticolo.
        \item La funzione $\wp_L(z)$ è una funzione ellittica pari.
    \end{enumerate}
\end{teorema}
\begin{proof}$ $
    \begin{enumerate}
        \item Dato che $L$ è discreto, esiste una costante $c=c(L)$ tale che per tutti gli $N$ valga $G_N=\#\{\omega\in L : N\le |\omega|<N+1\}\le cN$. Ma allora possiamo stimare $$\sum_{\omega\in L\ |\omega|\ge1}\dfrac1{|\omega|^{2k}}\le \sum_{N=1}^\infty \dfrac{G_N}{N^{2k}}\le \sum_{N=1}^\infty \dfrac{c}{N^{2k-1}}$$ che converge per $2k-1>1$ ovvero $k>1$.

        \item Osserviamo che per $|\omega|>2|z|$ vale la stima $$ \left|\frac{1}{(z-\omega)^2}-\frac{1}{\omega^2}\right| = \left|\frac{z(2\omega-z)}{\omega^2(z-\omega)^2}\right| \le \frac{\abs{z}(2\abs{\omega}+\abs{z})}{\abs{\omega}^2(\abs{\omega}-\abs{z})^2} \le \frac{10\abs{z}}{\abs{\omega}^3} $$
        Come visto al punto sopra gli $\omega$ di modulo $N$ sono $O(N)$, e quindi $\displaystyle\sum_{\omega\in L\ |\omega|\ge2|z|}\left|\frac{1}{(z-\omega)^2}-\frac{1}{\omega^2}\right|$ è boundata da $\sum_{N\ge1} 10|z|cN\cdot\frac{1}{N^3}=10|z|\sum\frac{c}{N^2}<+\infty$. Quindi c'è convergenza assoluta per ogni $z\not\in L$.

        Allo stesso modo si può vedere che per $K\subseteq\bbC\setminus L$ la serie dei  $\sup_{z\in K}\left|\frac{1}{(z-\omega)^2}-\frac{1}{\omega^2}\right|$ converge, quindi ho convergenza uniforme sui compatti.

        Dunque la serie per $\wp_L(z)$ definisce una funzione olomorfa su $\bbC\setminus L$, e dall'espansione in serie si vede facilmente che i poli sono esattamente nei punti del reticolo e sono di ordine $2$ e residuo $0$.

        \item È chiaro dalla scrittura come serie che $\wp_L(-z)=\wp_L(z)$ poiché basta scambiare $\omega$ con $-\omega$.\\
        Dato che la serie converge uniformemente, possiamo calcolare la derivata termine a termine come $$ \wp'_L(z)=-2\sum_{\omega\in L}\frac1{(z-\omega)^3} $$
        Da questa espressione è chiaro che $\wp'_L$ è una funzione ellittica (è biperiodica perché basta riarrangiare i termini della serie). Fissato ora un $\omega\in L$ possiamo considerare $d_\omega(z)=\wp_L(z+\omega)-\wp_L(z)$ che ha derivata nulla, quindi è costante; cioè vale $\wp_L(z+\omega)=\wp_L(z)+c$ per qualche $c=c(\omega)$ costante. Ponendo $z=-\frac\omega2$ e usando il fatto che $\wp_L$ è pari, otteniamo $c=0$, cioè anche $\wp_L$ è biperiodica, quindi è una funzione ellittica.
    \end{enumerate}
\end{proof}

Possiamo anche passare al quoziente e pensare $\wp:T\rar\bbP_1\bbC$ come funzione olomorfa tra superfici di Riemann.\\
Attenzione: tale definizione dipende dal reticolo $L$ scelto per rappresentare $T$! Però non cambia in modo sostanziale: infatti la $\wp$ trasforma, per un cambio di reticolo $L \mapsto \alpha L$:
\[
 \wp_L(z) = \alpha^2 \wp_{\alpha L}(\alpha z)
\]

In particolare, non vengono modificati né gli zeri né i poli (che saranno le cose che ci interesseranno maggiormente).


\begin{proposizione}
La funzione meromorfa $\wp:T\rar\bbC$ ha le seguenti proprietà:

(i) $\wp(-z)=\wp(z)$.

(ii) $\wp$ ha un polo doppio in $0$ e nessun altro polo.

(iii) Fissato $u\in T$ con $u\not=-u$, si ha $\ord_u(\wp-\wp(u))=\ord_{-u}(\wp-\wp(u))=1$ e $\ord_0(\wp-\wp(u))=-2$ e $\ord_x(\wp-\wp(u))=0$ per $x\not=0,u,-u$.

In altre parole, $\wp-\wp(u)$ ha uno zero semplice in $u$, uno zero semplice in $-u$, un polo doppio in $0$ e nessun altro zero o polo.

(iii') Fissato $u\in T$ con $u=-u\not=0$, si ha $\ord_u(\wp-\wp(u))=2$ e $\ord_0(\wp-\wp(u))=-2$ e $\ord_x(\wp-\wp(u))=0$ per $x\not=0,u$.

In altre parole, $\wp-\wp(u)$ ha uno zero doppio in $u$, un polo doppio in $0$ e nessun altro zero o polo.
\end{proposizione}
\begin{proof}
(i) Ovvio direttamente dalla definizione (riordinando gli addendi).

(ii) Ovvio (perchè sappiamo dove stanno i poli di $P$).

(iii) Prendiamo la funzione meromorfa $\wp-\wp(u):T\rar\bbC$ e usiamo il teorema \ref{residui-funzioni-toriche}: ho un polo di ordine $2$ in $0$, quindi ho esattamente due zeri che sono quelli in $u$ e in $-u$ (che sono punti distinti).

(iii') Come sopra, usando la \ref{fell-pari} per dire che non posso avere $\ord_u(\wp-\wp(u))=1$, perché l'ordine deve essere pari.
\end{proof}


\begin{teorema}
Non esiste un polinomio $A\in\bbC[x]$ tale che $A(\wp):T\rar\bbC$ sia la funzione costantemente nulla.
\end{teorema}
\begin{proof}
Sia $A(x)=(x-a_1)...(x-a_n)$. Se $A(\wp)\equiv0$, allora almeno una tra le $(\wp-a_i)$ dovrebbe essere $\equiv0$, ma $\wp-a_i$ ha un polo doppio in $0$...
\end{proof}


\section{Struttura del campo delle funzioni ellittiche}
Cerchiamo ora di dare una struttura semplice al campo delle funzioni ellittiche su un reticolo $L$, e vediamo perché la fuznione $\wp_L$ è così importante.

\begin{teorema}
Il campo delle funzioni ellittiche pari è $\bbC(\wp)$.
\end{teorema}
\begin{proof}
Prendiamo $f:T\rar\bbC$ meromorfa pari (non costante).

Prendiamo $u\not=-u$ con $\ord_uf\not=0$ (quindi uno zero o un polo). Allora $\ord_{-u}f=\ord_uf$ e possiamo considerare $f\cdot(\wp-\wp(u))^{-\ord_uf}$:
questa è meromorfa pari ed ha gli stessi zeri e poli di $f$ (con le stesse molteplicità) eccetto
in zero ed in $u,-u$ (in cui $\ord_uf\cdot(\wp-\wp(u))^{-\ord_uf}=\ord_{-u}f\cdot(\wp-\wp(u))^{-\ord_uf}=0$).
Sostituiamo $f$ con $f\cdot(\wp-\wp(u))^{-\ord_uf}$.

Prendiamo $u=-u\not=0$ con $\ord_uf\not=0$. Allora $\ord_uf$ è pari (se $u$ è uno zero, osserviamo che $f'$ è dispari e quindi $f'(u)=0$, e analogo per tutte le derivate di ordine dispari;
se $u$ è un polo, facciamo lo stesso ragionamento su $\frac{1}{f}$). Possiamo quindi considerare $f\cdot(\wp-\wp(u))^{-\frac{1}{2}\ord_uf}$:
questa è meromorfa pari ed ha gli stessi zeri e poli di $f$ (con le stesse molteplicità) eccetto in zero ed in $u$ (in cui $\ord_uf\cdot(\wp-\wp(u))^{-\frac{1}{2}\ord_uf}=0$).
Sostituiamo $f$ con $f\cdot(\wp-\wp(u))^{-\frac{1}{2}\ord_uf}$.

Reiterando le operazioni sopra descritte, otteniamo un'uguaglianza del tipo $f\cdot\frac{A(\wp)}{B(\wp)}=g$ con $g$ meromorfa pari senza zeri e poli eccetto eventualmente in $0$.
Ma dal teorema \ref{residui-funzioni-toriche} otteniamo che $g$ non ha uno zero o polo nemmeno in $0$. Ma allora $g$ deve essere costante
(sollevate $g:T\rar\bbC$ ad una $G:\bbC\rar\bbC$: vi viene che $G$ è olomorfa su tutto $\bbC$ e biperiodica, quindi limitata, e quindi costante per Liouville).

Quindi $f\in\bbC(\wp)$, come voluto.
\end{proof}

\begin{teorema}
    Il campo delle funzioni ellittiche è isomorfo a $\bbC(\wp_L(z),\wp'_L(z))$
\end{teorema}
\begin{proof}
    Fissiamo una funzione meromorfa dispari $h:T\rar\bbC$ non costante (ad esempio $\wp'$).
    Essendo $h^2$ pari, si ha $h^2\in\bbC(\wp)$.\\
    Ogni funzione meromorfa $g:T\rar\bbC$ si scrive (in modo unico) come $g_1+g_2$ con $g_1$ meromorfa pari e $g_2$ meromorfa dispari.
    Ma allora $g=g_1+h(\frac{g_2}{h})$ e $\frac{g_2}{h}$ è pari e quindi sta in $\bbC(\wp)$.
    Quindi abbiamo ottenuto che il campo delle funzioni meromorfe su $T$ è $\bbC(\wp,h)$ per una qualsiasi $h$ dispari non costante, e in particolare che la molteplicità di $h$ al denominatore è al più 1.
\end{proof}

\begin{corollario}
    Le funzioni ellittiche con poli solo in $L$ sono polinomi in $\wp_L$ e $\wp'_L$
\end{corollario}



\section{Relazione algebrica tra $\wp$ e $\wp'$}
Osserviamo ora che $(\wp')^2$ è una funzione ellittica pari, e in quanto tale si scrive come funzione razionale di $\wp$:
$(\wp')^2=\frac{A(\wp)}{B(\wp)}$; cerhiamo ora di capire come sono fati $A$ e $B$.

\begin{itemize}
 \item Contiamo i poli in 0. A sinistra ho un polo sestuplo, mentre a destra l'ordine del polo è dato da $2\cdot (\deg A - \deg B)$. Quindi $\deg A = \deg B + 3$.
 \item Contiamo ora gli zeri. A sinistra ho 6 zeri, a destra il numero di zeri è dato da $2*\deg A$: infatti il numeratore contribuisci con zeri in posti diversi da 0 e con poli in zero; il denominatore contribuisce con zeri in 0 (che però vengono mangiati dai poli del denominatore) e con poli altrove. Quindi $\deg A = 3$ e $\deg B=0$
\end{itemize}

In conclusione otteniamo la seguente relazione tra la funzione di Weierstrass e la sua derivata:

$$\wp_L'(z)^2 = A_L(\wp_L(z)) $$

Dove $A_L$ è un polinomio cubico. Cerchiamo di capire le caratteristiche di $A_L$; in particolare
\begin{itemize}
\item Proveremo che $A_L$ non ha radici multiple;
\item Determineremo i coefficienti di $A_L$ in termini del reticolo $L$.
\end{itemize}

Sia $E_L$ il luogo di zeri dell'equazione $y^2=A_L(x)$ in $\bbC^2$, $\tilde{E}_L$ il suo completamento proiettivo.
\begin{proposizione}
Il polinomio $A_L$ non ha radici multiple.
\end{proposizione}

\begin{proof}[Dimostrazione 1]
Evitiamo di esplicitare la dipendenza dal reticolo in $\wp, E,A \dots$ per non appesantire la notazione. Supponiamo per assurdo che $A(t) = (t-c)^2(at+b)$. Allora, sostituendo $\wp, \wp'$ nell'equazione di $E$ si ottiene
$$\wp'(z)^2 = (\wp(z) - c)^2(a\wp(z)+b) $$
$$ \left ( \frac{\wp'(z)}{\wp(z)-c} \right )^2 = a\wp(z)+b $$

Contiamo il numero di poli con molteplicità a destra e a sinistra. A destra ho un polo doppio in $0$. A sinistra ho un polo doppio ogni volta che $\wp - c$ si annulla: infatti se $\ord_{z}(\wp -c) = m > 0$, vale $\ord_{z}(\wp') = m-1$, e perciò $\ord_{z}(LHS) = 2(m-1-m) = -2$. Ma $\wp-c$ ha un polo doppio in zero, dunque ha due zeri (con molteplicità). In definitiva, LHS ha almeno 4 poli mentre RHS ne ha solo 2, assurdo.
\end{proof}

\begin{proof}[Dimostrazione 2]
Riprendiamo l'equazione
$$\wp'(z)^2 = (\wp(z) - c)^2(a\wp(z)+b) $$
E sia $z_0$ uno zero di $\wp-c$. Contiamo gli ordini a destra e a sinistra, indicando con $m:=\ord_{z_0}(\wp-c) > 0$.
A sinistra ho $\ord_{z_0}( \wp'^2) = 2\ord_{z_0}( \wp') = 2(m-1)$. A destra ho $\ord_{z_0}(RHS) = 2m + \ord_{z_0}(a\wp+b) >= 2m$, perchè $a \wp(z_0)+b = ac+b \neq \infty$. Ma $2m> 2(m-1)$, assurdo.
\end{proof}

\begin{proof}[Dimostrazione 3]
Dimostriamo che se $L=\omega_1\bbZ + \omega_2 \bbZ$, gli zeri di $A$ sono $ \wp(\omega_1/2), \wp(\omega_2/2), \wp((\omega_1+\omega_2)/2)$. Questi sono distinti:
\begin{itemize}
\item Dato $z_0 \in \bbC$ diverso da 0, la funzione $\wp(z) - \wp(z_0)$ ha un polo doppio in 0. Visto che il numero di zeri e poli è uguale, avrà due zeri. Per parità di $\wp$, essi sono $\pm z_0$.
\item I complessi $\omega_1/2, \omega_2/2, (\omega_1+\omega_2)/2$ sono a due a due non opposti.
\end{itemize}
Dunque ci basta dimostrare che questi sono effettivamente zeri di $A$ (che ne avrà tre, essendo di grado 3). La funzione $\wp'$ è dispari e periodica di periodo $L$, perciò per ogni $x$ tale che $2x \in L$ si ha
$\wp'(x) = \wp'(x-2x) = \wp'(-x) = -\wp'(x)$
da cui $\wp'(x) = 0$. I numeri $x=\omega_1/2, \omega_2/2, (\omega_1+\omega_2)/2$ sono effettivamente tali che $2x \in L$, perciò zeri di $\wp'$. Per tali $x$ ottengo
$$ 0 = \wp'(x)^2 = A(\wp(x))$$
che $\wp(x)$ è uno zero di $A$, as desired.
\end{proof}

\begin{teorema}$ $
    \begin{enumerate}
        \item La serie di Laurent per $\wp_L(z)$ intorno a $0$ è data da $$ \wp(z)= \frac1{z^2}+\sum_{k=1}^\infty (2k+1)G_{k+1}z^{2k}$$
        \item Il polinomio $A_L(x)$ è della forma $4x^3-g_2x-g_3$, dove $g_2=60G_2$ e $g_3=140G_3$
    \end{enumerate}
\end{teorema}
\begin{proof}
    Questo conto è lungo. Bisogna avere pazienza, alle volte anche mesi. Ma lo portiamo a casa: è una promessa.

    Chiamiamo per comodità $\displaystyle s_n = \sum_{0 \neq \omega \in L} \frac{1}{\omega^n} $

    Sia $L^*$ il reticolo senza lo zero. Richiamiamo l'espressione in serie per la $\wp, \wp'$:
    \begin{eqnarray*}
    \wp(z) & = & \frac{1}{z^2} + \sum_{\omega \in L^*} \frac{1}{(\omega-z)^2} -\frac{1}{\omega^2} \\
    \wp'(z) & = & -\frac{2}{z^3} + \sum_{\omega \in L^*} \frac{2}{(\omega-z)^3} \\
    \end{eqnarray*}
    E le seguenti identità per $|x| < 1$ \notamargine{la seconda e la terza si ottengono derivando le precedenti}:
    \begin{eqnarray*}
    \frac{1}{1-x} & = & \sum_{k \ge 0} x^k \\
    \frac{1}{(1-x)^2} & = & \sum_{k \ge 0} (k+1)x^k \\
    \frac{2}{(1-x)^3} & = & \sum_{k \ge 0} (k+1)(k+2) x^k \\
    \end{eqnarray*}
    Si noti inoltre che $s_n=0$ per $n$ dispari, perchè se $\omega \in L^*$ allora $-\omega \in L^*$.
    Vogliamo intanto sviluppare le $\wp, \wp'$ per $|z| < |\omega|, \ \forall \ \omega \in L^*$. Poi calcoleremo i coefficienti espliciti delle potenze negative di $z$ che compaiono nella serie, e troveremo i coefficienti di $A_L$ comparando le due espressioni $(\wp')^2 = A(\wp)$.
    Vale, per $|z| < |\omega| \ \forall \omega \in L^*$

    $$ \frac1{(z-\omega)^2}-\frac1{\omega^2} = \frac1{\omega^2}\left( \frac1{(1-z/\omega)^2}-1 \right) = \sum_{n\ge1}(n+1)\frac{z^n}{\omega^{n+2}}$$

    Sostituendo nell'espressione per $\wp$, scambiando l'ordine delle sommatorie (posso farlo perché c'è convergenza assoluta) e ricordando che $s_{2k+1}=0$ ottengo

    \begin{eqnarray*}
    \wp(z) & = & \frac{1}{z^2} + \frac1{(z-\omega)^2}-\frac1{\omega^2} = \\
           & = & \frac{1}{z^2} + \sum_{\omega \in L^*} \frac{1}{\omega^2} \sum_{n \ge 1} (n+1) \left ( \frac{z}{\omega} \right ) ^n =  \\
           & = & \frac{1}{z^2} + \sum_{n \ge 1} (n+1) z^n s_{n+2} = \\
           & = & \frac{1}{z^2} + \sum_{\substack{n \text{ pari} \\ n \ge 2}} (n+1)s_{n+2} z^n \\
           & = & \frac{1}{z^2} + \sum_{k\ge1} (2k+1)G_{k+1} z^{2k}
    \end{eqnarray*}

    Per ricavare l'espressione di $\wp'$ possiamo o espandere di nuovo la serie
    \begin{eqnarray*}
    \wp'(z)& = & -\frac{2}{z^3}+ \sum_{\omega \in L^*} \frac{1}{\omega^3} \frac{2}{(1-z/\omega)^3}= \\
           & = & -\frac{2}{z^3}+ \sum_{\omega \in L^*} \frac{1}{\omega^3} \sum_{k \ge 0} (k+1)(k+2) \left ( \frac{z}{\omega} \right ) ^k =\\
           & = & -\frac{2}{z^3} + \sum_{k \ge 0} (k+1)(k+2) z^k s_{k+3} =\\
           & = & -\frac{2}{z^3} + \sum_{\substack{k \text{ dispari} \\ k \ge 1}} (k+1)(k+2) z^k s_{k+3}
    \end{eqnarray*}

    Oppure semplicemente derivare termine a termine la formula trovata per $\wp$ e avere

    $$ \wp'(z) = -\frac{2}{z^3}+\sum_{k\ge1}(2k)(2k+1)G_{k+1}z^{2k-1}$$


    A questo punto scriviamo i primi termini delle serie di Laurent delle varie potenze di $\wp,\wp'$:

    \begin{eqnarray*}
    \wp'(z)^2& = & 4z^{-6}-24G_2z^{-2}-80G_3+O(z) \\
    \wp(z)^3 & = & z^{-6}+9G_2z^{-2}+15G_3+O(z)  \\
    \wp(z) & = & z^{-2}+O(z)
    \end{eqnarray*}

    Ma allora se prendiamo la funzione $$ f(z)= \wp'(z)^2 -4\wp(z)^3+60G_2\wp(z)+140G_3 $$
    questa è olomorfa (nell'espansione in serie compaiono solo potenze positive di $z$) e inoltre $f(0)=0$.\\
    D'altra parte $f$ è una funzione ellittica; ma una funzione ellittica olomorfa è costante, e dunque $f(z)\equiv 0$, ovvero abbiamo trovato la relazione algebrica che cercavamo.

\end{proof}
