% This is a modified version of the tufte-latex book example in which the title page and the contents page resemble Tufte's
% VDQI book, using Kevin Godby's code from this thread at https://groups.google.com/forum/#!topic/tufte-latex/ujdzrktC1BQ.
% Taken from https://www.overleaf.com/5902299yxkgzs#/19519356/
\documentclass[a4paper,oneside]{tufte-book}
\usepackage{nameref}
\usepackage[utf8]{inputenc}
\usepackage[T1]{fontenc}
\usepackage[italian]{babel}
\usepackage{ragged2e}
\usepackage{marginfix}
\usepackage{tikz}
\usepackage{epstopdf}
\usepackage{amsmath}
\usepackage{amssymb}
\usepackage{etoolbox}
\usepackage{afterpage}

% \DeclareSymbolFont{operators}   {OT1}{cmr} {m}{n}
% \DeclareSymbolFont{letters}     {OML}{cmm} {m}{it}
% \DeclareSymbolFont{symbols}     {OMS}{cmsy}{m}{n}
% \DeclareSymbolFontAlphabet{\mathrm}    {operators}
% \DeclareSymbolFontAlphabet{\mathnormal}{letters}
% \DeclareSymbolFontAlphabet{\mathcal}   {symbols}
% \DeclareSymbolFontAlphabet{\mathfrak}  {symbols}
% \DeclareMathAlphabet      {\mathbf}{OT1}{cmr}{bx}{n}
% \DeclareMathAlphabet      {\mathsf}{OT1}{cmss}{m}{n}
% \DeclareMathAlphabet      {\mathit}{OT1}{cmr}{m}{it}
% \DeclareMathAlphabet      {\mathtt}{OT1}{cmtt}{m}{n}
% \DeclareMathAlphabet      {\mathbb}{OT1}{cmr}{m}{n}
% \DeclareMathAlphabet      {\mathfrak}{OT1}{euler}{m}{n}

\usepackage{enumitem}
\setlist{nolistsep}
\setitemize[0]{leftmargin=20pt, itemindent=0pt, labelsep=10pt,
  listparindent=0pt}
\setenumerate[0]{leftmargin=40pt, itemindent=0pt, labelsep=10pt,
  listparindent=0pt}

\geometry{
  a4paper,
  left=18mm,
  textwidth=170mm, % main text block
  marginparsep=0mm, % gutter between main text block and margin notes
  marginparwidth=0mm % width of margin notes
}

% Cambiamo l'impostazione di tufte-common e non mettiamo spazio sopra i
% titoli di inizio capitolo
\titlespacing*{\chapter}{0pt}{0pt}{0pt}

\titleformat{\chapter}%
[hang]%shape
{\scshape\huge\raggedright}%
{\thesection}%
{0.5em}%
{}%
[\vskip 1em]%

% Cambiamo anche l'impostazione di come vengono visualizzati i titoli di
% sezione, li vogliamo in maiuscoletto
\titleformat{\section}%
[hang]% shape
{\scshape\LARGE\raggedright}% format to label + text
{\thesection}% label
{1em}% separation between label and title body
{}% before the title body
[]% after the title body

\titleformat{\subsection}%
[hang]%
{\scshape\Large\raggedright}%
{\thesection}%
{1em}%
{}%
[]%

\RequirePackage[utf8]{inputenc}
\RequirePackage[italian]{babel}
\usepackage{amsmath}
\usepackage{amsthm}
\usepackage{amsfonts}
\usepackage{xifthen}
%\usepackage{xparse}
%\usepackage{etoolbox}% http://ctan.org/pkg/etoolbox
%\usepackage[linktoc=all]{hyperref} % Per i collegamenti ipertestuali. [linktoc=all] mette i link nell'indice

% https://it.sharelatex.com/blog/2011/03/27/how-to-write-a-latex-class-file-and-design-your-own-cv.html

\newcommand{\frdx}{ \framebox[\width]{ $\Rightarrow$ } }
\newcommand{\frsx}{ \framebox[\width]{ $\Leftarrow$ } }
\newcommand{\hrar}{\hookrightarrow}
\newcommand{\opp}{\text{ oppure }}
\def\checkmark{\tikz\fill[scale=0.4](0,.35) -- (.25,0) -- (1,.7) -- (.25,.15) -- cycle;} 
\newcommand{\crossmark}{$\times$}

%mathbb mathcal mathfrak e mathbm per le lettere dell'alfabeto e anche mathbb per quelle greche
\def\mydeflett#1{\expandafter\def\csname bb#1\endcsname{\mathbb{#1}}
		\expandafter\def\csname c#1\endcsname{\mathcal{#1}}
		\expandafter\def\csname k#1\endcsname{\mathfrak{#1}}
		\expandafter\def\csname bl#1\endcsname{\mathbf{#1}}}
\def\mydefalllett#1{\ifx#1\mydefalllett\else\mydeflett#1\expandafter\mydefalllett\fi}
\mydefalllett ABCDEFGHIJKLMNOPQRSTUVWXYZ\mydefalllett

\def\mydeffrakmath#1{\expandafter\def\csname k#1\endcsname{\mathfrak{#1}}}
\def\mydefallfrak#1{\ifx#1\mydefallfrak\else\mydeffrakmath#1\expandafter\mydefallfrak\fi}
\mydefallfrak abcdefghijklmnopqrstuvwxyz\mydefallfrak

\def\mydefgreek#1{\expandafter\def\csname bl#1\endcsname{\text{\boldmath$\mathbf{\csname #1\endcsname}$}}}
\def\mydefallgreek#1{\ifx\mydefallgreek#1\else\mydefgreek{#1}%
   \lowercase{\mydefgreek{#1}}\expandafter\mydefallgreek\fi}
\mydefallgreek {Gamma}{Delta}{Theta}{Lambda}{Xi}{Pi}{Sigma}{Upsilon}{Phi}{Varphi}{Psi}{Omega}{alpha}{beta}{gamma}{delta}{epsilon}{varepsilon}{zeta}{eta}{theta}{iota}{kappa}{lambda}{mu}{nu}{xi}{omicron}{pi}{rho}{sigma}{tau}{upsilon}{phi}{varphi}{chi}{psi}{omega}\mydefallgreek

%\NewDocumentCommand{\de}{gg}{
%\IfNoValueTF{#1}
%		{\text{ d}}
%		{\IfNoValueTF{#2}	{\text{ d}#1}
%			{\frac{\text{d}#1}{\text{d}#2}}
%	}
%}

%\NewDocumentCommand{\dpar}{gg}{
%	\IfNoValueTF{#1}
%		{\partial}
%		{\IfNoValueTF{#2}	{\partial_{#1}}
%			{\frac{\partial {#1}}{\partial {#2}}}
%	}
%}


%nuovi comandi per svariate cose
\newcommand{\sse}{\Leftrightarrow}
\newcommand{\Rar}{\Rightarrow}
\newcommand{\rar}{\rightarrow}
\newcommand{\ol}[1]{\overline{#1}}
\newcommand{\ot}[1]{\widetilde{#1}}
\newcommand{\oc}[1]{\widehat{#1}}
\newcommand{\tc}{\mbox{ t.c. }}

\newcommand{\norma}[1]{\mid\mid #1 \mid\mid}
\newcommand{\abs}[1]{\mid #1 \mid}
\newcommand{\scal}[2]{\langle #1 \mid #2 \rangle}
\newcommand{\floor}[1]{\lfloor #1 \rfloor}

\newcommand{\Ker}{\mbox{Ker } }
\newcommand{\Deg}{\mbox{deg }}
\newcommand{\Det}{\mbox{det }}
\newcommand{\Dim}{\mbox{dim }}
\newcommand{\End}{\mbox{End }}
\newcommand{\Rad}{\mbox{Rad }}
\newcommand{\Ann}{\mbox{Ann }}
\newcommand{\Sp}{\mbox{Sp }}
\newcommand{\Rk}{\mbox{rk }}
\newcommand{\Tr}{\mbox{tr }}
\newcommand{\GL}{\mbox{GL}}
\newcommand{\Isom}{\mbox{Isom}}
\newcommand{\Fix}{\mbox{Fix }}
\newcommand{\Giac}{\mbox{Giac }}
\newcommand{\Ort}{\mbox{O}}
\newcommand{\Aff}{\mbox{Aff }}
\newcommand{\Supp}{\mbox{Supp }}
\newcommand{\Span}{\mbox{Span }}
\newcommand{\Symm}{\mbox{Sym }}
\newcommand{\Asymm}{\mbox{Asym }}
\newcommand{\Img}{\mbox{Im }}
\newcommand{\Id}{\mbox{id}}
\newcommand{\PS}{\mbox{PS }}
\newcommand{\Mtr}{\mathfrak{m}}
\newcommand{\fucknullset}{\{0\}}

% Comando per plottare una curva algebrica
% Nell'ordine i parametri sono:
% 1. Equazione della curva (in due variabili)
% 2. xrange (in notazione [inizio:fine])
% 3. yrange (come sopra)
% DAMETTERE 4. opzioni (tipo se metterci la griglia o no)
\newcommand{\curvegraph}[3]{
    \begin{tikzpicture}
    \draw[very thin,color=gray] (-1.9,-3.9) grid (3.9,3.9);
    \draw[->] (-2,0) -- (4.2,0) node[right] {$x$};
    \draw[->] (0,-4.2) -- (0,4.2) node[above] {$y$};
    \draw plot[id=curve, raw gnuplot, smooth] function{
    f(x,y) = #1;
    set xrange #2;
    set yrange #3;
    set view 0,0;
    set isosample 1000,1000;
    set size square;
    set cont base;
    set cntrparam levels incre 0,0.1,0;
    unset surface;
    splot f(x,y);
    };
    \end{tikzpicture}
}

\sloppy




% Manteniamo le note a piè di pagina nella stessa pagina
\interfootnotelinepenalty=10000
\makeatletter
% Eventualmente si può mettere un \nohyphenation prima di justify ma poi
% le note a margine vengono in maniera terribile.
\renewcommand\@makefntext[1]{\justify\@makefnmark#1}
\makeatother

% Book metadata
\title{Funzioni Ellittiche e Modulari}
\date{Note di un corso del Prof. Umberto Zannier}
\author[]{Classe del Terzo Anno SNS}
\publisher{Versione del \dateitalian\today}

% Just some sample text
%\usepackage{lipsum}

% For nicely typeset tabular material
\usepackage{booktabs}

% For graphics / images
\usepackage{graphicx}
\setkeys{Gin}{width=\linewidth,totalheight=\textheight,keepaspectratio}
\graphicspath{{lezioni/immagini/}}

% The fancyvrb package lets us customize the formatting of verbatim
% environments.  We use a slightly smaller font.
\usepackage{fancyvrb}
\fvset{fontsize=\normalsize}

% Prints argument within hanging parentheses (i.e., parentheses that take
% up no horizontal space).  Useful in tabular environments.
\newcommand{\hangp}[1]{\makebox[0pt][r]{(}#1\makebox[0pt][l]{)}}

% Prints an asterisk that takes up no horizontal space.
% Useful in tabular environments.
\newcommand{\hangstar}{\makebox[0pt][l]{*}}

% Prints a trailing space in a smart way.
\usepackage{xspace}

% Prints the month name (e.g., January) and the year (e.g., 2008)
\newcommand{\monthyear}{%
  \ifcase\month\or January\or February\or March\or April\or May\or June\or
  July\or August\or September\or October\or November\or
  December\fi\space\number\year
}


% Prints an epigraph and speaker in sans serif, all-caps type.
\newcommand{\openepigraph}[2]{%
  %\sffamily\fontsize{14}{16}\selectfont
  \begin{fullwidth}
  \sffamily\large
  \begin{doublespace}
  \noindent\allcaps{#1}\\% epigraph
  \noindent\allcaps{#2}% author
  \end{doublespace}
  \end{fullwidth}
}

% Inserts a blank page
\newcommand{\blankpage}{\newpage\hbox{}\thispagestyle{empty}\newpage}

\usepackage{units}

% Typesets the font size, leading, and measure in the form of 10/12x26 pc.
\newcommand{\measure}[3]{#1/#2$\times$\unit[#3]{pc}}

% Macros for typesetting the documentation
\newcommand{\hlred}[1]{\textcolor{Maroon}{#1}}% prints in red
\newcommand{\hangleft}[1]{\makebox[0pt][r]{#1}}
\newcommand{\hairsp}{\hspace{1pt}}% hair space
\newcommand{\hquad}{\hskip0.5em\relax}% half quad space
\newcommand{\TODO}{\textcolor{red}{\bf TODO!}\xspace}
\newcommand{\ie}{\textit{i.\hairsp{}e.}\xspace}
\newcommand{\eg}{\textit{e.\hairsp{}g.}\xspace}
\newcommand{\na}{\quad--}% used in tables for N/A cells
\providecommand{\XeLaTeX}{X\lower.5ex\hbox{\kern-0.15em\reflectbox{E}}\kern-0.1em\LaTeX}
\newcommand{\tXeLaTeX}{\XeLaTeX\index{XeLaTeX@\protect\XeLaTeX}}
% \index{\texttt{\textbackslash xyz}@\hangleft{\texttt{\textbackslash}}\texttt{xyz}}
\newcommand{\tuftebs}{\symbol{'134}}% a backslash in tt type in OT1/T1
\newcommand{\doccmdnoindex}[2][]{\texttt{\tuftebs#2}}% command name -- adds backslash automatically (and doesn't add cmd to the index)
\newcommand{\doccmddef}[2][]{%
  \hlred{\texttt{\tuftebs#2}}\label{cmd:#2}%
  \ifthenelse{\isempty{#1}}%
    {% add the command to the index
      \index{#2 command@\protect\hangleft{\texttt{\tuftebs}}\texttt{#2}}% command name
    }%
    {% add the command and package to the index
      \index{#2 command@\protect\hangleft{\texttt{\tuftebs}}\texttt{#2} (\texttt{#1} package)}% command name
      \index{#1 package@\texttt{#1} package}\index{packages!#1@\texttt{#1}}% package name
    }%
}% command name -- adds backslash automatically
\newcommand{\doccmd}[2][]{%
  \texttt{\tuftebs#2}%
  \ifthenelse{\isempty{#1}}%
    {% add the command to the index
      \index{#2 command@\protect\hangleft{\texttt{\tuftebs}}\texttt{#2}}% command name
    }%
    {% add the command and package to the index
      \index{#2 command@\protect\hangleft{\texttt{\tuftebs}}\texttt{#2} (\texttt{#1} package)}% command name
      \index{#1 package@\texttt{#1} package}\index{packages!#1@\texttt{#1}}% package name
    }%
}% command name -- adds backslash automatically
\newcommand{\docopt}[1]{\ensuremath{\langle}\textrm{\textit{#1}}\ensuremath{\rangle}}% optional command argument
\newcommand{\docarg}[1]{\textrm{\textit{#1}}}% (required) command argument
\newenvironment{docspec}{\begin{quotation}\ttfamily\parskip0pt\parindent0pt\ignorespaces}{\end{quotation}}% command specification environment
\newcommand{\docenv}[1]{\texttt{#1}\index{#1 environment@\texttt{#1} environment}\index{environments!#1@\texttt{#1}}}% environment name
\newcommand{\docenvdef}[1]{\hlred{\texttt{#1}}\label{env:#1}\index{#1 environment@\texttt{#1} environment}\index{environments!#1@\texttt{#1}}}% environment name
\newcommand{\docpkg}[1]{\texttt{#1}\index{#1 package@\texttt{#1} package}\index{packages!#1@\texttt{#1}}}% package name
\newcommand{\doccls}[1]{\texttt{#1}}% document class name
\newcommand{\docclsopt}[1]{\texttt{#1}\index{#1 class option@\texttt{#1} class option}\index{class options!#1@\texttt{#1}}}% document class option name
\newcommand{\docclsoptdef}[1]{\hlred{\texttt{#1}}\label{clsopt:#1}\index{#1 class option@\texttt{#1} class option}\index{class options!#1@\texttt{#1}}}% document class option name defined
\newcommand{\docmsg}[2]{\bigskip\begin{fullwidth}\noindent\ttfamily#1\end{fullwidth}\medskip\par\noindent#2}
\newcommand{\docfilehook}[2]{\texttt{#1}\index{file hooks!#2}\index{#1@\texttt{#1}}}
\newcommand{\doccounter}[1]{\texttt{#1}\index{#1 counter@\texttt{#1} counter}}

% Generates the index
\usepackage{makeidx}
\makeindex

%%%% Kevin Godny's code for title page and contents from https://groups.google.com/forum/#!topic/tufte-latex/ujdzrktC1BQ
\makeatletter
\renewcommand{\maketitlepage}{%
  \setlength{\parindent}{0pt}
  
  \fontsize{24}{24}\selectfont\textit{\@author}
  
  \vspace{1.75in}\fontsize{36}{54}\selectfont\@title
  
  \vspace{0.5in}\fontsize{14}{14}\selectfont\textsf{\smallcaps{\@date}}
  
  \vfill\fontsize{14}{14}\selectfont\textit{\@publisher}
  
  \thispagestyle{empty}
}
\newcommand{\makefrontcover}{%
% Cerchiamo di includere la copertina con la figura
\afterpage{%
  \newpage%
  \thispagestyle{empty}%
  \newgeometry{margin=0mm, top=0mm, bottom=0mm, left=0mm, right=0mm}
  \begin{textblock*}{\paperwidth}(0pt,0pt)%
    {%
      \transparent{0.4}%
      \begin{center}
        \noindent\includegraphics*[height=\paperheight,width=\paperwidth]{cubicgrouplaw.eps}%
      \end{center}%
    }%
  \end{textblock*}%
  \begin{textblock*}{\paperwidth}(-20mm,63mm)%
    {%
      \transparent{0.90}%
      \crule[white]{2\paperwidth}{50mm}
      %\fbox{\crule[white]{\paperwidth}{50mm}}
    }%
  \end{textblock*}%
  \begin{textblock*}{\paperwidth}(0mm,67mm)%
    {
      \makeatletter
      \begin{center}
        {\fontsize{32}{20}\selectfont\@title}
      \end{center}
      \vspace{0.5in}
      \begin{center}
        {\fontsize{14}{14}\selectfont\textsf{\smallcaps{\@date}}}
      \end{center}
      \makeatother
    }
  \end{textblock*}
  \null%
  \restoregeometry%
  \newpage%
}
}
\makeatother

\titlecontents{part}%
    [0pt]% distance from left margin
    {\addvspace{0.25\baselineskip}}% above (global formatting of entry)
    {\allcaps{Part~\thecontentslabel}\allcaps}% before w/ label (label = ``Part I'')
    {\allcaps{Part~\thecontentslabel}\allcaps}% before w/o label
    {}% filler and page (leaders and page num)
    [\vspace*{0.5\baselineskip}]% after

\titlecontents{chapter}%
    [4em]% distance from left margin
    {}% above (global formatting of entry)
    {\contentslabel{2em}\textit}% before w/ label (label = ``Chapter 1'')
    {\hspace{0em}\textit}% before w/o label
    {\qquad\thecontentspage}% filler and page (leaders and page num)
    [\vspace*{0.5\baselineskip}]% after
%%%% End additional code by Kevin Godby

    \usepackage{amsthm}
    
    \newcounter{gencounter}
    \newcounter{defcounter}
    \newcounter{escounter}

    \linespread{1.2}

    \newtheoremstyle{importante}% <name>
    {\baselineskip}% <Space above>
    {\baselineskip}% <Space below>
    {}% <Body font>
    {}% <Indent amount>
    {\scshape}% <Theorem head font>
    {:}% <Punctation after theorem head>
    {1em}% <Space after theorem head>
    {}% <Theorem head spec>
    
    \theoremstyle{importante}
    \newtheorem{teorema}[gencounter]{Teorema}
    \newtheorem{lemma}[gencounter]{Lemma}
    \newtheorem{definizione}[defcounter]{Definizione}

    \newtheorem{esercizio}[escounter]{Esercizio}
    
    \newtheorem{corollario}[gencounter]{Corollario}
    \newtheorem{osservazione}[defcounter]{Osservazione}
    \newtheorem{remark}[defcounter]{Remark}

    \newcommand{\notamargine}[1]{
      \vspace{-7\baselineskip}
      \let\thefootnote\relax\footnotetext{#1}
      \vspace{7\baselineskip}
    }

    \newcommand\crule[3][black]{\textcolor{#1}{\rule{#2}{#3}}}

    \usepackage{transparent}
    \usepackage[absolute]{textpos}
    
\begin{document}
\makefrontcover

% Front matter
\frontmatter

% r.1 blank page
% \blankpage

% v.2 epigraphs
% \newpage\thispagestyle{empty}
% \openepigraph{%
% The public is more familiar with bad design than good design.
% It is, in effect, conditioned to prefer bad design, 
% because that is what it lives with. 
% The new becomes threatening, the old reassuring.
% }{Paul Rand%, {\itshape Design, Form, and Chaos}
% }
% \vfill
% \openepigraph{%
% A designer knows that he has achieved perfection 
% not when there is nothing left to add, 
% but when there is nothing left to take away.
% }{Antoine de Saint-Exup\'{e}ry}
% \vfill
% \openepigraph{%
% \ldots the designer of a new system must not only be the implementor and the first 
% large-scale user; the designer should also write the first user manual\ldots 
% If I had not participated fully in all these activities, 
% literally hundreds of improvements would never have been made, 
% because I would never have thought of them or perceived 
% why they were important.
% }{Donald E. Knuth}


% r.3 full title page
\maketitle

% v.4 copyright page
\newpage
\newcommand{\contributori}{Nomi di tutti da aggiungere}
\begin{fullwidth}
\begin{doublespace}
  \thispagestyle{empty}
  \setlength{\parindent}{0pt}
  \setlength{\parskip}{\baselineskip}
  \noindent\fontsize{16}{16}\selectfont\scshape
  \par\nohyphenation Hanno collaborato alla stesura di questo testo: \contributori
\end{doublespace}

~\vfill
\thispagestyle{empty}
\setlength{\parindent}{0pt}
\setlength{\parskip}{\baselineskip}
Copyright \copyright\ \the\year\ \contributori

%\par\smallcaps{Pubblicato da \thanklesspublisher}

\par\smallcaps{zannier1617.surge.sh}

\par\justify\nohyphenation This text is licensed under a Creative Commons
\smallcaps{``Attribution-ShareAlike 4.0 International''}
license (the ``License''). You may not use this file
except in compiance with the License. You may obtain a
copy of the License at \url{https://creativecommons.org/licenses/by-sa/4.0/legalcode}.

\includegraphics[width=5em]{by-sa.eps}
\index{license}

%\par\textit{First printing, \monthyear}
\end{fullwidth}

% r.5 contents
\tableofcontents

%\listoffigures
%\listoftables

% r.7 dedication
%\cleardoublepage
%~\vfill
%\begin{doublespace}
%\noindent\fontsize{18}{22}\selectfont\itshape
%\nohyphenation
%Scritto con la collaborazione di tutta la classe del
%Terzo Anno di Matematica della SNS dell' anno 2016-2017
%
%Qui ci andranno i nomi di tutti quanti.
%\end{doublespace}
%\vfill
%\vfill

% r.9 introduction
% \cleardoublepage
%\include{lezioni/lezione-introduzione}

% Start the main matter (normal chapters)
\mainmatter
% Regoliamo ora i margini del documento e lo spazio che lasciamo alle
% note a margine.
\newgeometry{
  left=12mm, % left margin
  textwidth=150mm, % main text block
  marginparsep=6mm, % gutter between main text block and margin notes
  marginparwidth=35mm, % width of margin notes
  headsep=8mm,
  footskip=20pt,
  top=2.5cm,
  bottom=1.5cm,
  showframe
}

% Cambiamo il font per metterlo più piccolo
\fontsize{11}{14}\selectfont

% Mettiamo indentazione e skip tra i paragrafi
\makeatletter
% Paragraph indentation and separation for normal text
\renewcommand{\@tufte@reset@par}{%
  \setlength{\RaggedRightParindent}{1.0pc}%
  \setlength{\JustifyingParindent}{1.0pc}%
  \setlength{\parindent}{0pc}%
  \setlength{\parskip}{0.3\baselineskip}%
}
\@tufte@reset@par

% Paragraph indentation and separation for marginal text
\renewcommand{\@tufte@margin@par}{%
  \setlength{\RaggedRightParindent}{0.5pc}%
  \setlength{\JustifyingParindent}{0.5pc}%
  \setlength{\parindent}{0pc}%
  \setlength{\parskip}{0.3\baselineskip}%
}
\makeatother

\setlength{\parskip}{0cm}
\setlength{\parindent}{0cm}

\chapter{19/09/16 - Introduzione}

In questo corso studieremo le funzioni meromorfe periodiche.
Partiamo dalle funzioni con un solo periodo, che a meno di rinormalizzazioni posso considerare essere $1$.
Lo spazio topologico quoziente $\bbC / \bbZ$ (ovvero $\bbC$ quozientato per la relazione di
equivalenza $a\textit{R}b \sse a-b \in \bbZ$) è un cilindro, e non è compatto.

\begin{definizione}
Si dice una superficie di Riemann una varietà complessa connessa di dimensione $1$.
Si intende che ogni punto deve avere un intorno $U_\alpha$ omeomorfo al disco unitario aperto di $\bbC$
tramite l'omeomorfismo $\phi_\alpha$, e che per ogni $\alpha$ e $\beta$ valga che $g:=\phi_\beta \circ {\phi_\alpha}^{-1}$
sia una funzione olomorfa.
\end{definizione}

\notamargine{La regolarità nei complessi è molto più forte che non nei reali. Essere olomorfe è davvero tanta roba in più che non essere $C^{\infty}$.}

\begin{osservazione}
Più periodi richiedo, più è difficile che la funzione sia anche solo continua. Ad esempio una funzione
$f: \bbR \rar \bbR$ con due periodi incommensurabili è necessariamente costante.
\end{osservazione}

\begin{lemma}
Le funzioni meromorfe con periodo $z_0$ sono tutte e sole quelle della forma $g\left(e^{2\pi i/z_0}\right)$
\end{lemma}

\begin{osservazione}
$\bbC / \bbZ$ è una superficie di Riemann, ma non è omeomorfa a $\bbC$, ad esempio perché $\bbC$ non è semplicemente connesso.
\end{osservazione}

\begin{lemma}
Sia $f: \bbC \rar \bbC$ una funzione meromorfa. Allora l'insieme $L$ dei periodi di $f$
forma un sottogruppo additivo di $\bbC$.
\end{lemma}
\begin{proof}
Esercizio (facile).
\end{proof}

\section{Reticoli}
Sia $f: \bbC \rar \bbC$ una funzione meromorfa non costante.
Sia $L_f$ il gruppo dei periodi di $f$. Allora:
\begin{enumerate}
 \item $L_f$ è discreto
 \item $L_f$ può essere isomorfo solo al gruppo banale, a $\bbZ$ o a $\bbZ^2$ 
\end{enumerate}

\begin{definizione}[Reticolo]
$L_f$ si chiama reticolo di $f$.
\end{definizione}

\begin{definizione}[Funzione ellittica]
Una funzione $f$ si dice ellittica se il suo reticolo ha rango $2$.
\end{definizione}

\notamargine{Tutti i reticoli di rango $2$ sono isomorfi come gruppi, ma la loro
struttura complessa vedremo che sarà completamente diversa.}

\begin{osservazione}
Se $L_f$ ha rango 2, $\bbC / L_f$ è omeomorfo ad un toro, e quindi è compatto.
\end{osservazione}

\paragraph{Finestra sul futuro}
Le funzioni ellittiche "provengono" da equazioni (???)


\begin{definizione}
Dato un polinomio in due (o più) variabili $p\left(x,y\right)$, si dice polinomio omogenizzato il
polinomio $z^{deg\left(p\right)}\cdot p\left(x/z,y/z\right)$
\end{definizione}

\begin{osservazione}
Ho ottenuto un polinomio omogeneo in tre variabili, con soluzioni in $\bbP^2 \left( \bbC \right)$, che è compatto.
\end{osservazione}

\section{Curve ellittiche}
\begin{definizione}[Curva ellittica]
Si dice curva ellittica un sottoinsieme di $\bbC^2$: $E= \left\{\left(x,y \right) \in \bbC^2 | y^2=p \left( x \right) \right\}$, dove $p$ è un polinomio a coefficienti complessi di terzo grado con radici distinte.
\end{definizione}

Le soluzioni di questa equazione coincidono con 
gli zeri della funzione $f \left( x,y \right) = y^2 - p \left( x \right)$.
Per il Teorema del Dini, intorno ad ognuno di questi zeri la curva si riesce ad esprimere come un grafico in almeno una delle due variabili.
(le ipotesi del Teorema sono soddisfatte grazie all'assenza di radici multiple di $p$ ).

\notamargine{ Si chiamano curve ellittiche, perché sono collegate con la lunghezza di archi di ellisse.
Data un'ellisse $y^2 = 1- \alpha x^2 $, per calcolare la lunghezza di un arco si giunge a:
$\int \frac{1-b^2 x}{\sqrt{\left( 1-b^2 x\right)\left(1-a^2 x\right)}}$ }, che con un'opportuna sostituzione...

\section{Le parametriche}
Buco

\section{Trasformazione razionale della curva in sé}
Fissiamo un punto $P$ appartenente alla curva algebrica, sarà la nostra origine. Fissato un qualsiasi altro punto $Q$
appartenente alla curva, consideriamo la retta che passa per $P$ e $Q$. Intersecherà la curva in esattamente un altro punto $R$.
Abbiamo quindi associato al punto $Q$ il punto $R$. Tale trasformazione è chiaramente iniettiva, e si può dimostrare (esercizio)
che è anche razionale (cioè le coordinate di $R$ sono una funzione razionale delle coordinate di $Q$).

\section{Grupi algebrici}
\begin{definizione}[Gruppo algebrico]
Si dice gruppo algebrico un luogo definito da equazioni algebriche 
su un qualche luogo, dotato di una struttura di gruppo razionale.
\end{definizione}

\begin{definizione}
$\bbG_a$ è la retta affine (???).
\end{definizione}

\begin{osservazione}
Sono gruppi algebrici non compatti di dimensione $1$.
\end{osservazione}

\begin{definizione}
$\bbG_m$ è $\bbG_a$ meno un punto, e l'operazione è il prodotto (??? componente per componente?
In $\bbR$ o in $\bbC$??).
\end{definizione}


\paraghaph{Finestra sul futuro}:
Le curve ellittiche saranno tutti e soli i gruppi algebrici di dimensione $2$. Saranno compatti
(moralmente, provengono da dei tori, che sono compatti).




\end{document}

