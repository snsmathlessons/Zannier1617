% This is a modified version of the tufte-latex book example in which the title page and the contents page resemble Tufte's
% VDQI book, using Kevin Godby's code from this thread at https://groups.google.com/forum/#!topic/tufte-latex/ujdzrktC1BQ.
% Taken from https://www.overleaf.com/5902299yxkgzs#/19519356/
\documentclass[nohyper]{tufte-book}
\usepackage{nameref}
% \hypersetup{colorlinks}% uncomment this line if you prefer colored hyperlinks (e.g., for onscreen viewing)
\usepackage[utf8]{inputenc}
\usepackage[italian]{babel}

% Book metadata
\title{Corso di Zannier 2016-2017}
\date{The edition number}
\author[]{Classe del Terzo Anno}
\publisher{Il nostro editore che non abbiamo}

% Just some sample text
\usepackage{lipsum}

% For nicely typeset tabular material
\usepackage{booktabs}

% For graphics / images
\usepackage{graphicx}
\setkeys{Gin}{width=\linewidth,totalheight=\textheight,keepaspectratio}
\graphicspath{{lezioni/immagini/}}

% The fancyvrb package lets us customize the formatting of verbatim
% environments.  We use a slightly smaller font.
\usepackage{fancyvrb}
\fvset{fontsize=\normalsize}

% Prints argument within hanging parentheses (i.e., parentheses that take
% up no horizontal space).  Useful in tabular environments.
\newcommand{\hangp}[1]{\makebox[0pt][r]{(}#1\makebox[0pt][l]{)}}

% Prints an asterisk that takes up no horizontal space.
% Useful in tabular environments.
\newcommand{\hangstar}{\makebox[0pt][l]{*}}

% Prints a trailing space in a smart way.
\usepackage{xspace}

% Prints the month name (e.g., January) and the year (e.g., 2008)
\newcommand{\monthyear}{%
  \ifcase\month\or January\or February\or March\or April\or May\or June\or
  July\or August\or September\or October\or November\or
  December\fi\space\number\year
}


% Prints an epigraph and speaker in sans serif, all-caps type.
\newcommand{\openepigraph}[2]{%
  %\sffamily\fontsize{14}{16}\selectfont
  \begin{fullwidth}
  \sffamily\large
  \begin{doublespace}
  \noindent\allcaps{#1}\\% epigraph
  \noindent\allcaps{#2}% author
  \end{doublespace}
  \end{fullwidth}
}

% Inserts a blank page
\newcommand{\blankpage}{\newpage\hbox{}\thispagestyle{empty}\newpage}

\usepackage{units}

% Typesets the font size, leading, and measure in the form of 10/12x26 pc.
\newcommand{\measure}[3]{#1/#2$\times$\unit[#3]{pc}}

% Macros for typesetting the documentation
\newcommand{\hlred}[1]{\textcolor{Maroon}{#1}}% prints in red
\newcommand{\hangleft}[1]{\makebox[0pt][r]{#1}}
\newcommand{\hairsp}{\hspace{1pt}}% hair space
\newcommand{\hquad}{\hskip0.5em\relax}% half quad space
\newcommand{\TODO}{\textcolor{red}{\bf TODO!}\xspace}
\newcommand{\ie}{\textit{i.\hairsp{}e.}\xspace}
\newcommand{\eg}{\textit{e.\hairsp{}g.}\xspace}
\newcommand{\na}{\quad--}% used in tables for N/A cells
\providecommand{\XeLaTeX}{X\lower.5ex\hbox{\kern-0.15em\reflectbox{E}}\kern-0.1em\LaTeX}
\newcommand{\tXeLaTeX}{\XeLaTeX\index{XeLaTeX@\protect\XeLaTeX}}
% \index{\texttt{\textbackslash xyz}@\hangleft{\texttt{\textbackslash}}\texttt{xyz}}
\newcommand{\tuftebs}{\symbol{'134}}% a backslash in tt type in OT1/T1
\newcommand{\doccmdnoindex}[2][]{\texttt{\tuftebs#2}}% command name -- adds backslash automatically (and doesn't add cmd to the index)
\newcommand{\doccmddef}[2][]{%
  \hlred{\texttt{\tuftebs#2}}\label{cmd:#2}%
  \ifthenelse{\isempty{#1}}%
    {% add the command to the index
      \index{#2 command@\protect\hangleft{\texttt{\tuftebs}}\texttt{#2}}% command name
    }%
    {% add the command and package to the index
      \index{#2 command@\protect\hangleft{\texttt{\tuftebs}}\texttt{#2} (\texttt{#1} package)}% command name
      \index{#1 package@\texttt{#1} package}\index{packages!#1@\texttt{#1}}% package name
    }%
}% command name -- adds backslash automatically
\newcommand{\doccmd}[2][]{%
  \texttt{\tuftebs#2}%
  \ifthenelse{\isempty{#1}}%
    {% add the command to the index
      \index{#2 command@\protect\hangleft{\texttt{\tuftebs}}\texttt{#2}}% command name
    }%
    {% add the command and package to the index
      \index{#2 command@\protect\hangleft{\texttt{\tuftebs}}\texttt{#2} (\texttt{#1} package)}% command name
      \index{#1 package@\texttt{#1} package}\index{packages!#1@\texttt{#1}}% package name
    }%
}% command name -- adds backslash automatically
\newcommand{\docopt}[1]{\ensuremath{\langle}\textrm{\textit{#1}}\ensuremath{\rangle}}% optional command argument
\newcommand{\docarg}[1]{\textrm{\textit{#1}}}% (required) command argument
\newenvironment{docspec}{\begin{quotation}\ttfamily\parskip0pt\parindent0pt\ignorespaces}{\end{quotation}}% command specification environment
\newcommand{\docenv}[1]{\texttt{#1}\index{#1 environment@\texttt{#1} environment}\index{environments!#1@\texttt{#1}}}% environment name
\newcommand{\docenvdef}[1]{\hlred{\texttt{#1}}\label{env:#1}\index{#1 environment@\texttt{#1} environment}\index{environments!#1@\texttt{#1}}}% environment name
\newcommand{\docpkg}[1]{\texttt{#1}\index{#1 package@\texttt{#1} package}\index{packages!#1@\texttt{#1}}}% package name
\newcommand{\doccls}[1]{\texttt{#1}}% document class name
\newcommand{\docclsopt}[1]{\texttt{#1}\index{#1 class option@\texttt{#1} class option}\index{class options!#1@\texttt{#1}}}% document class option name
\newcommand{\docclsoptdef}[1]{\hlred{\texttt{#1}}\label{clsopt:#1}\index{#1 class option@\texttt{#1} class option}\index{class options!#1@\texttt{#1}}}% document class option name defined
\newcommand{\docmsg}[2]{\bigskip\begin{fullwidth}\noindent\ttfamily#1\end{fullwidth}\medskip\par\noindent#2}
\newcommand{\docfilehook}[2]{\texttt{#1}\index{file hooks!#2}\index{#1@\texttt{#1}}}
\newcommand{\doccounter}[1]{\texttt{#1}\index{#1 counter@\texttt{#1} counter}}

% Generates the index
\usepackage{makeidx}
\makeindex

%%%% Kevin Godny's code for title page and contents from https://groups.google.com/forum/#!topic/tufte-latex/ujdzrktC1BQ
\makeatletter
\renewcommand{\maketitlepage}{%
\setlength{\parindent}{0pt}

\fontsize{24}{24}\selectfont\textit{\@author}

\vspace{1.75in}\fontsize{36}{54}\selectfont\@title

\vspace{0.5in}\fontsize{14}{14}\selectfont\textsf{\smallcaps{\@date}}

\vfill\fontsize{14}{14}\selectfont\textit{\@publisher}

\thispagestyle{empty}
}
\makeatother

\titlecontents{part}%
    [0pt]% distance from left margin
    {\addvspace{0.25\baselineskip}}% above (global formatting of entry)
    {\allcaps{Part~\thecontentslabel}\allcaps}% before w/ label (label = ``Part I'')
    {\allcaps{Part~\thecontentslabel}\allcaps}% before w/o label
    {}% filler and page (leaders and page num)
    [\vspace*{0.5\baselineskip}]% after

\titlecontents{chapter}%
    [4em]% distance from left margin
    {}% above (global formatting of entry)
    {\contentslabel{2em}\textit}% before w/ label (label = ``Chapter 1'')
    {\hspace{0em}\textit}% before w/o label
    {\qquad\thecontentspage}% filler and page (leaders and page num)
    [\vspace*{0.5\baselineskip}]% after
%%%% End additional code by Kevin Godby

    \usepackage{amsthm}
    
    \newcounter{gencounter}
    \newcounter{defcounter}
    \newtheorem{teorema}[gencounter]{Teorema}
    \newtheorem{lemma}[gencounter]{Lemma}
    \newtheorem{corollario}[gencounter]{Corollario}
    \newtheorem{definizione}[defcounter]{Definizione}
    \newtheorem{osservazione}[defcounter]{Osservazione}
    \newtheorem{remark}[defcounter]{Remark}
    
\newcommand{\notamargine}[1]{\vspace{-7\baselineskip}\footnotetext{#1}\vspace{7\baselineskip}}

\begin{document}

% Front matter
\frontmatter

% r.1 blank page
% \blankpage

% v.2 epigraphs
% \newpage\thispagestyle{empty}
% \openepigraph{%
% The public is more familiar with bad design than good design.
% It is, in effect, conditioned to prefer bad design, 
% because that is what it lives with. 
% The new becomes threatening, the old reassuring.
% }{Paul Rand%, {\itshape Design, Form, and Chaos}
% }
% \vfill
% \openepigraph{%
% A designer knows that he has achieved perfection 
% not when there is nothing left to add, 
% but when there is nothing left to take away.
% }{Antoine de Saint-Exup\'{e}ry}
% \vfill
% \openepigraph{%
% \ldots the designer of a new system must not only be the implementor and the first 
% large-scale user; the designer should also write the first user manual\ldots 
% If I had not participated fully in all these activities, 
% literally hundreds of improvements would never have been made, 
% because I would never have thought of them or perceived 
% why they were important.
% }{Donald E. Knuth}


% r.3 full title page
\maketitle

% v.4 copyright page
\newpage
\begin{fullwidth}
~\vfill
\thispagestyle{empty}
\setlength{\parindent}{0pt}
\setlength{\parskip}{\baselineskip}
Copyright \copyright\ \the\year\ \thanklessauthor

\par\smallcaps{Pubblicato da \thanklesspublisher}

\par\smallcaps{tufte-latex.googlecode.com}

\par Licensed under the Apache License, Version 2.0 (the ``License''); you may not
use this file except in compliance with the License. You may obtain a copy
of the License at \url{http://www.apache.org/licenses/LICENSE-2.0}. Unless
required by applicable law or agreed to in writing, software distributed
under the License is distributed on an \smallcaps{``AS IS'' BASIS, WITHOUT
WARRANTIES OR CONDITIONS OF ANY KIND}, either express or implied. See the
License for the specific language governing permissions and limitations
under the License.\index{license}

\par\textit{First printing, \monthyear}
\end{fullwidth}

% r.5 contents
\tableofcontents

%\listoffigures
%\listoftables

% r.7 dedication
%\cleardoublepage
%~\vfill
%\begin{doublespace}
%\noindent\fontsize{18}{22}\selectfont\itshape
%\nohyphenation
%Dedicated to those who appreciate \LaTeX{} 
%and the work of \mbox{Edward R.~Tufte} 
%and \mbox{Donald E.~Knuth}.
%\end{doublespace}
%\vfill
%\vfill

% r.9 introduction
\cleardoublepage
%\include{lezioni/lezione-introduzione}

% Start the main matter (normal chapters)
\mainmatter

%% Lezione di esempio. Copiate questo file nella lezione che dovete creare
%% per avere già uno scheletro di come scrivere le lezioni

%% Diamo un nome al capitolo. Idealmente mettiamo la data della lezione ed
%% una sua breve descrizione / argomenti trattati
\chapter{DD/MM/YY - Lezione di esempio}

%% Le parole dentro a \newthought vengono scritte in maiuscoletto. Usatele
%% per sottolineare meglio alcune parole ad inizio paragrafo
\newthought{Queste} poche righe di esempio servono per dare uno scheletro
al modo in cui si dovrebbe scrivere una lezione: useremo un capitolo per
ogni lezione e, all'interno di questi divideremo il testo in vari paragrafi
e/o varie sezioni come si può vedere da questo file.

%% Le cose scritte dentro a \footnotetext appaiono come note al margine
\notamargine{Le scritte dentro all'elemento notamargine appaiono al fianco
  della pagina, come questa scritta. Le si può utilizzare per dare alcune
  spiegazioni su punti che sono poco chiari, o per mettere dei riferimenti
  ad alcuni libri che spiegano meglio l'argomento}

Inoltre potete scrivere le lettere accentate tranquillamente dentro al
sorgente, infatti ho importato il pacchetto opportuno che ci consente di
inserire àèìòùy come se fosse antani e verranno visualizzate bene nel file.

%% Si può poi inserire una section per scrivere degli argomenti specifici
\section{Argomento Rilevante}
Qui siamo ad esempio all'interno di una section e possiamo scrivere anche
alcune formule $\forall x \in X \quad \varphi(x) = x^2$. Oppure anche su
una linea a parte per le cose significative
    $$ \int_a^b \cos(x)\sin^2(x) dx $$

\notamargine{Notate come per separare un $\forall$ dalla parte successiva
  è stato utilizzato un quad, per lasciare un po' di spazio.}

\paragraph{Idea della Dimostrazione} Paragrafetti come questo si possono
utilizzare per dare un'idea della dimostrazione, oppure per far notare un
fatto importante o subdolo che potrebbe sfuggire.

Si possono inoltre inserire delle tabelle come la seguente nel caso ci sia
bisogno di distinguere un po' di casi in maniera ordinata:

\begin{table}[h]
  \caption{Nota a margine della tabella senza senso, che potrebbe contenere
  il titolo della stessa ed anche qualche spiegazione se fosse il caso}
  \begin{center}
    \footnotesize % Gli diciamo di farla in carattere piccolo
    \begin{tabular}{lcr}
      \toprule % Per poter avere la linea sopra la tabella
      Prima colonna & Seconda colonna & Terza Colonna \\
      \midrule % Per la linea a metà, subito sotto gli header
      Misure & Gruppi & Senza senso \\
      Grafi & Integrali & Efelanti \\
      Numeri & Operazioni & Schifo \\
      \bottomrule % Mettiamo una riga anche sotto alla tabella
    \end{tabular}
  \end{center}
\end{table}

\paragraph{Nota} Utilizzate gli elementi teorema, definizione, lemma tra i
tag begin ed end in modo da avere un modo uniforme di scriverli. Sono presenti
gli elementi seguenti: teorema, definizione, lemma, osservazione, remark,
proof, corollario.

\begin{osservazione}
  Scriviamo qualche osservazione importante tipo $2 = 1 + 3$
\end{osservazione}

\begin{lemma}[Lemma del Grande Puffo]
  Potete dare un nome ai lemmi e scriverci davvero dei lemmi dentro tipo
  $a = 1$ che è un importantissimo lemma
\end{lemma}

\begin{teorema}[del Gelato]
  $1 = 1$
\end{teorema}
\begin{proof}
  Ovvia
\end{proof}

%% Le subsection sono, ovviamente, le sottosezioni. State attenti però che
%% con il tipo di testo che stiamo usando le subsubsection non esistono
\subsection{Prime Definizioni}
\begin{Verbatim}
    Potreste voler scrivere qualcosa esattamente come appare, ad esempio
    se ci fosse bisogno di inserire del codice, ma non penso proprio che
    ce ne sia bisogno.
\end{Verbatim}

\subsection{Teorema Principale}


\section{Altro Argomento}
Ormai non so più cosa scrivere, e dovreste aver imparato i principali comandi
utili. Nel caso qualcosa non fosse chiaro non esitate a chiedere.

\notamargine{E non esitate nemmeno a cercare su Google}



\end{document}
